\documentclass{rs}
\usepackage{amsmath,amssymb,amsthm,stmaryrd}
\usepackage[all]{xy}
\usepackage[usenames,dvipsnames]{xcolor}
\usepackage{tikz}
\usepackage{url}
\usepackage{hyperref}
\usepackage{enumerate}
\usepackage{tensor}
%/home/grad/zyf/TeX/inputs/
\usepackage{mathrsfs}
\usepackage{graphicx}
\usepackage{mathtools}
\usetikzlibrary{arrows}
%\usepackage{amsrefs}
%\usepackage{setspace}
%\doublespacing

\title{Teaching statement}
\author{Yifei Zhu}
\givenname{Yifei}
\surname{Zhu}
%\address{Department of Mathematics\\University of Minnesota\\
%         Minneapolis, MN 55455\\USA}
%\email{zyf@math.umn.edu}

%\subject{primary}{msc2000}{55P99}
%\subject{secondary}{msc2000}{55Q99}

%\bibliographystyle{gtart}
\parskip 0.7pc
\parindent 0pt

\newtheorem{thm}[equation]{Theorem}
\newtheorem{cor}[equation]{Corollary}
\newtheorem{prop}[equation]{Proposition}
\newtheorem{lem}[equation]{Lemma}
\theoremstyle{definition}
\newtheorem{defn}[equation]{Definition}
\newtheorem{cstr}[equation]{Construction}
\theoremstyle{remark}
\newtheorem{rmk}[equation]{Remark}
\newtheorem{ex}[equation]{Example}
\newtheorem{case}[equation]{Case}
\newtheorem{slogan}[equation]{Slogan}
\newtheorem{ques}[equation]{Question}

\def\co{\colon\thinspace}
\newcommand{\mb}[1]{\mathbb{#1}}
\newcommand{\mf}[1]{\mathfrak{#1}}
\newcommand{\Hom}{\ensuremath{{\rm Hom}}}
\newcommand{\Aut}{{\rm Aut}}
\newcommand{\LT}{{\rm LT}}
\newcommand{\Spec}{{\rm Spec\thinspace}}
\newcommand{\Proj}{{\rm Proj\thinspace}}
\newcommand{\Spf}{{\rm Spf\thinspace}}
\newcommand{\Ell}{{\rm Ell}}
\newcommand{\Sch}{{\rm Sch}}
\newcommand{\cF}{\overline {\mb F}}
\newcommand{\ck}{\overline k}
\newcommand{\CA}{{\cal A}}
\newcommand{\CB}{{\cal B}}
\newcommand{\CC}{{\cal C}}
\newcommand{\CE}{{\cal E}}
\newcommand{\CF}{{\cal F}}
\newcommand{\CG}{{\cal G}}
\newcommand{\CH}{{\cal H}}
\newcommand{\CHom}{{\cal H}om}
\newcommand{\CLie}{{\cal L}ie}
\newcommand{\CM}{{\cal M}}
\newcommand{\CO}{{\cal O}}
\newcommand{\CP}{{\cal P}}
\newcommand{\CS}{{\cal S}}
\newcommand{\Mod}{{\rm Mod}}
\newcommand{\Alg}{{\rm Alg}}
\newcommand{\dl}{{\rm DL}}
\newcommand{\Set}{{\rm Set}}
\newcommand{\Sq}{{\rm Sq}}
\newcommand{\Sub}{{\rm Sub}}
\newcommand{\Frob}{{\rm Frob}}
\renewcommand{\gcd}{{\rm gcd}}
\newcommand{\cmp}{{\rm cmp}}
\newcommand{\DF}{{{\rm DefFrob}_\BG}}
\newcommand{\Model}{{\rm Model}}
\newcommand{\HGa}{{\widehat{\mb G}_a}}
\newcommand{\HGm}{{\widehat{\mb G}_m}}
\newcommand{\Gm}{{{\mb G}_m}}
\newcommand{\DL}{Dyer-Lashof~}
\newcommand{\EM}{Eilenberg-Mac~Lane~}
\newcommand{\BC}{{\mb C}}
\newcommand{\BE}{{\mb E}}
\newcommand{\BF}{{\mb F}}
\newcommand{\BG}{{\mb G}}
\newcommand{\BN}{{\mb N}}
\newcommand{\BP}{{\mb P}}
\newcommand{\BQ}{{\mb Q}}
\newcommand{\BR}{{\mb R}}
\newcommand{\BW}{{\mb W}}
\newcommand{\BZ}{{\mb Z}}
\newcommand{\fm}{{\mf m}}
\newcommand{\HC}{\widehat{C~}\!}
\newcommand{\HE}{\widehat{E~}\!}
\newcommand{\Hf}{\widehat{f}}
\newcommand{\Hphi}{\widehat{\phi}}
\newcommand{\Hpsi}{\widehat{\psi}}
\newcommand{\HS}{\widehat{S~}\!}
\newcommand{\TA}{\tilde{\A}}
\newcommand{\Tc}{\tilde{c}}
\newcommand{\TE}{\widetilde{E\thinspace}\!}
\newcommand{\Tf}{\widetilde{f}}
\newcommand{\Tp}{\widetilde{\psi}}
\newcommand{\TW}{\widetilde{W\thinspace}\!}
\newcommand{\md}{~~{\rm mod}~}
\newcommand{\ad}{{\rm and}}
\newcommand{\DR}{{\scriptscriptstyle \rm DR}}
\newcommand{\HT}{{\rm ht}}
\newcommand{\id}{{\rm id}}
\newcommand{\op}{{\rm op}}
\newcommand{\pt}{{\rm point}}
\newcommand{\tf}{{\rm tf}}
\newcommand{\TMF}{{\rm TMF}}
\newcommand{\MF}{{\rm MF}}
\newcommand{\tr}{{\rm trace}}
\newcommand{\univ}{{\rm univ}}
\newcommand{\Ext}{{\rm Ext}}
\newcommand{\Tor}{{\rm Tor}}
\newcommand{\nul}{{\rm nul}}
\newcommand{\A}{\alpha}
\newcommand{\B}{\beta}
\renewcommand{\D}{\Delta}
\renewcommand{\d}{\delta}
\newcommand{\f}{\phi}
\newcommand{\G}{\Gamma}
\newcommand{\g}{\gamma}
\newcommand{\K}{\kappa}
\renewcommand{\l}{\lambda}
\newcommand{\si}{\sigma}
\newcommand{\T}{\tau}
\newcommand{\om}{\underline{\omega\!}_{~E/S}}
\newcommand{\p}{\psi^3}
\newcommand{\s}{S^\bullet}
\newcommand{\ce}{\coloneqq}
\newcommand{\lb}{\llbracket}
\newcommand{\rb}{\rrbracket}
\newcommand{\lp}{(\!(}
\newcommand{\rp}{)\!)}
\newcommand{\Ht}{\widehat{T}}
\newcommand{\Tt}{\widetilde{T}}
\newcommand{\mt}{\widetilde{m}}
\newcommand{\lt}{\widetilde{\lambda}}
\newcommand{\todo}{\spadesuit}
\newcommand{\totodo}{\heartsuit}
\renewcommand{\=}{\approx}
\renewcommand{\-}{\sim}
\newcommand{\isog}[1]{Proposition \ref{prop:isog}\thinspace \eqref{isog(#1)}}
\newcommand{\q}[1]{Proposition \ref{prop:Q}\thinspace \eqref{Q(#1)}}
\newcommand{\go}[1]{Definition \ref{def:go}\thinspace \eqref{go(#1)}}
\newcommand{\rd}[1]{{\textcolor{red}{#1}}}
\newcommand{\bl}[1]{{\textcolor{blue}{#1}}}
\newcommand{\wt}[1]{\textcolor{white}{#1} \!~}
\newcommand{\GL}{{\rm GL}}
\newcommand{\SL}{{\rm SL}}
\newcommand{\Tate}{{\rm Tate}}
\renewcommand{\c}[2]{{#1 \choose #2}}

\makeatletter
\DeclareRobustCommand\widecheck[1]{{\mathpalette\@widecheck{#1}}}
\def\@widecheck#1#2{%
    \setbox\z@\hbox{\m@th$#1#2$}%
    \setbox\tw@\hbox{\m@th$#1%
       \widehat{%
          \vrule\@width\z@\@height\ht\z@
          \vrule\@height\z@\@width\wd\z@}$}%
    \dp\tw@-\ht\z@
    \@tempdima\ht\z@ \advance\@tempdima2\ht\tw@ \divide\@tempdima\thr@@
    \setbox\tw@\hbox{%
       \raise\@tempdima\hbox{\scalebox{1}[-1]{\lower\@tempdima\box
\tw@}}}%
    {\ooalign{\box\tw@ \cr \box\z@}}}
\makeatother

\numberwithin{equation}{section}
\renewcommand{\theequation}{\thesection.\arabic{equation}}



\begin{document}

%\begin{abstract}
% I don't have a lot of good advice for teaching statements.  
%\end{abstract}
\maketitle



In her book {\em Sleeping in Temples}, the pianist Susan Tomes writes about 
how the ideas and practices {\em in} music can be effectively communicated 
through teaching, and how vital this particular form of communication is 
{\em to} music as an art and discipline.  Mathematics is also an art and a 
discipline, and I've found that it naturally imposes a rhythm and direction to 
how I can learn better and communicate more.  



\subsection*{``Tracing the curve of the waves with the tip of his violin bow''}

That was how the mighty Hungarian violinist S\'andor V\'egh, looking out to sea, 
often spoke to his young students in Prussia Cove about the ebb and flow of 
music, the type of movement inside a piece.  There seems to me a similar 
mechanism in communicating mathematics.  

Math 230, differential multivariable calculus, is a class I've been teaching 
frequently.  A great feature of it that I like, as I always told my students, is 
that we get to draw a lot of pictures.  In the first class, to motivate the 
passage from 2-D to 3-D, I start by drawing curves---a circle, a parabola, two 
branches of a hyperbola---and rotating each of them to form a surface.  Already 
in the last case, two distinct ``hyperboloids'' are born depending on which axis 
we choose to rotate a hyperbola along.  Now here's a question: What kind of 
surfaces can we obtain from two straight lines?  Over the years, it took varying 
time for the students to ``jump out of the blackboard'' and begin thinking about 
the possibility of two lines that are neither parallel nor intersecting.  
Meanwhile, new ideas and insights kept popping up: ``a washer,'' which I didn't 
initially get, and, this past winter, ``a DNA chain!''  To me, such responses 
were illuminating, both in the math that was being communicated, and in 
{\em how} it could be communicated.  

Recently I taught Math 300, a transition course from calculus to ``higher 
mathematics'' that emphasized how to write proofs and centered on foundational 
notions such as sets, functions, and equivalence relations.  Instead of the 
traditional form of lectures, most of the time it was the students who gave 
presentations of their proofs to flesh out various abstract concepts.  There was 
a livelier exchange of ideas between the speakers and their audience.  To 
visualize partial orderings using trees one has just learned from a graph theory 
course, to imagine an infinite set with greater cardinality as particles that 
are expanding faster, \ldots it was invigorating to see and listen to these 
ideas being explained, questioned, and further explained.  

A new dimension came along: how to {\em present} your math?  By the end of the 
first round of presentations, I encouraged everyone to try and free themselves 
from referring to notes, by extracting and absorbing the key ideas and 
organizing their proofs in a more structural way.  That was actually a 
challenge.  Then, after the second round, I proposed the speakers be more 
relaxed and try to engage their audience better.  Almost immediately, we all 
noticed how changes happened, small and big.  One of the last presentations was 
so well-paced---slow but crystal clear---that it reminded me of the style of a 
mathematician in my field, a charismatic speaker.  It seemed that no one wanted 
to interrupt the talk with a question because we were all taken with the clarity 
and flow.  Besides their substantial work behind this individually (and with 
me), the progress of the class as a whole had to do with the dynamics that 
potentially underlie every experience of learning (and of teaching).  



\subsection*{``It was all part of that great `chain of musicians' \ldots''}

When I first began lecturing, consciously and unconsciously I tried to recall 
what my own teachers had done and how I'd reacted as their student.  I was lucky 
to take a differential geometry course with an inspiring professor, who was a 
student of Shiing-Shen Chern.  One particularly memorable thing from his class 
was the quick drawing of four arcs to form a curved surface---that seemed to be 
the arena on which everything developed.  The professor would draw it over and 
over again, which generated a certain aesthetic aspect as well, and this picture 
was gradually built into my mind so that things around it felt concrete and 
organic.  Now, in the multivariable calculus class I teach, the four-arc surface 
has become an imagery thread that affords and runs through ``tangent planes,'' 
``chain rules,'' ``directional derivatives,'' and ``gradient vectors.''  It made 
its way from his blackboard to mine.  

Sometimes things happened less visibly.  On the first day of Math 300, to give 
an idea of the course, I drew a curve in the shape of $y = e^x$, and flagged the 
point $(0,1)$ as ``We are here''---a transitional stage between the culmination 
of calculus and a vastly increasing amount of more advanced mathematics.  I then 
told the students that actually this picture wasn't quite true, and drew a 
second non-smooth curve in the shape of a heart rate graph that kept climbing 
up.  I explained to them that, given my own experience, the accumulation of 
one's math knowledge and sophistication could be a highly nonlinear and 
rough-edged process.  Specifically, I pointed to the first big jump on the 
graph, which corresponded to my first rigorous training of writing proofs in a 
mathematical analysis course using, in particular, a challenging textbook chosen 
by my professor.  

Preparing for this class brought back to mind my struggle as an undergraduate: 
how I slowly trudged through the exercises in the chapter ``Topology'' of my 
textbook,\footnote{A mimeographed copy of \cite{Browder}.} formulating one proof 
and then another, and how I gradually found the effort deeply rewarding and my 
``future'' subject particularly beautiful.  It also brought to mind the man who 
taught me this book---his physical largeness and his resonant voice, especially 
the accentuated way he pronounced the names of the mathematical giants, ``David 
Hilbert,'' ``Hermann Weyl,'' as he referred every once in a while during his 
lectures.  His sound waves hit me strong enough to indicate that each of those 
names was the tip of an iceberg.  These recollections prompted me to think about 
how to orient my own teaching from a long-term perspective for my students, who 
more or less were in the same phase as I had.  

A surprise came later in Math 300.  One of my students emailed me a six-page 
paper he wrote and typed up, titled ``The exclusiveness of partial and linear 
orderings,'' in which he developed a different system of axioms than the one 
introduced by the textbook, and explained, from several perspectives, the 
intuition behind his axioms.  What he acknowledged as ``challenging me to 
think'' was, honestly, an email I'd written to him after his in-class 
presentation.  In it, I explained my thoughts on points where confusion had 
arisen, in a way a particular professor did to me when I was his student, for 
years.  So it simply echoed when I read \cite[p.151]{Tomes}: 

\begin{quote}
 \em Not so very long afterwards, it seemed, we were recalling his words in a 
 more sober spirit of passing them on to our own students.  We began to 
 understand why it was important to say things over and over again.  And there 
 came a point when we realised that his speech about `knowing you were a link in 
 a chain going back to Brahms' was not just amusing but actually true.  
\end{quote}



% \bibliographystyle{amsalpha}
% \bibliography{me}
% \end{document}

\vspace{.3in}
\renewcommand\refname{}
\newcommand{\AX}[1]{\href{http://arxiv.org/abs/#1}{arXiv:#1}}
\newcommand{\MRn}[2]{\href{http://www.ams.org/mathscinet-getitem?mr=#1}{MR#1#2}}
\wt{.}\vspace{-1.04in}
\begin{thebibliography}

\section*{\leftskip=-.44in References \vspace{.17in}}

\bibitem[Browder1996]{Browder}
Andrew Browder, \emph{Mathematical analysis}, Undergraduate Texts in
  Mathematics, Springer-Verlag, New York, 1996, An introduction. 
  \MRn{1411675}{(97g:00001)}

\bibitem[Tomes2014]{Tomes}
Susan Tomes, \emph{Sleeping in Temples}, The Boydell Press, Woodbridge, 2014.

\end{thebibliography}



\end{document}