\documentclass{rs}
\usepackage{amsmath,amssymb,amsthm,stmaryrd}
\usepackage[all]{xy}
\usepackage[usenames,dvipsnames]{xcolor}
\usepackage{tikz}
\usepackage{url}
\usepackage{hyperref}
\usepackage{enumerate}
\usepackage{tensor}
%/home/grad/zyf/TeX/inputs/
\usepackage{mathrsfs}
\usepackage{graphicx}
\usepackage{mathtools}
\usetikzlibrary{arrows}
%\usepackage{amsrefs}
%\usepackage{setspace}
%\doublespacing

\title{Research statement}
\author{Yifei Zhu}
\givenname{Yifei}
\surname{Zhu}
%\address{Department of Mathematics\\University of Minnesota\\
%         Minneapolis, MN 55455\\USA}
%\email{zyf@math.umn.edu}

%\subject{primary}{msc2000}{55P99}
%\subject{secondary}{msc2000}{55Q99}

%\bibliographystyle{gtart}
\parskip 0.7pc
\parindent 0pt

\newtheorem{thm}[equation]{Theorem}
\newtheorem{cor}[equation]{Corollary}
\newtheorem{prop}[equation]{Proposition}
\newtheorem{lem}[equation]{Lemma}
\theoremstyle{definition}
\newtheorem{defn}[equation]{Definition}
\newtheorem{cstr}[equation]{Construction}
\theoremstyle{remark}
\newtheorem{rmk}[equation]{Remark}
\newtheorem{ex}[equation]{Example}
\newtheorem{case}[equation]{Case}
\newtheorem{slogan}[equation]{Slogan}
\newtheorem{ques}[equation]{Question}

\def\co{\colon\thinspace}
\newcommand{\mb}[1]{\mathbb{#1}}
\newcommand{\mf}[1]{\mathfrak{#1}}
\newcommand{\Hom}{\ensuremath{{\rm Hom}}}
\newcommand{\Aut}{{\rm Aut}}
\newcommand{\LT}{{\rm LT}}
\newcommand{\Spec}{{\rm Spec\thinspace}}
\newcommand{\Proj}{{\rm Proj\thinspace}}
\newcommand{\Spf}{{\rm Spf\thinspace}}
\newcommand{\Ell}{{\rm Ell}}
\newcommand{\Sch}{{\rm Sch}}
\newcommand{\cF}{\overline {\mb F}}
\newcommand{\ck}{\overline k}
\newcommand{\CA}{{\cal A}}
\newcommand{\CB}{{\cal B}}
\newcommand{\CC}{{\cal C}}
\newcommand{\CE}{{\cal E}}
\newcommand{\CF}{{\cal F}}
\newcommand{\CG}{{\cal G}}
\newcommand{\CH}{{\cal H}}
\newcommand{\CHom}{{\cal H}om}
\newcommand{\CLie}{{\cal L}ie}
\newcommand{\CM}{{\cal M}}
\newcommand{\CO}{{\cal O}}
\newcommand{\CP}{{\cal P}}
\newcommand{\CS}{{\cal S}}
\newcommand{\Mod}{{\rm Mod}}
\newcommand{\Alg}{{\rm Alg}}
\newcommand{\dl}{{\rm DL}}
\newcommand{\Set}{{\rm Set}}
\newcommand{\Sq}{{\rm Sq}}
\newcommand{\Sub}{{\rm Sub}}
\newcommand{\Frob}{{\rm Frob}}
\renewcommand{\gcd}{{\rm gcd}}
\newcommand{\cmp}{{\rm cmp}}
\newcommand{\DF}{{{\rm DefFrob}_\BG}}
\newcommand{\Model}{{\rm Model}}
\newcommand{\HGa}{{\widehat{\mb G}_a}}
\newcommand{\HGm}{{\widehat{\mb G}_m}}
\newcommand{\Gm}{{{\mb G}_m}}
\newcommand{\DL}{Dyer-Lashof~}
\newcommand{\EM}{Eilenberg-Mac~Lane~}
\newcommand{\BC}{{\mb C}}
\newcommand{\BE}{{\mb E}}
\newcommand{\BF}{{\mb F}}
\newcommand{\BG}{{\mb G}}
\newcommand{\BN}{{\mb N}}
\newcommand{\BP}{{\mb P}}
\newcommand{\BQ}{{\mb Q}}
\newcommand{\BR}{{\mb R}}
\newcommand{\BW}{{\mb W}}
\newcommand{\BZ}{{\mb Z}}
\newcommand{\fm}{{\mf m}}
\newcommand{\HC}{\widehat{C~}\!}
\newcommand{\HE}{\widehat{E~}\!}
\newcommand{\Hf}{\widehat{f}}
\newcommand{\Hphi}{\widehat{\phi}}
\newcommand{\Hpsi}{\widehat{\psi}}
\newcommand{\HS}{\widehat{S~}\!}
\newcommand{\TA}{\tilde{\A}}
\newcommand{\Tc}{\tilde{c}}
\newcommand{\TE}{\widetilde{E\thinspace}\!}
\newcommand{\Tf}{\widetilde{f}}
\newcommand{\Tp}{\widetilde{\psi}}
\newcommand{\TW}{\widetilde{W\thinspace}\!}
\newcommand{\tj}{\widetilde{j}}
\newcommand{\md}{~~{\rm mod}~}
\newcommand{\ad}{{\rm and}}
\newcommand{\DR}{{\scriptscriptstyle \rm DR}}
\newcommand{\HT}{{\rm ht}}
\newcommand{\id}{{\rm id}}
\newcommand{\op}{{\rm op}}
\newcommand{\pt}{{\rm point}}
\newcommand{\tf}{{\rm tf}}
\newcommand{\TMF}{{\rm TMF}}
\newcommand{\MF}{{\rm MF}}
\newcommand{\tr}{{\rm trace}}
\newcommand{\univ}{{\rm univ}}
\newcommand{\Ext}{{\rm Ext}}
\newcommand{\Tor}{{\rm Tor}}
\newcommand{\nul}{{\rm nul}}
\newcommand{\A}{\alpha}
\newcommand{\B}{\beta}
\renewcommand{\D}{\Delta}
\renewcommand{\d}{\delta}
\newcommand{\f}{\phi}
\newcommand{\G}{\Gamma}
\newcommand{\g}{\gamma}
\newcommand{\K}{\kappa}
\renewcommand{\l}{\lambda}
\newcommand{\si}{\sigma}
\newcommand{\T}{\tau}
\newcommand{\om}{\underline{\omega\!}_{~E/S}}
\newcommand{\p}{\psi^3}
\newcommand{\s}{S^\bullet}
\newcommand{\ce}{\coloneqq}
\newcommand{\lb}{\llbracket}
\newcommand{\rb}{\rrbracket}
\newcommand{\lp}{(\!(}
\newcommand{\rp}{)\!)}
\newcommand{\Ht}{\widehat{T}}
\newcommand{\Tt}{\widetilde{T}}
\newcommand{\mt}{\widetilde{m}}
\newcommand{\lt}{\widetilde{\lambda}}
\newcommand{\todo}{\spadesuit}
\newcommand{\totodo}{\heartsuit}
\renewcommand{\=}{\approx}
\renewcommand{\-}{\sim}
\newcommand{\isog}[1]{Proposition \ref{prop:isog}\thinspace \eqref{isog(#1)}}
\newcommand{\q}[1]{Proposition \ref{prop:Q}\thinspace \eqref{Q(#1)}}
\newcommand{\go}[1]{Definition \ref{def:go}\thinspace \eqref{go(#1)}}
\newcommand{\rd}[1]{{\textcolor{red}{#1}}}
\newcommand{\bl}[1]{{\textcolor{blue}{#1}}}
\newcommand{\wt}[1]{\textcolor{white}{#1} \!~}
\newcommand{\GL}{{\rm GL}}
\newcommand{\SL}{{\rm SL}}
\newcommand{\Tate}{{\rm Tate}}
\renewcommand{\c}[2]{{#1 \choose #2}}

\makeatletter
\DeclareRobustCommand\widecheck[1]{{\mathpalette\@widecheck{#1}}}
\def\@widecheck#1#2{%
    \setbox\z@\hbox{\m@th$#1#2$}%
    \setbox\tw@\hbox{\m@th$#1%
       \widehat{%
          \vrule\@width\z@\@height\ht\z@
          \vrule\@height\z@\@width\wd\z@}$}%
    \dp\tw@-\ht\z@
    \@tempdima\ht\z@ \advance\@tempdima2\ht\tw@ \divide\@tempdima\thr@@
    \setbox\tw@\hbox{%
       \raise\@tempdima\hbox{\scalebox{1}[-1]{\lower\@tempdima\box
\tw@}}}%
    {\ooalign{\box\tw@ \cr \box\z@}}}
\makeatother

\numberwithin{equation}{section}
\renewcommand{\theequation}{\thesection.\arabic{equation}}



\begin{document}

% \begin{abstract}
%  A research statement outlines your current research and your future plans.  
%  It should be relatively short, with some background material that can be 
%  understood by a general mathematician but including some specific statements 
%  about your current and future plans.  Try to be concrete.  
% \end{abstract}
\maketitle

Algebraic topologists attach algebraic structures, such as groups, rings, and 
categories, to geometric objects, such as manifolds, simplicial complexes, and 
even big data sets.  In homotopy theory, the main goal is to study 
{\em invariants} of geometric objects under ``homotopy'' transformations.  This 
type of transformations usually turns out to provide the ``right'' criterion, 
neither too rigid nor too loose, for gaining useful insights to how these 
objects look and behave.  

Effective {\em calculational} tools with the associated algebraic structures 
give homotopy theory its unique flavor.  The main players in realizing this 
process are the various ``cohomology theories,'' each a systematic way of 
attaching specific algebraic structures to geometric objects, and each 
successfully capturing some aspects of the objects in question, while being 
blind to some others \cite{blind}.  A local-to-global property, manifested by 
Mayer-Vietoris sequences, makes cohomology theories particularly amenable to 
computations.  

The study of Morava $E$-theories is like a raindrop in which all of modern 
homotopy theory is reflected.  These cohomology theories are prominent players 
promoted by the ``chromatic viewpoint''---a deep and fruitful relationship 
between homotopy theory and the theory of one-dimensional formal groups that has 
been steadily developing and pervading the field since Quillen's work on complex 
cobordism \cite{Quillen}.  This approach brings homotopy theorists a 
``chromatic'' view (and soundscape) of the stable homotopy category, by 
filtering cohomology theories through heights and primes according to their 
corresponding formal groups.  The family of Morava $E$-theories determines this 
chromatic filtration via Bousfield localizations \cite{Ravenel}.  

Hendrik Lenstra and Peter Stevenhagen wrote that ``nothing can match the clarity 
of a formula when it comes to conveying a mathematical truth.''\footnote{From 
their book review of {\em Solving the Pell equation}, Bull. Amer. Math. Soc. 
(N.S.) \textbf{52} (2015), no.~2, 345--351.}  To understand $E$-theories in the 
specific case of height 2, our research has been centering on calculations with 
their ``power operations.''  These can be viewed as algebraic structures of the 
algebraic structures attached to geometric objects.  They impose restrictions 
and refinements to the algebraic structures carried by cohomology theories, so 
as to enhance their ability to recognize and distinguish the subtleties between 
geometric objects, making those players clear-sighted.  

For $E$-theories at height 2, contacts with algebraic geometry and number 
theory, particularly through the arithmetic moduli of elliptic curves, avail 
homotopy theorists effective tools to carry out explicit calculations.  In terms 
of formulas for power operations, this approach imposes vastly more intricate 
data from arithmetic.  The goal is to make the cohomology theories more 
sensitive and powerful in studying geometric questions, with a potential to 
witness the deep interplay between numbers and spaces.  

We give an outline of our current research and future plans, with an emphasis on 
some specific aspects where algebraic topology, algebraic geometry, and number 
theory interact.  



\section{Elliptic curves: power operation structures at small primes}
\label{sec:p3}

Cohomology operations have been a calculational tool central to algebraic 
topology.  A classical example that has been extensively studied and widely 
applied is the Steenrod operations in ordinary cohomology with 
$\BZ/p$-coefficients.  When $p = 2$, for all integers $i \geq 0$ and $n \geq 0$, 
each Steenrod square takes the form 
$\Sq^i \co H^n(X;\,\BZ/2) \to H^{n + i}(X;\,\BZ/2)$, natural in spaces $X$.  
Together they generate the mod-$2$ Steenrod algebra subject to a set of axioms, 
among which, notably, the Adem relations 
\[
 \Sq^i\,\Sq^{\,j} = \sum_{k = 0}^{\left[\frac{i}{2}\right]} 
                    \c{j - k - 1}{i - 2 k} \Sq^{i + j - k}\,\Sq^k 
 \hskip 2cm 0 < i < 2\,j 
\]
In-depth study of the structure of the Steenrod algebra, and of analogous 
structures for other cohomology theories such as $K$-theory and motivic 
cohomology, has led to spectacular applications: Adams' solution to the problem 
of counting vector fields on spheres \cite{Adams}, and Voevodsky's proof of the 
Milnor conjecture \cite{V1, V2}, just to name two.  

For a Morava $E$-theory, after the foundational work of Ando \cite{Ando95}, 
Strickland \cite{Str97, Str98}, and Ando-Hopkins-Strickland \cite{AHS04}, Rezk 
computed the first example of an explicit presentation for its algebra of power 
operations, in the case of height 2 at the prime 2 \cite{h2p2}.  This algebra is 
generated over $E^0(\pt) \cong \BW\big(\cF_2\big)\lb h \rb$ by operations 
$Q_i \co E^0(X) \to E^0(X)$, $0 \leq i \leq 2$, which, in particular, satisfy 
``Adem relations'' 
\[
  Q_1 Q_0 = 2 Q_2 Q_1 - 2 Q_0 Q_2 \qquad \ad \qquad 
  Q_2 Q_0 = Q_0 Q_1 + h Q_0 Q_2 - 2 Q_1 Q_2 
\]
Unlike their classical analogue, these formulas are computed from a specific 
moduli space of the universal formal deformation of a supersingular elliptic 
curve over $\cF_2$.  Here, bridging algebraic topology and algebraic geometry 
are the work of Ando, Hopkins, Strickland, and Rezk \cite{AHS04, cong} and the 
theorem of Serre and Tate \cite{LST}.  

We systematically studied and generalized Rezk's methods, and have obtained 
analogous results for $E$-theories at the primes 3 and 5.  

\begin{thm}[{\cite[Corollary 2.6 and Definition 3.8]{p3}, 
             \cite[Examples 3.4 and 6.1]{ho}}]
 \label{thm:p3}
 \mbox{}\\
 Let $E$ be a Morava $E$-theory spectrum of height $2$ at the prime $p$.  There 
 is a total power operation 
 \[
  \begin{split}
           \psi^p \co E^0(\pt) \to & ~ E^0(B\Sigma_p) / I \\
   \BW\big(\cF_p\big)\lb h \rb \to & ~ \BW\big(\cF_p\big)\lb h,\A \rb 
                                     / \big( w(h,\A) \big) 
  \end{split}
 \]
 where $I$ is an ideal of transfers.  

 \begin{enumerate}[{\em (i)}]
  \item When $p = 3$, we have $w(h,\A) = \A^4 - 6 \A^2 + (h - 9) \A - 3$ and 
  \[
   \psi^3(h) = h^3 - 27 h^2 + 201 h - 342 + (-6 h^2 + 108 h - 334) \A 
             + (3 h - 27) \A^2 + (h^2 - 18 h + 57) \A^3 
  \]

  \item When $p = 5$, we obtain analogous (but admittedly less readable) 
  formulas.  

  \item These lead to presentations for the respective algebra of power 
  operations on $K(2)$-local commutative $E$-algebra spectra, in terms of 
  explicit generators and quadratic relations.  
 \end{enumerate}
\end{thm}

\begin{ques}
 \label{q:dl}
 Is there a presentation that applies to {\em all} primes for the algebra of 
 power operations in a Morava $E$-theory at height 2?  
\end{ques}

Rezk gave such a uniform presentation, {\em modulo p}, in \cite{mc1}, which 
relies on explicit formulas for the mod-$p$ reduction of a certain moduli space 
of elliptic curves from \cite{KM}.  {\em Integrally}, we will provide an answer 
in Theorem \ref{cor} below.  

We should emphasize that the arithmetic data extracted from the particular 
moduli of elliptic curves cannot be homotopy-theoretically meaningful without 
the aforementioned deep theorems of Ando-Hopkins-Strickland, of Rezk, and of 
Serre-Tate.  In the program of understanding higher chromatic levels, these 
{\em computational} experiments supply tangible raw materials to studying 
{\em structural} properties of the stable homotopy category.  Below are some 
directions that we have investigated and plan to explore further in.  

\begin{enumerate}[(i)]
 \item Rezk shows that the algebra of power operations for a Morava $E$-theory 
 is ``Koszul'' at all heights and primes \cite{Koszul, mc1}.  For height 2 and 
 small primes, the power operation formulas give rise to explicit Koszul 
 complexes.  More generally, the homological algebra of these Koszul complexes 
 has been applied, e.g., to studying Bousfield-Kuhn functors and unstable 
 periodic homotopy groups by Behrens and Rezk \cite{BKTAQ}.  

 \item The power operations at height 2 ``descend'' to height 1 via a 
 $K(1)$-localization \cite[Section 4]{p3}.  At small primes, we observe that the 
 resulting formulas match up numerically to the rings studied by Lubin which 
 parametrize ``canonical'' subgroups of formal groups \cite{can}.  These rings 
 can be explicitly determined, one for each height and prime.  The patterns 
 re-occur in our study of Rezk's logarithms (see Section \ref{sec:ho} below).  
 These suggest a more precise relationship between the first and second 
 chromatic layers from the perspective of power operations and subgroups of 
 formal groups.  

 \item Again, at height 2, the explicit formulas have led to a partial 
 understanding of the ``center'' for the algebra of power operations in an 
 $E$-theory \cite[Theorem 6.8]{ho}.  This is related to the Hecke operators that 
 we discuss next.  
\end{enumerate}



\section{Modular forms: Hecke operators and Rezk's logarithms}
\label{sec:ho}

In \cite{log}, using Bousfield-Kuhn functors, Rezk constructed ``logarithmic'' 
cohomology operations that naturally act on the units of any strictly 
commutative ring spectrum.  In particular, given a Morava $E$-theory spectrum 
$E$ of height $n$ at the prime $p$, he wrote down a formula for its 
``logarithm'' $\ell_{n,\,p} \co E^0(X)^\times \to E^0(X)$ in terms of its power 
operations $\psi_A$ \cite[Theorem 1.11]{log} and he interpreted this formula in 
terms of certain ``Hecke operators'' $T_{j,\,p}$ as follows: 
\begin{equation}
 \label{log}
 \begin{split}
  \ell_{n,\,p}(x) = & ~ \frac{1}{p} \log\Bigg( \prod_{j = 0}^n ~ 
                        \prod_{\stackrel{\scriptstyle 
                        A\,\subset\,(\BQ_p/\BZ_p)^n [p]}{|A| = p^{\,j}}} 
                        \psi_A(x)^{(-1)^{\,j} p^{(\,j - 1) (\,j - 2) / 2}} 
                        \Bigg) \\
                  = & ~ \sum_{j = 0}^n (-1)^{\,j} \, p^{\,j \, (\,j - 1) / 2} \, 
                        T_{j,\,p} (\log x) 
 \end{split}
\end{equation}

These Hecke operators are cohomology operations constructed from power 
operations that were known to act on the $E$-cohomology of a space 
\cite{Ando95}.  Based on explicit calculations and a particular choice of 
parameters in the case of height 2, we revisited these operators to make a 
precise connection with Hecke operators acting on modular forms.  In particular, 
we obtained the following.  

\begin{thm}[{\cite[Proposition 2.8 and Theorem 4.13]{ho}}]
 \label{thm:ho}
 Let $E$ be a Morava $E$-theory spectrum of height $2$ at the prime $p$, and let 
 $N > 3$ be any integer prime to $p$.  
 \begin{enumerate}[{\em (i)}]
  \item There is a ring homomorphism 
  $\B_N^{(p)} \co \MF\big(\G_1(N)\big) \to E^0(\pt)$, where 
  $\MF\big(\G_1(N)\big)$ is the graded ring of weakly holomorphic modular forms 
  on $\G_1(N)$.  

  \item Given $f \in \MF\big(\G_1(N)\big)^{\!\times}$ with trivial Nebentypus, 
  if its Serre derivative $\vartheta \, f = 0$, then $\B_N^{(p)}(\,f)$ is 
  contained in the kernel of the logarithmic operation 
  $\ell_{2,\,p} \co E^0(\pt)^\times \to E^0(\pt)$.  
 \end{enumerate}
\end{thm}

The logarithms in $E$-theories at height 2 are critical in the work of Ando, 
Hopkins, and Rezk on rigidification of the string-bordism elliptic genus 
\cite[Theorem 12.3]{koandtmf}.  Roughly, in their setting, the kernel of a 
logarithm contains the desired genera, which they identified with certain 
Eisenstein series.  

\begin{ques}
 With our result in Theorem \ref{thm:ho} about the kernel of a logarithmic 
 operation, can we develop an analysis of $E_\infty$-orientations analogous to 
 the work of Ando, Hopkins, and Rezk?  
\end{ques}

The logarithm of a meromorphic modular form (on which Hecke operators act) 
appears in Rezk's formula \eqref{log}.  Serre's differential operator 
$\vartheta$ appears in Theorem \ref{thm:ho}.  In view of these, we ask the 
following.  

\begin{ques}
 Do these specific pieces of number theory enter homotopy theory in a 
 {\em structural} way?  For example, do Rezk's logarithmic operations bring in a 
 wider class of automorphic functions to homotopy theory?  What is present at 
 chromatic level higher than 2?  
\end{ques}

In \cite[Section 5]{ho}, we have started investigating certain aspects of the 
aforementioned type of elliptic functions, not totally modular, in the framework 
of ``logarithmic $q$-series'' originally studied by Knopp and Mason 
\cite{KnoppMason}.  It has a curious relationship to mock modular forms 
\cite[Remark 5.2]{ho}.  



\section{Formal groups: modular equations for Lubin-Tate deformations}

Classically, the Kronecker congruence 
\[
 \big(\,\tj - \,j^{\,p}\big) \,\! \big(\,\tj^{\,p} - \,j\,\big) \equiv 0 \md p 
\]
gives a (local) equation, reduced modulo $p$, for the curve that represents 
$[\G_0(p)]$, the moduli problem of finite flat subgroup schemes of rank $p$ for 
elliptic curves.  Indeed, this is precisely the formula that underlies Rezk's 
uniform presentation for the mod-$p$ reduction of the power operation algebra 
(see Section \ref{sec:p3}).  

Strickland studied various moduli problems for formal groups of finite height 
\cite{Str97} and he applied them to the study of power operations in Morava 
$E$-theories \cite{Str98}.  At height 2, we have obtained an integral lift of 
the Kronecker congruence above, in a different pair of parameters.  

\begin{thm}[{\cite[Theorem 1.2]{me}}]
 \label{thm:me}
 Let $\BG_0$ be a formal group over $\cF_p$ of height $2$, and let $\BG$ be its 
 universal deformation.  Write $A_m$ for the ring $\CO_{\Sub_m(\BG)}$ studied in 
 \cite{Str97}, which classifies degree-$p^m$ subgroups of the formal group 
 $\BG$.  In particular, write $A_0 \cong \BW\big(\cF_p\big)\lb h \rb$ according 
 to the Lubin-Tate theorem \cite{LubinTate}.  

 Then the ring $A_1 \cong \BW\big(\cF_p\big)\lb h, \A \rb / \big(w(h,\A)\big)$ 
 is determined by the polynomial 
 \begin{equation}
  \label{w}
  w(h,\A) = (\A - p) \big(\A + (-1)^{\,p}\big)^p 
          - \big(h - p^2 + (-1)^{\,p}\big) \A 
 \end{equation}
 which reduces to $\A (\A^{\,p} - h)$ modulo $p$.  
\end{thm}

This gives an explicit description of $[\G_0(p)]$ at a supersingular point 
(cf.~\cite[Section 7.7]{KM}).  It is {\em not} an equation for the modular curve 
over $\Spec \BZ$, which, as hinted in \cite{RezkMSRI}, might connect to power 
operations for a ``globally equivariant'' elliptic cohomology.  It would be 
interesting to explore this local-global relationship, intertwined by the 
actions of the Morava stabilizer groups and of the modular group, which unites 
the chromatic and equivariant perspectives on homotopy theory.  We have started 
investigating related functorial constructions, such as Witt ring schemes 
\cite{FG}, plethories \cite{BorgerWieland}, and topological modular forms of 
higher level \cite{logetaletmf}.  

Put in the context of homotopy theory, Theorem \ref{thm:me} yields the following 
answer to Question \ref{q:dl}.  

\begin{thm}[{\cite[Theorems 1.6 and 1.7]{me}}]
 \label{cor}
 Continue with the notation in Theorem \ref{thm:p3}.  

 \begin{enumerate}[{\em (i)}]
  \item In the total power operation $\psi^p \co E^0(\pt) \to E^0(B\Sigma_p) / I 
  \cong \BW\big(\cF_p\big)\lb h,\A \rb / \big( w(h,\A) \big)$, the polynomial 
  \[
   w(h,\A) = w_{p + 1} \A^{\,p + 1} + \cdots + w_1 \A + w_0 
   \hskip 2cm w_i \in \BW\big(\cF_p\big)\lb h \rb 
  \]
  can be given as \eqref{w} from Theorem \ref{thm:me} above.  In particular, 
  $w_{p + 1} = 1$, $w_1 = -h$, $w_0 = (-1)^{\,p + 1} p$, and the remaining 
  coefficients 
  \[
   w_i = (-1)^{\,p\,(\,p - i + 1)} \left[ \c{p}{i - 1} + (-1)^{\,p + 1} \, p \, 
         \c{\,p\,}{i} \right] 
  \]

  \item The image $\psi^p(h) = \sum_{i = 0}^p Q_i(h) \, \A^i$ is then given by 
  \[
   \psi^p(h) = \A + \sum_{i = 0}^p \A^i \sum_{\T = 1}^p w_{\T + 1} \, d_{i,\T} 
  \]
  where 
  \[
   d_{i,\T} = \sum_{n = 0}^{\T - 1} (-1)^{\T - n} \, w_0^n 
              \sum_{\stackrel{\scriptstyle m_1 + \cdots + m_{\T - n} = \T + i} 
              {1 \,\leq\, m_s \,\leq\, m_{s + 1} \,\leq\, p + 1}} w_{m_1} \cdots 
              w_{m_{\T - n}} 
  \]
  In particular, $Q_0(h) \equiv h^{\,p} \md p$.  

  \item These lead to a presentation for the algebra of power operations on 
  $K(2)$-local commutative $E$-algebra spectra.  In particular, the generators 
  $Q_i \co E^0(X) \to E^0(X)$ satisfy quadratic relations 
  \[
   Q_k Q_0 = -\sum_{j = 1}^{p - k} w_0^{\,j} \, Q_{k + j} Q_j - \sum_{j = 1}^p 
             \sum_{i = 0}^{j - 1} w_0^i \, d_{k,\,j - i} \, Q_i Q_j 
   \hskip 2cm 1 \leq k \leq p 
  \]
  where the first summation is vacuous if $k = p$.  
 \end{enumerate}
\end{thm}

\begin{ques}
 To study power operations in Morava $E$-theories at height greater than 2, can 
 we generalize Theorem \ref{thm:me} for the ring $A_2$?  How to formulate this 
 for $p$-divisible groups in general?  
\end{ques}

A difficulty for such a generalization lies in the current methods for proving 
Theorem \ref{thm:me}: we argue with $q$-expansions of certain Hauptmoduln and 
their Hecke translates, as in \cite[Example 2.4]{Choi} and 
\cite[Section 1.11]{padicprop}, which are specific to heights 1 and 2.  
Moreover, we are in much need of results from concrete ``computational 
experiments'' at higher chromatic levels, indeed, at height 3 (see 
\cite{prime2, picsurf}).  It would be interesting to work on this.  



% \bibliographystyle{amsalpha}
% \bibliography{me}
% \end{document}

\vspace{.3in}
\renewcommand\refname{}
\newcommand{\AX}[1]{\href{http://arxiv.org/abs/#1}{arXiv:#1}}
\newcommand{\MRn}[2]{\href{http://www.ams.org/mathscinet-getitem?mr=#1}{MR#1#2}}
\wt{.}\vspace{-1.04in}
\begin{thebibliography}

\section*{\leftskip=-.44in References \vspace{.17in}}

\bibitem[Adams1962]{Adams}
J.~F. Adams, \emph{Vector fields on spheres}, Ann. of Math. (2) \textbf{75}
  (1962), 603--632. \MRn{0139178}{(25 \#2614)}

\bibitem[Ando1995]{Ando95}
Matthew Ando, \emph{Isogenies of formal group laws and power operations in the
  cohomology theories {$E\sb n$}}, Duke Math. J. \textbf{79} (1995), no.~2,
  423--485. \MRn{1344767}{(97a:55006)}

\bibitem[Ando-Hopkins-Rezk2010]{koandtmf}
Matthew Ando, Michael~J. Hopkins, and Charles Rezk, \emph{Multiplicative
  orientations of {$KO$}-theory and the spectrum of topological modular forms}, 
  available at \\ \href{http://www.math.uiuc.edu/~mando/papers/koandtmf.pdf}
  {http://www.math.uiuc.edu/\textasciitilde mando/papers/koandtmf.pdf}.

\bibitem[Ando-Hopkins-Strickland2004]{AHS04}
Matthew Ando, Michael~J. Hopkins, and Neil~P. Strickland, \emph{The sigma
  orientation is an {$H\sb \infty$} map}, Amer. J. Math. \textbf{126} (2004),
  no.~2, 247--334. \MRn{2045503}{(2005d:55009)}

\bibitem[Behrens-Rezk2015]{BKTAQ}
Mark Behrens, Charles Rezk, \emph{The Bousfield-Kuhn functor and topological 
  Andr\'e-Quillen cohomology}, available at \,\!
  \href{http://www3.nd.edu/~mbehren1/papers/BKTAQ6.pdf}
  {http://www3.nd.edu/\textasciitilde mbehren1/papers/BKTAQ6.pdf}.

\bibitem[Borger-Wieland2005]{BorgerWieland}
James Borger and Ben Wieland, \emph{Plethystic algebra}, Adv. Math.
  \textbf{194} (2005), no.~2, 246--283. \MRn{2139914}{(2006i:13044)}

\bibitem[Choi2006]{Choi}
D.~Choi, \emph{On values of a modular form on {$\Gamma\sb 0(N)$}}, Acta Arith.
  \textbf{121} (2006), no.~4, 299--311. \MRn{2224397}{(2006m:11051)}

\bibitem[Greenlees1988]{blind}
J.~P.~C. Greenlees, \emph{How blind is your favourite cohomology theory?},
  Exposition. Math. \textbf{6} (1988), no.~3, 193--208.
  \MRn{949783}{(89j:55001)}

\bibitem[Hazewinkel1978]{FG}
Michiel Hazewinkel, \emph{Formal groups and applications}, Pure and Applied
  Mathematics, vol.~78, Academic Press Inc. [Harcourt Brace Jovanovich
  Publishers], New York, 1978. \MRn{506881}{(82a:14020)}

\bibitem[Hill-Lawson2015]{logetaletmf}
Michael Hill and Tyler Lawson, \emph{Topological modular forms with level
  structure}. \AX{1312.7394}

\bibitem[Katz1973]{padicprop}
Nicholas~M. Katz, \emph{{$p$}-adic properties of modular schemes and modular
  forms}, Modular functions of one variable, {III} ({P}roc. {I}nternat.
  {S}ummer {S}chool, {U}niv. {A}ntwerp, {A}ntwerp, 1972), Springer, Berlin,
  1973, pp.~69--190. Lecture Notes in Mathematics, Vol. 350. \MRn{0447119}{(56
  \#5434)}

\bibitem[Katz-Mazur1985]{KM}
Nicholas~M. Katz and Barry Mazur, \emph{Arithmetic moduli of elliptic curves},
  Annals of Mathematics Studies, vol. 108, Princeton University Press,
  Princeton, NJ, 1985. \MRn{772569}{(86i:11024)}

\bibitem[Knopp-Mason2011]{KnoppMason}
Marvin Knopp and Geoffrey Mason, \emph{Logarithmic vector-valued modular
  forms}, Acta Arith. \textbf{147} (2011), no.~3, 261--262. 
  \MRn{2773205}{(2012c:11094)}

\bibitem[Lawson2015]{prime2}
Tyler Lawson, \emph{The {S}himura curve of discriminant 15 and topological
  automorphic forms}, Forum Math. Sigma \textbf{3} (2015), e3, 32.
  \MRn{3324940}{}

\bibitem[Lubin1979]{can}
Jonathan Lubin, \emph{Canonical subgroups of formal groups}, Trans. Amer. Math.
  Soc. \textbf{251} (1979), 103--127. \MRn{531971}{(80j:14039)}

\bibitem[Lubin-Serre-Tate1964]{LST}
J.~Lubin, J.-P. Serre, and J.~Tate, \emph{Elliptic curves and formal groups}, 
  available at \\ \href{http://www.ma.utexas.edu/users/voloch/lst.html}
  {http://www.ma.utexas.edu/users/voloch/lst.html}.

\bibitem[Lubin-Tate1966]{LubinTate}
Jonathan Lubin and John Tate, \emph{Formal moduli for one-parameter formal
  {L}ie groups}, Bull. Soc. Math. France \textbf{94} (1966), 49--59.
  \MRn{0238854}{(39 \#214)}

\bibitem[Meier2014]{picsurf}
Lennart Meier,
  \emph{Concrete examples of Shimura surfaces}, MathOverflow, \\
  \href{http://mathoverflow.net/q/188780}{http://mathoverflow.net/q/188780}
 (version: 2014-12-03).

\bibitem[Quillen1969]{Quillen}
Daniel Quillen, \emph{On the formal group laws of unoriented and complex
  cobordism theory}, Bull. Amer. Math. Soc. \textbf{75} (1969), 1293--1298.
  \MRn{0253350}{(40 \#6565)}

\bibitem[Ravenel1984]{Ravenel}
Douglas~C. Ravenel, \emph{Localization with respect to certain periodic
  homology theories}, Amer. J. Math. \textbf{106} (1984), no.~2, 351--414.
  \MRn{737778}{(85k:55009)}

\bibitem[Rezk2006]{log}
Charles Rezk, \emph{The units of a ring spectrum and a logarithmic cohomology
  operation}, J. Amer. Math. Soc. \textbf{19} (2006), no.~4, 969--1014.
  \MRn{2219307}{(2007h:55006)}

\bibitem[Rezk2008]{h2p2}
Charles Rezk, \emph{Power operations for {M}orava {$E$}-theory of height 2 at 
  the prime 2}. \AX{0812.1320}

\bibitem[Rezk2009]{cong}
Charles Rezk, \emph{The congruence criterion for power operations in {M}orava
  {$E$}-theory}, Homology, Homotopy Appl. \textbf{11} (2009), no.~2, 327--379.
  \MRn{2591924}{(2011e:55021)}

\bibitem[Rezk2012a]{mc1}
Charles Rezk, \emph{Modular isogeny complexes}, Algebr. Geom. Topol. \textbf{12}
  (2012), no.~3, 1373--1403. \MRn{2966690}{}

\bibitem[Rezk2012b]{Koszul}
Charles Rezk, \emph{Rings of power operations for {M}orava {$E$}-theories are
  {K}oszul}. \AX{1204.4831}

\bibitem[Rezk2014]{RezkMSRI}
Charles Rezk, \emph{Calculations in multiplicative stable homotopy theory at
  height 2}, available at \\
  \href{http://www.msri.org/workshops/689/schedules/18240}
  {http://www.msri.org/workshops/689/schedules/18240}.

\bibitem[Strickland1997]{Str97}
Neil~P. Strickland, \emph{Finite subgroups of formal groups}, J. Pure Appl.
  Algebra \textbf{121} (1997), no.~2, 161--208. \MRn{1473889}{(98k:14065)}

\bibitem[Strickland1998]{Str98}
N.~P. Strickland, \emph{Morava {$E$}-theory of symmetric groups}, Topology
  \textbf{37} (1998), no.~4, 757--779. \MRn{1607736}{(99e:55008)}

\bibitem[Voevodsky2003a]{V1}
Vladimir Voevodsky, \emph{Reduced power operations in motivic cohomology}, Publ.
  Math. Inst. Hautes \'Etudes Sci. (2003), no.~98, 1--57.
  \MRn{2031198}{(2005b:14038a)}

\bibitem[Voevodsky2003b]{V2}
Vladimir Voevodsky, \emph{Motivic cohomology with {${\bf Z}/2$}-coefficients},
  Publ. Math. Inst. Hautes \'Etudes Sci. (2003), no.~98, 59--104.
  \MRn{2031199}{(2005b:14038b)}

\bibitem[Zhu2014]{p3}
Yifei Zhu, \emph{The power operation structure on {M}orava {$E$}-theory of
  height 2 at the prime 3}, Algebr. Geom. Topol. \textbf{14} (2014), no.~2,
  953--977. \MRn{3160608}{}

\bibitem[Zhu2015a]{ho}
Yifei Zhu, \emph{The {H}ecke algebra action on {M}orava {$E$}-theory of height
  2}. \AX{1505.06377}

\bibitem[Zhu2015b]{me}
Yifei Zhu, \emph{Modular equations for {L}ubin-{T}ate formal groups at chromatic
  level 2}. \AX{1508.03358}

\end{thebibliography}



\end{document}