\documentclass{rs}
\usepackage{amsmath,amssymb,amsthm,stmaryrd}
\usepackage[all]{xy}
\usepackage{tikz}
\usepackage{url}
\usepackage{hyperref}
\usepackage{enumerate}
\usepackage{tensor}
%/home/grad/zyf/TeX/inputs/
\usepackage{mathrsfs}
\usepackage{graphicx}
\usepackage{mathtools}
%\usepackage{amsrefs}
%\usepackage{setspace}
%\doublespacing

\title{Research statement}
\author{Yifei Zhu}
\givenname{Yifei}
\surname{Zhu}
%\address{Department of Mathematics\\University of Minnesota\\Minneapolis, MN 55455\\USA}
%\email{zyf@math.umn.edu}

%\subject{primary}{msc2000}{55P99}
%\subject{secondary}{msc2000}{55Q99}

%\bibliographystyle{gtart}
\parskip 0.7pc
\parindent 0pt

\newtheorem{thm}{Theorem}
\newtheorem{cor}[thm]{Corollary}
\newtheorem{prop}[thm]{Proposition}
\newtheorem{lem}[thm]{Lemma}
\theoremstyle{definition}
\newtheorem{defn}[thm]{Definition}
\theoremstyle{remark}
\newtheorem{rmk}[thm]{Remark}
\newtheorem{exam}[thm]{Example}
\newtheorem{case}[thm]{Case}
\newtheorem{slogan}[thm]{Slogan}
\newtheorem{ques}[thm]{Question}

\def\co{\colon\thinspace}
\newcommand{\mb}[1]{\mathbb{#1}}
\newcommand{\mf}[1]{\mathfrak{#1}}

\newcommand{\NB}[1]{{\bf (NB: #1)}}

\newcommand{\Ext}{\ensuremath{{\rm Ext}}}
\newcommand{\Coext}{\ensuremath{{\rm Coext}}}
\newcommand{\Hom}{\ensuremath{{\rm Hom}}}
\newcommand{\Tor}{\ensuremath{{\rm Tor}}}
\newcommand{\Ind}{\ensuremath{{\rm Ind}}}
\newcommand{\Res}{\ensuremath{{\rm Res}}}
\newcommand{\colim}{\ensuremath{\mathop{\rm colim}}}
\newcommand{\hocolim}{\ensuremath{\mathop{\rm hocolim}}}
\newcommand{\holim}{\ensuremath{\mathop{\rm holim}}}
\newcommand{\overto}{\mathop\rightarrow}
\newcommand{\overfrom}{\mathop\leftarrow}
\newcommand{\into}{\mathop\hookrightarrow}
\newcommand{\onto}{\mathop\twoheadrightarrow}
\newcommand{\longoverto}{\mathop{\longrightarrow}}
\newcommand{\Irr}{\ensuremath{\mathop{\rm Irr}}}
\newcommand{\Rep}{\ensuremath{\mathop{\rm Rep}}}
\newcommand{\Map}{\ensuremath{{\rm Map}}}
\newcommand{\GL}{{\rm GL}}
\newcommand{\Uni}{{\rm U}}
\newcommand{\BU}{{\rm BU}}
\newcommand{\Sp}{{\cal S}p}
\newcommand{\Sym}{{\rm Sym}}
\newcommand{\thh}{{\rm thh}}
\newcommand{\TAF}{{\rm TAF}}
\newcommand{\TAQ}{{\rm TAQ}}
\newcommand{\End}{{\rm End}}
\newcommand{\Aut}{{\rm Aut}}
\newcommand{\As}{{\cal A}s}
\newcommand{\LT}{{\rm LT}}
\newcommand{\Spec}{{\rm Spec\thinspace}}
\newcommand{\Spf}{{\rm Spf}}
\newcommand{\tmf}{{\rm tmf}}
\newcommand{\eo}{{\rm eo}}
\newcommand{\TMF}{{\rm TMF}}
\newcommand{\Ell}{{\cal E}ll}
\newcommand{\Lie}{{\rm Lie}}
\newcommand{\Sh}{{\rm Sh}}
\newcommand{\HF}{{\rm H}{\mb F}}
\newcommand{\cF}{\overline {\mb F}}
\newcommand{\cQ}{\overline {\mb Q}}

\newcommand{\eilm}[1]{\ensuremath{{\mb H} #1}}
\newcommand{\smsh}[1]{\ensuremath{\mathop{\wedge}_{#1}}}
\newcommand{\tens}[1]{\ensuremath{\mathop{\otimes}_{#1}}}
\newcommand{\susp}{\ensuremath{\Sigma}}
\newcommand{\mapset}[3]{\ensuremath{\left[#2,#3\right]_{#1}}}
\newcommand{\form}[2]{\ensuremath{\left\langle#1,#2\right\rangle}}
\newcommand{\bilin}[2]{\ensuremath{\left(#1,#2\right)}}
\newcommand{\comp}[1]{\ensuremath{#1^\wedge}}
\newcommand{\loca}[3]{\ensuremath{L^{#1}_{#2}(#3)}}
\newcommand{\tc}[3]{\ensuremath{\Omega_{#2 / #1}^{#3}}}
\newcommand{\atc}[3]{\ensuremath{\L_{#2 / #1}^{#3}}}

\newcommand{\xym}[1]{
\vskip 0.7pc
\centerline{\xymatrix{#1}}
\vskip 0.7pc
}

\newcommand{\CA}{{\cal A}}
\newcommand{\Mod}{{\rm Mod}}
\newcommand{\Alg}{{\rm Alg}}
\newcommand{\Frob}{{\rm Frob}}
\newcommand{\DF}{{{\rm DefFrob}_\Gamma}}
\newcommand{\Model}{{\rm Model}}
\newcommand{\Gm}{{{\mb G}_m}}
\newcommand*{\longhookrightarrow}{\ensuremath{\lhook\joinrel\relbar\joinrel\rightarrow}}

\newcommand{\DL}{Dyer-Lashof~}
\newcommand{\BF}{{\mb F}}
\newcommand{\BG}{{\mb G}}
\newcommand{\BP}{{\mb P}}
\newcommand{\BZ}{{\mb Z}}
\newcommand{\HC}{\widehat{C}}
\newcommand{\HS}{\widehat{S}}
\newcommand{\Tf}{\widetilde{f}}
\newcommand{\Tp}{\widetilde{\psi}}
\newcommand{\md}{~~{\rm mod}~}
\newcommand{\ad}{{\rm and}}
\newcommand{\A}{\alpha}
\newcommand{\G}{\Gamma}
\newcommand{\g}{\gamma}
\newcommand{\K}{\kappa}
\newcommand{\p}{\psi^3}
\newcommand{\s}{S^\bullet}
\newcommand{\isog}[1]{Proposition \ref{prop:isog}\thinspace \eqref{isog(#1)}}
\newcommand{\q}[1]{Proposition \ref{prop:Q}\thinspace \eqref{Q(#1)}}
\newcommand{\go}[1]{Definition \ref{def:go}\thinspace \eqref{go(#1)}}


%single word
%end space
%page break
%labels
%newcommands
%.bib
%line break .tex


\begin{document}
%\begin{abstract}
% A research statement outlines your current research and your future plans.  
% It should be relatively short, with some background material that can be understood by a general mathematician 
% but including some specific statements about your current and future plans.  
% Try to be concrete.  
%\end{abstract}


\maketitle

Algebraic topology is a field that solves topological or geometric problems by translating them into algebra.  
It is often aided by an interaction between computations and conceptual interpretations.  
My current research is study cohomology operations in algebraic topology---specifically, 
power operations for elliptic cohomology theories, including Morava $E$-theory of height 2---with connections to algebraic geometry and number theory.  

\section{Background and motivation}
\label{sec:intro}

The problem of vector fields on spheres asks: 
What is the maximum integer $k(n)$ for which there exist $k$ continuous vector fields on the unit sphere in Euclidean $n$-space, 
linearly independent everywhere?  
This was solved by Adams \cite{adamsvector} as a classic application of algebraic topology to geometry.  
The solution was built upon the study of cohomology operations---the work of Steenrod and Whitehead \cite{steenrodwhitehead} 
which used the Steenrod squares to study the same problem, 
and Adams' original attempt to replace the Steenrod squares by 
``cohomology operations of higher kinds.''\footnote{Cf.~\cite[Section 2]{adamsvector}.  
Adams used secondary cohomology operations to solve the Hopf invariant one problem \cite{adamshopf}.  }  
These led to the construction of the Adams operations which were the key tool for solving the vector-field problem for spheres.  

The Steenrod squares and the Adams operations are operations in ordinary cohomology with $\BZ/2$-coefficients and in $K$-theory respectively, 
and both are examples of an important family of cohomology operations called {\em power operations}.  
Foundational work on the Steenrod operations and the {\em Steenrod algebra} $A^*$ 
established a model for studying power operations in other cohomology theories, as encapsulated in \cite[Section 1]{blind}: 
``Steenrod constructed a family of elements of $A^*$, \ldots \cite{steenrod}, 
Adem found some relations between them \cite{adem}, 
Serre showed that Steenrod's elements generated $A^*$ and Adem's relations implied all relations \cite{serre} 
and Milnor elucidated the full structure of $A^*$ \cite{milnor}.  
Nonetheless much remains to be understood about $A^*$ and it is an active area of research.''  

Following earlier work for ordinary cohomology and $K$-theory, 
the study of power operations for other cohomology theories has been carried out 
by tom Dieck and Quillen for complex cobordism \cite{tomdieck, quillenmu}, 
by McClure for $p$-adic $K$-theory \cite{mcclure, H_infty}, 
by Voevodsky for motivic cohomology \cite{V}, 
and by Ando, Hopkins, and Strickland for Morava $E$-theories \cite{AHS04}, among others.  
In particular, complex cobordism, $p$-adic $K$-theory, and Morava $E$-theories, together with ordinary cohomology, 
all fit into {\em the chromatic filtration}, 
an organizing principle for understanding large-scale phenomena in modern stable homotopy theory (cf.~\cite{quillenfgl, orange, tafoverview}).  
Under this framework, cohomology theories are organized according to the {\em heights} of their associated {\em formal group laws}, and also prime by prime: 
ordinary cohomology with rational coefficients at height 0, 
$p$-adic $K$-theory at height 1, complex cobordism and ordinary cohomology with $\BZ/p$-coefficients at height $\infty$, 
and Morava $E$-theories as a family of cohomology theories at height $n$ for each $n$---with $n = 2$ as instances of elliptic cohomology 
\cite{morava, hopkinsmahowald, survey}---and at all primes.  
The chromatic filtration has an intimate connection with algebraic geometry and number theory, 
and studying power operations from this viewpoint brings in new tools and insight from beyond the range of classical algebraic topology.  

In elliptic cohomology, the study of power operations is aided by 
a ``bridge'' connecting homotopy theory and the theory of elliptic curves (see \cite[Theorem B]{cong} and \cite[Theorem 2.9.1]{KM}), 
and is based on work of Ando, Hopkins, and Strickland for the case of supersingular curves (see \cite{AHS04}), 
and work of Ando and Ganter for the case of the Tate curve (see \cite{andotrans, ganterTate}).  
Thanks to the correspondence across this bridge, 
explicit constructions and computations for power operations flow from calculations with isogenies of elliptic curves (see \cite{h2p2}).  
With concrete formulas, we hope to study the structure underlying power operations in elliptic cohomology, 
along the lines of understanding the Steenrod algebra $A^*$ for ordinary cohomology.  
Moreover, as explained at the beginning of \cite{andoduke}, 
we hope to learn about the conjectural geometric interpretation of Morava $E$-theories (at height $n = 2$) by examining power operations, 
analogous to vector bundles and representation theory as encoded in the Adams operations for $K$-theory (cf.~\cite{adamsvector, atiyah}).  


\section{Current research}

\subsection{The power operation structure on Morava $E$-theory of height 2 at the prime 3}
\label{subsec:h2p3}

Let $p$ be a prime, and $q$ be a power of $p$.  
We use the symbols $\BF_q$ and $\BZ_q$ to denote a field with $q$ elements 
and the ring of $p$-typical Witt vectors over $\BF_q$ respectively.  
In particular $\BZ_p$ is the ring of $p$-adic integers.  

For a Morava $E$-theory, the analog of the Steenrod algebra in ordinary cohomology is 
the {\em \DL algebra} (modulo instability relations) as the collection of all (additive) power operations.  
In \cite{h2p2} Rezk explicitly computes the \DL algebra for a specific $E$-theory of height 2 at the prime 2.  
Following his work, at height 2 for the prime 3, 
we studied the power operation structure on an $E$-theory $E$ whose homotopy groups are 
\[
 \pi_* E = E_* \cong \BZ_9 \llbracket h \rrbracket [u^{\pm 1}] 
\]
with $h$ and $u$ in degrees 0 and 2 respectively \cite[Sections 2 and 3]{p3}.  

As in \cite{h2p2}, our computation of power operations follows the approach of \cite{steenrod}: 
one first defines a total power operation, 
and then uses the computation of the cohomology of the classifying space $B\Sigma_m$ for the symmetric group $\Sigma_m$ to obtain individual power operations.  
By doing calculations with elliptic curves associated to $E$, 
we get formulas for a total power operation $\p$ on $E_0$ and a ring $S_3$ which represents a corresponding moduli problem.  
Based on the computation of $E^* B\Sigma_m$ in \cite{Str98} as reflected in the formula for $S_3$, 
we then define individual power operations $Q_k$, and derive the relations they satisfy---action on scalars, additivity, 
commutation relations, Adem relations, Cartan formulas, and the Frobenius congruence.  
In view of the general structure studied in \cite{cong}, we get an explicit description of the corresponding \DL algebra $\G$ as below.  

\begin{defn}
\label{def}
 \mbox{}
 \begin{enumerate}[(i)]
  \item \label{gamma} Let $i$ be an element generating $\BZ_9$ over $\BZ_3$ with $i^2 = -1$.  
  Define $\G$ to be the associative ring generated over $\BZ_9 \llbracket h \rrbracket$ 
  by elements $Q_0$, $Q_1$, $Q_2$, and $Q_3$ subject to the following relations: 
  the $Q_k$'s commute with elements in $\BZ_3 \subset \BZ_9 \llbracket h \rrbracket$, 
  and satisfy {\em commutation relations} 
  \begin{equation*}
  \begin{split}
   Q_0 h = & ~ (h^3 - 27 h^2 + 201 h - 342) Q_0 + (3 h^2 - 54 h + 171) Q_1 + (9 h - 81) Q_2 ~ \\
           & + 24 Q_3, 
  \end{split}
  \end{equation*}
  \begin{equation*}
  \begin{split}
   Q_1 h = & ~ (-6 h^2 + 108 h - 334) Q_0 + (-18 h + 171) Q_1 + (-72) Q_2 + (h - 9) Q_3, \\
   Q_2 h = & ~ (3 h - 27) Q_0 + 8 Q_1 + 9 Q_2 + (-24) Q_3, \\
   Q_3 h = & ~ (h^2 - 18 h + 57) Q_0 + (3 h - 27) Q_1 + 8 Q_2 + 9 Q_3, \\
   Q_k i ~ = & ~ (-i) Q_k \text{~for all~} k, 
  \end{split}
  \end{equation*}
  and {\em Adem relations} 
  \begin{equation*}
  \begin{split}
   Q_1Q_0 = & ~ (-6) Q_0Q_1 + 3 Q_2Q_1 + (6 h - 54) Q_0Q_2 + 18 Q_1Q_2 + (-9) Q_3Q_2 \\
            & + (-6 h^2 + 108 h - 369) Q_0Q_3 + (-18 h + 162) Q_1Q_3 + (-54) Q_2Q_3, ~ \\
   Q_2Q_0 = & ~ 3 Q_3Q_1 + (-3) Q_0Q_2 + (3 h - 27) Q_0Q_3 + 9 Q_1Q_3, \\
   Q_3Q_0 = & ~ Q_0Q_1 + (-h + 9) Q_0Q_2 + (-3) Q_1Q_2 + (h^2 - 18 h + 63) Q_0Q_3 \\
            & + (3 h - 27) Q_1Q_3 + 9 Q_2Q_3.  
  \end{split}
  \end{equation*}

  \item \label{omega} Write $\omega \coloneqq \pi_2 E$, viewed as a free 
  module with one generator $u$ over 
  $E_0 \cong \BZ_9 \llbracket h \rrbracket$.  Define $\omega$ as a left 
  $\G$-module, compatible with its $E_0$-module structure, by 
  \[
   Q_k \cdot u \coloneqq \left\{
   \begin{array}{ll}
     u,  & \quad {\rm if}~k = 1, \\
     0,  & \quad {\rm if}~k \neq 1.  
   \end{array}
   \right.
  \]
 \end{enumerate}
\end{defn}
The relations in Definition \ref{def}\thinspace \eqref{gamma}, together with additivity, Cartan formulas, and a natural grading, 
describe explicitly the structure of $\G$ as that of a {\em graded twisted bialgebra over $E_0$} in the sense of \cite[Section 5]{cong}.  

Our computation is motivated by Rezk's result \cite{cong, h2p2} on the general pattern of power operations for Morava $E$-theory spectra.  
The formulas in Definition \ref{def} give the following concrete version of his theorem, at height 2 for the prime 3.  
(See \cite{cong, h2p2} for details about the notation and terminology.)  

\begin{thm}
 Let $R$ be a $K(2)$-local commutative $E$-algebra spectrum.  
 Let $\G$ be the graded twisted bialgebra over $E_0$ in Definition \ref{def}\thinspace \eqref{gamma}, 
 and $\omega$ be the $\G$-module in Definition \ref{def} \eqref{omega}.  
 Then $R_*$ has the structure of an {\em $\omega$-twisted $\BZ/2$-graded amplified $\G$-ring} in the sense of \cite[Section 2]{cong} and \cite[2.5 and 2.6]{h2p2}.  
 In particular, for a free commutative $E$-algebra spectrum $\BP_E (\Sigma^d E)$, 
 \[
  \pi_* L_{K(2)} \BP_E (\Sigma^d E) \cong \big( F_d \big)_{(3,h)}^\wedge 
 \]
 where $F_d$ is the free graded amplified $\G$-ring with one generator in dimension $d$.  
\end{thm}

With explicit formulas, we hope to get a real understanding of this structure, and it is one of our main research goals (see Section \ref{sec:plans} below).  

The calculations in \cite{h2p2} at the prime 2 provide a model for computing power operations in Morava $E$-theories of height 2.  
At the prime 3, along with extra difficulty in some aspects of the computation, there also arise patterns previously hidden beneath the formulas at the prime 2.  
Our attempt for the above calculations involves choices of universal families of elliptic curves for the $E$-theory $E$, 
choices of affine coordinate charts both for elliptic curves and for the moduli stack associated to the universal family, 
and choices of parameters for the total power operation $\p$.  
Different choices result in different formal group laws of $E$ and different bases of the \DL algebra $\G$.  
Keeping in mind the comparison with Rezk's calculations at the prime 2, 
we seek for choices which are both geometrically interesting and computationally convenient.  
Here is a brief summary.  
\begin{enumerate}[(i)]
 \item For elliptic curves, the affine coordinate chart $\{u = X/Y, v = Z/Y\}$ 
 carries (formal) geometric information relevant to the construction of power operations.  

 \item We need to do calculations with the universal elliptic curve over {\em all} of the moduli stack, 
 and then pass these to an affine coordinate chart containing the supersingular locus 
 and get formulas for power operations.  

 \item \label{A} There is a parameter $\A$ coming from the relative cotangent space of the elliptic curve at the identity 
 which is convenient for deriving Adem relations.  
\end{enumerate}
In particular, closely related to (iii), we studied a commutative diagram of elliptic curves over the ring $S_3$.  
It involves a composite of deformations of the third-power Frobenius endomorphism 
on a supersingular elliptic curve over the field $\BF_9 \cong S_3 / (3,h,\A)$.  
This diagram brings in geometric intuition and clarifies our derivation of Adem relations.  
Formulas for Adem relations flow from an explicitation of the endomorphism of a moduli scheme 
which is induced by an involution of the corresponding moduli problem (cf.~\cite[11.3.1]{KM} and see Section \ref{sec:plans} below).  


\subsection{Power operations on the $K(1)$-localization}

On the $K(1)$-localization $L$ of our Morava $E$-theory $E$, 
the power operation structure is simpler: 
the \DL algebra has a single generator $\p_L$ over the ring 
\[
 L_0 \cong \BZ_9 \llbracket h \rrbracket [h^{-1}]_3^\wedge = \left.\left\{\sum_{n = -\infty}^{\infty} k_n h^n~\right|~k_n \in \BZ_9, \lim_{n \to -\infty} k_n = 0\right\}.  
\]
We derived formulas for this $K(1)$-local power operation $\p_L$ 
from the calculations in Section \ref{subsec:h2p3} \cite[Section 4]{p3}.  
Based on \cite[Sections 2.4 and 8.2]{level3}, 
it boils down to interpreting the $K(1)$-local power operations 
in terms of the elliptic curve and formal group data 
for our construction of the total power operation $\p$.  
Here are the formulas.  

\begin{prop}
\label{prop}
 The total power operation 
 \[
  \p \co E^0 \to E^0 [\A] \big/ \big( w(\A) \big) 
 \]
 is given by 
 \begin{equation*}
 \begin{split}
  \p(h) = & ~ h^3 - 27 h^2 + 201 h -342 + (-6 h^2 + 108 h - 334) \A + (3 h - 27) \A^2 \\
          & + (h^2 - 18 h + 57) \A^3, \\
  \p(i) \thinspace = & -i, 
 \end{split}
 \end{equation*}
 where 
 \begin{equation}
 \label{w}
  w(\A) = \A^4 - 6 \A^2 + (h - 9) \A - 3.  
 \end{equation}
 Correspondingly the power operation 
 \[
  \p_L \co L^0 \to L^0 
 \]
 is given by 
 \begin{equation*}
 \begin{split}
  \psi_L^3(h) = & ~ h^3 - 27 h^2 + 183 h - 180 + 186 h^{-1} + 1674 h^{-2} + (\text{lower-order terms}), \\
  \psi_L^3(i) \thinspace = & -i.  
 \end{split}
 \end{equation*}
\end{prop}


\section{Future plans}
\label{sec:plans}

A uniform presentation of the $E$-\DL algebra at height 2 that works for all primes has yet to be determined.  
As a continuation of our current work on explicit formulas for power operations, 
we would like to extend our computation to other universal families of elliptic curves and other primes.  
Specifically, these include: 
\begin{itemize}
 \item The universal elliptic curve with a choice of a point of exact order 5 and a nowhere-vanishing invariant one-form (see \cite[Section 1.1]{behrensormsby}) at the prime 2 

 \item The Legendre family (see \cite[Section 1.3.2]{behrens}) at the prime 3 

 \item The universal elliptic curve with a choice of a point of exact order 3 and a nowhere-vanishing invariant one-form 
 (see \cite[Proposition 3.2]{tmf3} and \cite[Section 3]{h2p2}) at the prime 5 

 \item The universal elliptic curve with a choice of a point of exact order 4 and a nowhere-vanishing invariant one-form (our choice of universal family at the prime 3) at the prime 5 
\end{itemize}

Based on explicit formulas, we hope to work toward a more intrinsic and structural understanding of power operations.  
Here are some specific plans.  

From the calculations in Section \ref{subsec:h2p3}, for power operations in Morava $E$-theory of height 2 at the prime 3, 
we get an analog of the relations satisfied by the Steenrod operations.  
In particular, Adem relations have interesting geometry underlying their derivation, 
but we still have no real understanding of how they arise, and what role they play in the structure of the \DL algebra.  
From the perspective of the chromatic filtration (see Section \ref{sec:intro}), 
it would be helpful to understand the connection to classical derivations for the Steenrod operations 
(cf.~\cite[Section VIII.1]{steenrod} and \cite{bullettmacdonald}).  

We hope to acquire a better understanding of the underlying geometry, 
specifically the moduli problems corresponding to the power operations (see \cite[Theorem B]{cong}).  
For example, at the prime 3, the parameter $\A$ in \eqref{A} and Proposition \ref{prop} 
gives a coordinate of the modular curve which represents (relative to our universal elliptic curve) 
the moduli problem [3-$Isog] = [\G_0(3)]$ (see \cite[Sections 6.5 and 6.8]{KM}, and cf.~the moduli problem $\big([\G_1(4)],[\G_0(3)]\big)$ in \cite[10.9.6]{KM}).  
This parameter satisfies a quartic polynomial equation $w(\A) = 0$ (cf.~\eqref{w}).  
Thus the number of the individual power operations $Q_k$ corresponds to the degree of the representing modular curve.  
We plan to study other properties of power operations in connection with the geometric properties of modular curves.  
For example, it would be interesting to analyze the coefficients in $w(\A)$ (also cf.~\cite[Sections 10.13 and 12.9]{KM}).  

From the formulas for the relations satisfied by the individual power operations $Q_k$, 
we hope to extract patterns invariant under change of basis of the \DL algebra 
(the coefficients in these formulas vary if we choose a different parameter for the total power operation $\p$).  
For example, we would like to study if there is an analog of the presentation for Adem relations in \cite{bullettmacdonald}, 
and to study congruences modulo 3 of our formulas as motivated by \cite[Section 4.8]{mc1}.  

We plan to study questions about functoriality of power operations for Morava $E$-theories: 
How does the \DL algebra change under maps between formal group laws of $E$-theories?  
For this we hope to seek functorial models for the power operation structure.  
Here are two possible starting points.  
\begin{itemize}
 \item Thanks to \cite[Section 9.1]{pearson}, we are able to compute the formulas in \cite[25.1.2 and 25.1.11]{FG} 
 for the Witt-vector-like group functors $W^F$ \cite[Theorem 25.1.12 and Remark 25.1.14]{FG} 
 attached to the formal group laws $F$ of Morava $E$-theories of height 2 at the primes 2 and 3.  
 Based on these formulas, we would like to explore if there is a connection 
 between power operations for Morava $E$-theories of height 2 
 and a certain generalization of Witt vectors (and Dieudonn\'e modules over rings of Witt vectors), 
 along the lines of power operations at height 1 (see \cite[Section 4]{hopkins}).  
 In particular, Rezk's congruence criterion for the algebraic theory of power operations in Morava $E$-theory \cite[Theorem A]{cong} 
 can be viewed as a ``higher chromatic''\footnote{With respect to the chromatic filtration by heights of formal group laws (see Section \ref{sec:intro}).  } analog 
 of Wilkerson's congruence criterion for torsion-free $\lambda$-rings \cite[Proposition 1.2]{wilkerson} (cf.~\cite[Section 1]{cong}).  
 In view of the analogy between Wilkerson's criterion 
 and a criterion for Witt vectors \cite[VII.4.6]{lazard} (attributed by Cartier to Dwork and Dieudonn\'e), 
 we may generalize the latter for formal group laws of Morava $E$-theories of height 2.  
 The congruences needed in the formulation may be helpful in our calculations exploring power operations and Witt vectors.  

 \item We plan to study the notion of plethory \cite{BW} in the context of power operations (cf.~\cite[Sections 4-6]{cong}, \cite{staceywhitehouse}, and \cite{bauer}), 
 and to see if it is possible to transport some of the functorial constructions for plethories to the \DL algebra.  
 For example, there are morphisms, base change, and amplifications of plethories (see \cite[1.8, 1.13, and Theorem 7.1]{BW}).  
 We are particularly interested in Borger and Wieland's ``formula-free'' construction 
 of the ring $W(R)$ of $p$-typical Witt vectors over a commutative ring $R$: 
 it is the $\Lambda_p$-ring co-induced from $R$, 
 where $\Lambda_p$ is the amplification along the trivial $\BF_p$-plethory 
 of a certain free $\BZ$-plethory generated by the $p$'th-power Frobenius map (see \cite[Section 12]{BW}).  
 We would like to explore a generalization of this construction.  
\end{itemize}

We hope to study power operations with a view toward applications, 
especially given the fruitful work on the Steenrod operations and the Adams operations, 
as well as motivated by \cite[Lemma 14 and Remark 1]{hopkins}, \cite{davis}, and \cite{behrenslec}.  


%\nocite{*}
%\bibliographystyle{amsalpha}
%\bibliography{rs}
%\end{document}
\renewcommand\refname{}
\newcommand{\MRn}[2]{\href{http://www.ams.org/mathscinet-getitem?mr=#1}{MR#1 #2}}
\begin{thebibliography}

\section*{\leftskip=-.44in References \vspace{.13in}}

\bibitem[Ada60]{adamshopf}
J.~F. Adams, \emph{On the non-existence of elements of {H}opf invariant one},
  Ann. of Math. (2) \textbf{72} (1960), 20--104. \MRn{0141119}{(25 \#4530)}

\bibitem[Ada62]{adamsvector}
\bysame, \emph{Vector fields on spheres}, Ann. of Math. (2) \textbf{75} (1962),
  603--632. \MRn{0139178}{(25 \#2614)}

\bibitem[Ade52]{adem}
Jos{\'e} Adem, \emph{The iteration of the {S}teenrod squares in algebraic
  topology}, Proc. Nat. Acad. Sci. U. S. A. \textbf{38} (1952), 720--726.
  \MRn{0050278}{(14,306e)}

\bibitem[AHS04]{AHS04}
Matthew Ando, Michael~J. Hopkins, and Neil~P. Strickland, \emph{The sigma
  orientation is an {$H\sb \infty$} map}, Amer. J. Math. \textbf{126} (2004),
  no.~2, 247--334. \MRn{2045503}{(2005d:55009)}

\bibitem[And95]{andoduke}
Matthew Ando, \emph{Isogenies of formal group laws and power operations in the
  cohomology theories {$E\sb n$}}, Duke Math. J. \textbf{79} (1995), no.~2,
  423--485. \MRn{1344767}{(97a:55006)}

\bibitem[And00]{andotrans}
\bysame, \emph{Power operations in elliptic cohomology and representations of
  loop groups}, Trans. Amer. Math. Soc. \textbf{352} (2000), no.~12,
  5619--5666. \MRn{1637129}{(2001b:55016)}

\bibitem[Ati66]{atiyah}
M.~F. Atiyah, \emph{Power operations in {$K$}-theory}, Quart. J. Math. Oxford
  Ser. (2) \textbf{17} (1966), 165--193. \MRn{0202130}{(34 \#2004)}

\bibitem[Bau]{bauer}
Tilman Bauer, \emph{Formal plethories}, preprint, available at
  \href{http://arxiv.org/abs/1107.5745}{arXiv:1107.5745}.

\bibitem[Beh]{behrenslec}
Mark Behrens, \emph{Exotic spheres and topological modular forms}, talk on
  joint work with Mike Hill, Mike Hopkins, and Mark Mahowald, available at
  \href{http://www.ima.umn.edu/videos/index.php?id=1789}{http://www.ima.umn.edu\linebreak/videos/index.php?id=1789}.

\bibitem[Beh06]{behrens}
\bysame, \emph{A modular description of the {$K(2)$}-local sphere at the prime
  3}, Topology \textbf{45} (2006), no.~2, 343--402. \MRn{2193339}{(2006i:55016)}

\bibitem[BM82]{bullettmacdonald}
S.~R. Bullett and I.~G. Macdonald, \emph{On the {A}dem relations}, Topology
  \textbf{21} (1982), no.~3, 329--332. \MRn{649764}{(83h:55035)}

\bibitem[BMMS86]{H_infty}
R.~R. Bruner, J.~P. May, J.~E. McClure, and M.~Steinberger, \emph{{$H\sb \infty
  $} ring spectra and their applications}, Lecture Notes in Mathematics, vol.
  1176, Springer-Verlag, Berlin, 1986. \MRn{836132}{(88e:55001)}

\bibitem[BO]{behrensormsby}
Mark Behrens and Kyle Ormsby, \emph{On the homotopy of ${Q}(3)$ and ${Q}(5)$ at
  the prime 2}, preprint, available at
  \href{http://arxiv.org/abs/1211.0076}{arXiv:1211.0076}.

\bibitem[BW05]{BW}
James Borger and Ben Wieland, \emph{Plethystic algebra}, Adv. Math.
  \textbf{194} (2005), no.~2, 246--283. \MRn{2139914}{(2006i:13044)}

\bibitem[Dav11]{davis}
Donald~M. Davis, \emph{Some new nonimmersion results for real projective
  spaces}, Bol. Soc. Mat. Mexicana (3) \textbf{17} (2011), no.~2, 159--166.
  \MRn{2986078}{}

\bibitem[Gan]{ganterTate}
Nora Ganter, \emph{Stringy power operations in {T}ate ${K}$-theory}, preprint,
  available at \href{http://arxiv.org/abs/math/0701565}{arXiv:0701565}.

\bibitem[Gre88]{blind}
J.~P.~C. Greenlees, \emph{How blind is your favourite cohomology theory?},
  Exposition. Math. \textbf{6} (1988), no.~3, 193--208. \MRn{949783}{(89j:55001)}

\bibitem[Haz78]{FG}
Michiel Hazewinkel, \emph{Formal groups and applications}, Pure and Applied
  Mathematics, vol.~78, Academic Press Inc. [Harcourt Brace Jovanovich
  Publishers], New York, 1978. \MRn{506881}{(82a:14020)}

\bibitem[HM]{hopkinsmahowald}
Mike Hopkins and Mark Mahowald, \emph{From elliptic curves to homotopy theory},
  preprint, available at
  \href{http://hopf.math.purdue.edu//Hopkins-Mahowald/eo2homotopy.pdf}{http://hopf.math.purdue.edu//Hopkins-Mahowald/eo2homotopy.pdf}.

\bibitem[Hop]{hopkins}
M.~J. Hopkins, \emph{{$K(1)$}-local {$E\sb \infty $} ring spectra}, unpublished
  notes, available at
  \href{http://www.math.rochester.edu/people/faculty/doug/otherpapers/knlocal.pdf}{http://\linebreak www.math.rochester.edu/people/faculty/doug/otherpapers/knlocal.pdf}.

\bibitem[KM85]{KM}
Nicholas~M. Katz and Barry Mazur, \emph{Arithmetic moduli of elliptic curves},
  Annals of Mathematics Studies, vol. 108, Princeton University Press,
  Princeton, NJ, 1985. \MRn{772569}{(86i:11024)}

\bibitem[Law09]{tafoverview}
Tyler Lawson, \emph{An overview of abelian varieties in homotopy theory}, New
  topological contexts for {G}alois theory and algebraic geometry ({BIRS}
  2008), Geom. Topol. Monogr., vol.~16, Geom. Topol. Publ., Coventry, 2009,
  pp.~179--214. \MRn{2544390}{(2010i:55006)}

\bibitem[Laz75]{lazard}
Michel Lazard, \emph{Commutative formal groups}, Lecture Notes in Mathematics,
  Vol. 443, Springer-Verlag, Berlin, 1975. \MRn{0393050}{(52 \#13861)}

\bibitem[LN12]{level3}
Tyler Lawson and Niko Naumann, \emph{Commutativity conditions for truncated
  {B}rown-{P}eterson spectra of height 2}, J. Topol. \textbf{5} (2012), no.~1,
  137--168. \MRn{2897051}{}

\bibitem[Lur09]{survey}
J.~Lurie, \emph{A survey of elliptic cohomology}, Algebraic topology, Abel
  Symp., vol.~4, Springer, Berlin, 2009, pp.~219--277. \MRn{2597740}{(2011f:55009)}

\bibitem[McC83]{mcclure}
James~E. McClure, \emph{Dyer-{L}ashof operations in {$K$}-theory}, Bull. Amer.
  Math. Soc. (N.S.) \textbf{8} (1983), no.~1, 67--72. \MRn{682824}{(84e:55014)}

\bibitem[Mil58]{milnor}
John Milnor, \emph{The {S}teenrod algebra and its dual}, Ann. of Math. (2)
  \textbf{67} (1958), 150--171. \MRn{0099653}{(20 \#6092)}

\bibitem[Mor89]{morava}
Jack Morava, \emph{Forms of {$K$}-theory}, Math. Z. \textbf{201} (1989), no.~3,
  401--428. \MRn{999737}{(90e:55011)}

\bibitem[MR09]{tmf3}
Mark Mahowald and Charles Rezk, \emph{Topological modular forms of level 3},
  Pure Appl. Math. Q. \textbf{5} (2009), no.~2, Special Issue: In honor of
  Friedrich Hirzebruch. Part 1, 853--872. \MRn{2508904}{(2010g:55010)}

\bibitem[Pea]{pearson}
Paul Pearson, \emph{Calculating formal group laws}, unpublished notes,
  available at
  \href{http://www.math.rochester.edu/people/faculty/pearson/papers/fgls-2up.pdf}{http://\linebreak www.math.rochester.edu/people/faculty/pearson/papers/fgls-2up.pdf}.

\bibitem[Qui69]{quillenfgl}
Daniel Quillen, \emph{On the formal group laws of unoriented and complex
  cobordism theory}, Bull. Amer. Math. Soc. \textbf{75} (1969), 1293--1298.
  \MRn{0253350}{(40 \#6565)}

\bibitem[Qui71]{quillenmu}
\bysame, \emph{Elementary proofs of some results of cobordism theory using
  {S}teenrod operations}, Advances in Math. \textbf{7} (1971), 29--56 (1971).
  \MRn{0290382}{(44 \#7566)}

\bibitem[Rav92]{orange}
Douglas~C. Ravenel, \emph{Nilpotence and periodicity in stable homotopy
  theory}, Annals of Mathematics Studies, vol. 128, Princeton University Press,
  Princeton, NJ, 1992, Appendix C by Jeff Smith. \MRn{1192553}{(94b:55015)}

\bibitem[Rez]{h2p2}
Charles Rezk, \emph{Power operations for {M}orava ${E}$-theory of height 2 at
  the prime 2}, preprint, available at
  \href{http://arxiv.org/abs/0812.1320}{arXiv:0812.1320}.

\bibitem[Rez09]{cong}
\bysame, \emph{The congruence criterion for power operations in {M}orava
  {$E$}-theory}, Homology, Homotopy Appl. \textbf{11} (2009), no.~2, 327--379.
  \MRn{2591924}{(2011e:55021)}

\bibitem[Rez12]{mc1}
\bysame, \emph{Modular isogeny complexes}, Algebr. Geom. Topol. \textbf{12}
  (2012), no.~3, 1373--1403. \MRn{2966690}{}

\bibitem[Ser53]{serre}
Jean-Pierre Serre, \emph{Cohomologie modulo {$2$} des complexes
  d'{E}ilenberg-{M}ac{L}ane}, Comment. Math. Helv. \textbf{27} (1953),
  198--232. \MRn{0060234}{(15,643c)}

\bibitem[Ste62]{steenrod}
N.~E. Steenrod, \emph{Cohomology operations}, Lectures by N. E. Steenrod
  written and revised by D. B. A. Epstein. Annals of Mathematics Studies, No.
  50, Princeton University Press, Princeton, N.J., 1962. \MRn{0145525}{(26
  \#3056)}

\bibitem[Str98]{Str98}
N.~P. Strickland, \emph{Morava {$E$}-theory of symmetric groups}, Topology
  \textbf{37} (1998), no.~4, 757--779. \MRn{1607736}{(99e:55008)}

\bibitem[SW51]{steenrodwhitehead}
N.~E. Steenrod and J.~H.~C. Whitehead, \emph{Vector fields on the
  {$n$}-sphere}, Proc. Nat. Acad. Sci. U. S. A. \textbf{37} (1951), 58--63.
  \MRn{0041436}{(12,847c)}

\bibitem[SW09]{staceywhitehouse}
Andrew Stacey and Sarah Whitehouse, \emph{The hunting of the {H}opf ring},
  Homology, Homotopy Appl. \textbf{11} (2009), no.~2, 75--132. \MRn{2559638}{(2011d:55034)}

\bibitem[tD68]{tomdieck}
Tammo tom Dieck, \emph{Steenrod-{O}perationen in {K}obordismen-{T}heorien},
  Math. Z. \textbf{107} (1968), 380--401. \MRn{0244989}{(39 \#6302)}

\bibitem[Voe03]{V}
Vladimir Voevodsky, \emph{Reduced power operations in motivic cohomology},
  Publ. Math. Inst. Hautes \'Etudes Sci. (2003), no.~98, 1--57. \MRn{2031198}{(2005b:14038a)}

\bibitem[Wil82]{wilkerson}
Clarence Wilkerson, \emph{Lambda-rings, binomial domains, and vector bundles
  over {${\bf C}P(\infty )$}}, Comm. Algebra \textbf{10} (1982), no.~3,
  311--328. \MRn{651605}{(83f:55003)}

\bibitem[Zhu]{p3}
Yifei Zhu, \emph{The power operation structure on {M}orava ${E}$-theory of
  height 2 at the prime 3}, accepted by Algebraic and Geometric Topology,
  preprint available at \href{http://arxiv.org/abs/1210.3730}{arXiv:1210.3730}.

\end{thebibliography}
\end{document}