%%%%%%%%%%%%%%%%%%          gtlatex.tem       %%%%%%%%%%%%%%%%%%
%
%  Template for articles written in LaTeX for publication in
%  G&T, G&TM and A&GT.  This template must be used with latex2e.  
%  If you use BiBTeX then you can collect the bibliography style 
%  file  gtart.bst  from the same directory as this file.  Full
%  instructions for using gtpart.cls are given in gtpartdoc.pdf.  
%
%
\documentclass[microtype]{gtpart}     % Basic GT/GTM/AGT style
%
%   Uncomment one of the next three lines to obtain a full "mock-up"
%   of a published article:
%   A&GT:  \agtart     G&T:  \gtart   G&TM:  \gtmonart
%
%   NOTE:  Please do not place your article in a public place (eg
%          on the arXiv) in "mock-up" form unless it has been accepted
%          for publication in the relevant journal.
%
%\gtart  
\agtart
%\gtmonart
%
%   Add necessary packages here.  Note that amsthm, amssymb and
%   amsmath are already loaded, so there is no need to add any 
%   of these.  Examples:
%
%\usepackage{pinlabel}  %%% the recommended graphics+labelling package
%\usepackage{graphicx}  %%% the recommended graphics package
%\usepackage[all]{xy}
%\usepackage{amscd}
\usepackage{stmaryrd}
\usepackage[all]{xy}
\usepackage{tikz}
\usepackage{url}
\usepackage{enumerate}
\usepackage{tensor}
\usepackage{mathrsfs}
\usepackage{graphicx}
\usepackage{mathtools}
\usepackage{/rhome/3/zyf/texmf/natbib}
\bibpunct{[}{]}{;}{a}{,}{,}
\usepackage{hyperref}
%
%
%%% Start of metadata
%

\title{The power operation structure on Morava $E$--theory\\of height 2 at the prime 3}

%  First author
%
\author{Yifei Zhu}
\givenname{Yifei}
\surname{Zhu}
\address{Department of Mathematics\\Northwestern University\\\newline 
2033 Sheridan Road\\Evanston\\IL 60208\\USA}
\email{zyf@math.northwestern.edu}
\urladdr{http://www.math.northwestern.edu/~zyf}

%  Second author (uncomment if necessary)
%
%\author{}
%\givenname{}
%\surname{}
%\address{}
%\email{}
%\urladdr{}
%
%  (Add a similar block for other authors)
%
%   Title and author both have running head options:
%
%   \title[Running head title]{Main title}
%   \author[Running head author]{Author}
%
% give a separate \keyword and \subject line for each keyword/phrase or 
% subject class eg \keyword{framed link} \subject{primary}{msc2000}{57M25}

\keyword{power operations}
\keyword{elliptic curves}
\keyword{Morava E-theory}
\keyword{K(1)-localization}
\subject{primary}{msc2000}{55S12}
\subject{secondary}{msc2000}{55N20}
\subject{secondary}{msc2000}{55N34}

%
%  fill in the reference and password if your article is stored at the
%  arXiv eg \arxivreference{math.GT/0512347}  \arxivpassword{5spud}

\arxivreference{1210.3730}
\arxivpassword{645di}

%
%  Leave the following items blank
%
\volumenumber{}
\issuenumber{}
\publicationyear{}
\papernumber{}
\startpage{}
\endpage{}
\doi{}
\MR{}
\Zbl{}
\received{}
\revised{}
\accepted{}
\published{}
\publishedonline{}
\proposed{}
\seconded{}
\corresponding{}
\editor{}
\version{}

%%% End of metadata
%
%%% Start of user-defined macros %%%
%
%   Theorem-type environments.  There are two predefined styles :
%
%   \theoremstyle{plain} : for theorems, corollaries etc with heading 
%   bold and left justified, optional note bracketed in roman type
%   and statement in slanted type.  This is the default style.
%
%   \theoremstyle{definition} : (alias remark)  for definitions, remarks 
%   etc with heading bold and left justified, optional note as before but
%   with statement in roman type.
%   
%   Some sample  \newtheorem's  (delete these unless you need
%   them and insert your own):
%
\newtheorem{thm}{Theorem}[section]    % Standard theorem environment
\newtheorem{lem}[thm]{Lemma}          % Lemma environment with numbering 
%                                     % consecutive to theorems
\newtheorem*{zlem}{Zorn's Lemma}      % A special unnumbered lemma.
%
%\theoremstyle{definition}
%\newtheorem{defn}[thm]{Definition}    % Definition environment with 
%                                     % numbering consecutive to theorems
%\newtheorem*{rem}{Remark}             % Unnumbered environment for remarks.
%
%   Type your macros (\newcommand's etc) below.
%
\newtheorem{cor}{Corollary}[section]
\makeatletter
\let\c@cor=\c@thm
\makeatother
\makeautorefname{cor}{Corollary}
\newtheorem{prop}{Proposition}[section]
\makeatletter
\let\c@prop=\c@thm
\makeatother
\makeautorefname{prop}{Proposition}
\theoremstyle{remark}
\newtheorem{rmk}{Remark}[section]
\makeatletter
\let\c@rmk=\c@thm
\makeatother
\makeautorefname{rmk}{Remark}
\newtheorem{ex}{Example}[section]
\makeatletter
\let\c@ex=\c@thm
\makeatother
\makeautorefname{ex}{Example}
\theoremstyle{definition}
\newtheorem{defn}{Definition}[section]
\makeatletter
\let\c@defn=\c@thm
\makeatother
\makeautorefname{defn}{Definition}

\def\co{\colon\thinspace}
\newcommand{\mb}[1]{\mathbb{#1}}
\newcommand{\Spec}{{\rm Spec\thinspace}}
\newcommand{\Proj}{{\rm Proj\thinspace}}
\newcommand{\Spf}{{\rm Spf\thinspace}}
\newcommand{\Mod}{{\rm Mod}}
\newcommand{\Alg}{{\rm Alg}}
\newcommand{\DL}{Dyer--Lashof~}
\newcommand{\BC}{{\mb C}}
\newcommand{\BF}{{\mb F}}
\newcommand{\BG}{{\mb G}}
\newcommand{\BP}{{\mb P}}
\newcommand{\BZ}{{\mb Z}}
\newcommand{\HC}{\widehat{C~}\!}
\newcommand{\HS}{\widehat{S~}\!}
\newcommand{\Tf}{\widetilde{f}}
\newcommand{\Tp}{\widetilde{\psi}}
\newcommand{\md}{~~{\rm mod}~}
\newcommand{\ad}{{\rm and}}
\newcommand{\ce}{\coloneqq}
\newcommand{\A}{\alpha}
\newcommand{\G}{\Gamma}
\newcommand{\g}{\gamma}
\newcommand{\K}{\kappa}
\newcommand{\p}{\psi^3}
\newcommand{\s}{S^\bullet}
\newcommand{\isog}[1]{\fullref{prop:isog}\thinspace \eqref{isog(#1)}}
\newcommand{\q}[1]{\fullref{prop:Q}\thinspace \eqref{Q(#1)}}
\newcommand{\go}[1]{\fullref{def:go}\thinspace \eqref{go(#1)}}
\renewcommand{\bibname}{References}
\let\oldbibsection\bibsection
\renewcommand{\bibsection}{\oldbibsection\addcontentsline{toc}{part}{References}}



%%% End of user-defined macros %%%

\begin{document}

\begin{abstract}    % type your abstract below
 We give explicit calculations of the algebraic theory of power operations 
 for a specific Morava $E$--theory spectrum and its $K(1)$--localization.  
 These power operations arise from the universal degree-3 isogeny of 
 elliptic curves associated to the $E$--theory.  
\end{abstract}

\maketitle

%%%%%%%%%%%%%%%%%%%%   Start of main body of article

\section{Introduction}

Suppose $E$ is a commutative $S$--algebra, in the sense of Elmendorf, Kriz, 
Mandell and May \cite{EKMM}, and $A$ is a commutative $E$--algebra.  We want 
to capture the properties and underlying structure of the homotopy groups 
$\pi_* A = A_*$ of $A$, by studying operations associated to the cohomology 
theory that $E$ represents.  

An important family of cohomology operations, called {\em power operations}, 
is constructed via the extended powers.  Specifically, consider the 
{\em $m$'th extended power} functor 
\[
 \BP_E^m (-) \ce (-)^{\wedge_E m} / \Sigma_m \co \Mod_E \to \Mod_E 
\]
on the category of $E$--modules, which sends an $E$--module to its $m$-fold 
smash product over $E$ modulo the action by the symmetric group on $m$ 
letters.  The $\BP_E^m (-)$'s assemble together to give the {\em free 
commutative $E$--algebra} functor 
\[
 \BP_E (-) \ce \bigvee_{m \geq 0} \BP_E^m (-) \co \Mod_E \to \Alg_E 
\]
from the category of $E$--modules to the category of commutative 
$E$--algebras.  These functors descend to homotopy categories.  In 
particular, for any integers $d$ and $i$, each 
$\A \in \pi_{d+i} \BP_E^m (\Sigma^d E)$ gives rise to a power operation 
\[
 Q_\A \co A_d \to A_{d+i} 
\]
(cf.~Bruner, May, McClure and Steinberger 
\cite[Sections I.2 and IX.1]{H_infty} and Rezk \cite[Section 3]{cong}).  

Under the action of power operations, $A_*$ is an algebra over some operad in 
$E_*$--modules involving the structure of $E_* B\Sigma_m$ for all $m$.  This 
operad is traditionally called a {\em \DL algebra}, or more precisely, a \DL 
{\em theory} as the {\em algebraic theory} of power operations acting on the 
homotopy groups of commutative $E$--algebras 
(cf.~\cite[Chapters III, VIII and IX]{H_infty} and Rezk 
\cite[Section 9]{lpo}).  

A specific case is when $E$ represents a Morava $E$--theory of height $n$ and 
$A$ is $K(n)$--local.  Morava $E$--theory spectra play a crucial role in 
modern stable homotopy theory, particularly in the work of Ando, Hopkins and 
Strickland on the topological approach to elliptic genera (see \cite{cube}).  
As recalled in Rezk \cite[1.5]{cong}, the $K(n)$--local $E$--\DL theory is 
largely understood based on work of those authors.  In \cite{cong}, Rezk maps 
out the foundations of this theory.  He gives a congruence criterion for an 
algebra over the \DL theory \cite[Theorem A]{cong}.  This enables one to 
study the \DL {\em theory}, which models all the algebraic structure 
naturally adhering to $A_*$, by working with a certain associative ring $\G$ 
as the \DL {\em algebra}.  Moreover, Rezk provides a geometric description of 
this congruence criterion, in terms of sheaves on the moduli problem of 
deformations of formal groups and Frobenius isogenies (see 
\cite[Theorem B]{cong}).  This connects the structure of $\G$ to the geometry 
underlying $E$, moving one step forward from a workable object $\G$ to things 
that are computable.  In a companion paper \cite{h2p2}, Rezk gives explicit 
calculations of the \DL theory for a specific Morava $E$--theory of height 
$n = 2$ at the prime 2.  

The purpose of this paper is to make available calculations analogous to some 
of the results in \cite{h2p2}, at the prime 3, together with calculations of 
the corresponding power operations on the $K(1)$--localization of the Morava 
$E$--theory spectrum.  


\subsection{Outline of the paper}

As in Rezk \cite{h2p2}, the computation of power operations in this paper 
follows the approach of Steenrod \cite{steenrod}: one first defines a total 
power operation, and then uses the computation of the cohomology of the 
classifying space $B\Sigma_m$ for the symmetric group to obtain individual 
power operations.  These two steps are carried out respectively in 
\fullref{sec:total} and \fullref{sec:individual}.  

In \fullref{sec:total}, by doing calculations with elliptic curves associated 
to our Morava $E$--theory $E$, we give formulas for the total power operation 
$\p$ on $E_0$ and the ring $S_3$ which represents the corresponding moduli 
problem.  

In \fullref{sec:individual}, based on calculations of $E^* B\Sigma_m$ in 
Strickland \cite{Str98} as reflected in the formula for $S_3$, we define 
individual power operations, and derive the relations they satisfy.  In view 
of the general structure studied in Rezk \cite{cong}, we then get an explicit 
description of the \DL algebra $\G$ for $K(2)$--local commutative 
$E$--algebras.  

In \fullref{sec:K(1)}, we describe the relationship between the total power 
operation $\p$, at height 2, and the corresponding $K(1)$--local power 
operations.  We then derive formulas for the latter from the calculations in 
\fullref{sec:total}.  

\begin{rmk}
\label{rmk:grading}
 In \fullref{sec:total}, we do calculations with a universal elliptic curve 
 over {\em all} of the moduli stack which is an affine open subscheme of a 
 weighted projective space (cf.~\fullref{prop:C}).  At the prime 3, the 
 supersingular locus consists of a single closed point, and the corresponding 
 Morava $E$--theory arises {\em locally} in an affine coordinate chart of 
 this weighted projective space containing the supersingular locus.  In this 
 paper we choose a particular affine coordinate chart for computing the 
 homotopy groups of the $E$--theory spectrum and the power operations; we 
 hope that the generality of the calculations in \fullref{sec:total} makes it 
 easier to work with other coordinate charts as well.  

 Some of the formulas involved in our calculations with this universal 
 elliptic curve are in fact valid only fiber by fiber over the base scheme 
 (for example, the polynomial $\psi_3$ in the proof of \fullref{prop:tors}, 
 and the group law algorithm in the proof of \fullref{prop:isog}).  As the 
 base scheme is connected, the statements for the universal elliptic curve 
 follow by rigidity (see Katz and Mazur \cite[Section 2.4]{KM}).  We write 
 those formulas formally to streamline the proofs.  
\end{rmk}

\begin{rmk}
 The ring $S_3$ turns out to be an algebra with one generator over the base 
 ring where our elliptic curve is defined (cf.~\isog{i} and \eqref{S_3}).  
 This generator appears as a parameter in the formulas for the total power 
 operation $\p$, and is responsible for how the individual power operations 
 are defined and how their formulas look.  Different choices of this 
 parameter result in different bases of the \DL algebra $\G$.  The parameter 
 in this paper comes from the relative cotangent space of the elliptic curve 
 at the identity (see \isog{iv}, \fullref{cor:K'} and \fullref{rmk:K'}).  
 This choice is convenient for deriving Adem relations in \q{iv}, and it fits 
 into the treatment of gradings in Rezk \cite[Section 2]{cong} (see \go{ii} 
 and \fullref{thm:gamma}).  We should point out that our choice is by no 
 means canonical.  We do not know yet, as part of the structure of the \DL 
 algebra, if there is a canonical basis which is both geometrically 
 interesting and computationally convenient.  Somewhat surprisingly, although 
 it appears to come from different considerations, our choice has an analog 
 at the prime 2 which coincides with the parameter used in Rezk \cite{h2p2} 
 (see \fullref{rmk:K} and \fullref{rmk:KK'}).  The calculations follow a 
 recipe in hope of generalizing to other Morava $E$--theories of height 2; we 
 hope to address these matters and recognize more of the general patterns 
 based on further computational evidence.  
\end{rmk}


\subsection{Acknowledgements}

I thank Charles Rezk for his encouragement on this work, and for his 
observation in a correspondence which led to \fullref{prop:frob^2}, 
\fullref{cor:K'} and eventually an approach to Adem relations as in the proof 
of \q{iv}.  

I thank Kyle Ormsby for helpful discussions on \fullref{sec:total}, and for 
directing me to places in the literature.  

I thank Tyler Lawson for the sustained support from him I received as a 
student.  


\subsection{Conventions}

Let $p$ be a prime, $q$ a power of $p$, and $n$ a positive integer.  We use 
the symbols 
\[
 \BF_q\text{,}~~\BZ_q~~\ad~~\BZ/n 
\]
to denote a field with $q$ elements, the ring of $p$--typical Witt vectors 
over $\BF_q$, and the additive group of integers modulo $n$, respectively.  

If $R$ is a ring, then $R\llbracket x \rrbracket$ and $R (\!(x)\!)$ denote 
the rings of formal power series and formal Laurent series over $R$ in the 
variable $x$ respectively.  If $I \subset R$ is an ideal, then $R_I^\wedge$ 
denotes the completion of $R$ with respect to $I$.  

If $E$ is an elliptic curve and $m$ is an integer, then $[m]$ denotes the 
multiplication-by-$m$ map on $E$, and $E[m]$ denotes the $m$--torsion 
subgroup scheme of $E$.  

All formal groups mentioned in this paper are commutative and 
one-dimensional.  

The terminology for the structure of a \DL theory follows Rezk 
\cite{cong, h2p2}; some of the notions there are taken in turn from Borger 
and Wieland \cite{BW} and Voevodsky \cite{V}.  


\section{Total power operations}
\label{sec:total}

\subsection{A universal elliptic curve and a Morava $E$--theory spectrum}
\label{subsec:ec}

A Morava $E$--theory of height 2 at the prime 3 has its formal group as the 
universal deformation of a height-2 formal group over a perfect field of 
characteristic 3.  Given a supersingular elliptic curve over such a field, 
its formal completion at the identity produces a formal group of height 2.  
To study power operations for the corresponding $E$--theory, we do 
calculations with a universal deformation of that supersingular elliptic 
curve which is a family of elliptic curves with a $\G_1(N)$--structure (see 
Katz and Mazur \cite[Section 3.2]{KM}) where $N$ is prime to 3.  Here is a 
specific model for such a universal family (cf.~Husem\"oller 
\cite[4(4.6a)]{husemoller}).  

\begin{prop}
\label{prop:C}
 Over $\BZ [1/4]$, the moduli problem of smooth elliptic curves with a choice 
 of a point of exact order 4 and a nowhere-vanishing invariant 1--form is 
 represented by 
 \begin{equation}
 \label{Cxy}
  C \co y^2 + a x y + a b y = x^3 + b x^2 
 \end{equation}
 with chosen point $(0,0)$ and 1--form 
 $-dx / (2 y + a x + a b) = dy / (a y - 3 x^2 - 2 b x)$ over the graded ring 
 \[
  \s \ce \BZ [1/4] [a, b, \Delta^{-1}] 
 \]
 where $|a| = 1$, $|b| = 2$ and $\Delta = a^2 b^4 (a^2 - 16 b)$.  
\end{prop}
\begin{proof}
 Let $P$ be the chosen point of exact order 4.  Since $2P$ is 2--torsion, the 
 tangent line of the elliptic curve at $P$ passes through $2P$, and the 
 tangent line at $2P$ passes through the identity at the infinity.  With this 
 observation, the rest of the proof is analogous to that of Mahowald and Rezk 
 \cite[Proposition 3.2]{tmf3}.  
\end{proof}

Over a finite field of characteristic 3, $C$ is supersingular precisely when 
the quantity 
\begin{equation}
\label{H}
 H \ce a^2 + b 
\end{equation}
vanishes (cf.~Silverman \cite[V.4.1a]{AEC}).  As $(3,H)$ is a homogeneous 
maximal ideal of $\s$ corresponding to the closed subscheme $\Spec \BF_3$, 
the supersingular locus consists of a single closed point, and $C$ restricts 
to $\BF_3$ as 
\[
 C_0 \co y^2 + x y - y = x^3 - x^2.  
\]

From the above universal deformation $C$ of $C_0$, we next produce a Morava 
$E$--theory spectrum which is 2--periodic.  We follow the convention that 
elements in algebraic degree $n$ lie in topological degree $2n$, and work in 
an affine \'etale coordinate chart of the weighted projective space 
$\Proj \BZ [1/4] [a, b]$ (see \fullref{rmk:grading}).  Define elements $u$ 
and $c$ such that 
\begin{equation}
\label{uc}
 a = u c \qquad \ad \qquad b = u^2.  
\end{equation}
Consider the graded ring 
\[
 \s [u^{-1}] \cong \BZ [1/4] [a, \Delta^{-1}] [u^{\pm1}] 
\]
where $|u| = 1$, and denote by $S$ its subring of elements in degree 0 so 
that 
\begin{equation}
\label{S}
 S \cong \BZ [1/4] [c, \delta^{-1}] 
\end{equation}
where $\delta = u^{-12} \Delta = c^2 (c^2 - 16)$.  Write 
\[
 \HS \ce \BZ_9 \llbracket h \rrbracket 
\]
where 
\begin{equation}
\label{h}
 h \ce u^{-2} H = c^2 + 1.  
\end{equation}
Let $i$ be an element generating $\BZ_9$ over $\BZ_3$ with $i^2 = -1$.  We 
may choose 
\[
 c \equiv i \md (3,h) 
\]
and we have 
\[
 \delta \equiv -1 \md (3,h) 
\]
where $(3,h)$ is the maximal ideal of the complete local ring $\HS$.  Then by 
Hensel's lemma, both $c$ and $\delta$ lie in $\HS$, and both are invertible.  
Thus 
\[
 \HS \cong S_{(3,h)}^\wedge.  
\]

Now $C$ restricts to $S$ as 
\begin{equation}
\label{Cc}
 y^2 + c x y + c y = x^3 + x^2.  
\end{equation}
Let $\HC$ be the formal completion of $C$ over $S$ at the identity.  It is a 
formal group over $\HS$, and its reduction to $\HS / (3,h) \cong \BF_9$ is a 
formal group $\BG$ of height 2 in view of \eqref{h} and \eqref{H}.  By the 
Serre--Tate theorem (see Katz and Mazur \cite[Theorem 2.9.1]{KM}), 
3--adically the deformation theory of an elliptic curve is equivalent to the 
deformation theory of its 3--divisible group, and thus $\HC$ is the universal 
deformation of $\BG$ in view of \fullref{prop:C}.  Let $E$ be the 
$E_\infty$--ring spectrum which represents the Morava $E$--theory associated 
to $\BG$ (see Goerss and Hopkins \cite[Corollary 7.6]{GH}).  Then 
\begin{equation}
\label{E_*}
 E_* \cong \BZ_9 \llbracket h \rrbracket [u^{\pm 1}] 
\end{equation}
where $u$ is in topological degree 2.  


\subsection{Points of exact order 3}

To study $C$ in a formal neighborhood of the identity, it is convenient to 
make a change of variables.  Let 
\[
 u = \frac{x}{y} \quad \ad \quad v = \frac{1}{y}, \qquad {\rm so} \qquad x = \frac{u}{v} \quad \ad \quad y = \frac{1}{v}.  
\]
The identity of $C$ is then $(u,v) = (0,0)$, with $u$ a local uniformizer.  
This coordinate $u$ corresponds to the element $u$ in \eqref{E_*} via a 
chosen isomorphism $\HC \cong \Spf E^0(\BC\BP^\infty)$ of formal groups over 
$\HS \cong E^0$ (see Ando, Hopkins and Strickland 
\cite[Definition 1.2]{cube}).  It is different from the element $u$ in 
\eqref{uc}; here $|u| = -1$.  We will use this abuse of notation and remind 
the reader when confusion may arise.  In $uv$--coordinates, the equation 
\eqref{Cxy} of $C$ becomes 
\begin{equation}
\label{Cuv}
 v + a u v + a b v^2 = u^3 + b u^2 v.  
\end{equation}

\begin{prop}
\label{prop:tors}
 On the elliptic curve $C$ over $\s$, the $uv$--coordinates $(d,e)$ of any 
 point of exact order 3 satisfy the identities 
 \begin{equation}
 \label{f}
  f(d) = 0 
 \end{equation}
 and 
 \begin{equation}
 \label{g}
  e = g(d) 
 \end{equation}
 where $f, g \in \s [u]$ are given by 
 \begin{equation*}
 \begin{split}
  f(u) = & ~ b^4 u^8 + 3 a b^3 u^7 + 3 a^2 b^2 u^6 + (a^3 b + 7 a b^2) u^5 + (6 a^2 b - 6 b^2) u^4 + 9 a b u^3 \\
         & + (-a^2 + 8 b) u^2 - 3 a u - 3, \\
  g(u) = & -\frac{1}{a (a^2 - 16 b)} \big( a b^3 u^7 + (3 a^2 b^2 - 2 b^3) u^6 + (3 a^3 b - 6 a b^2) u^5 + (a^4 + a^2 b \\
         & + 2 b^2) u^4 + (4 a^3 - 15 a b) u^3 + 18 b u^2 - 12 a u - 18 \big).  
 \end{split}
 \end{equation*}
\end{prop}
\begin{proof}
 \footnote{See \fullref{apx:tors} for explicit formulas for the polynomials 
 $\Tf$, $Q_1$, $R_1$, $Q_2$, $R_2$, $K$, $L$, $M$ and $N$ that appear in the 
 proof.  }
 Given the elliptic curve $C$ with equation \eqref{Cxy}, a point $Q$ is of 
 exact order 3 if and only if the polynomial 
 \[
  \psi_3 (x) \ce 3 x^4 + (a^2 + 4 b) x^3 + 3 a^2 b x^2 + 3 a^2 b^2 x + a^2 b^3 
 \]
 vanishes at $Q$ (cf.~Silverman \cite[Exercise 3.7f]{AEC}).  Substituting 
 $x = u/v$ and clearing the denominators, we get a polynomial 
 \[
  \Tp_3(u,v) \ce 3 u^4 + (a^2 + 4 b) u^3 v + 3 a^2 b u^2 v^2 + 3 a^2 b^2 u v^3 + a^2 b^3 v^4.  
 \]
 As $Q = (d,e)$ in $uv$--coordinates, we then have 
 \begin{equation}
 \label{Tp}
  \Tp_3(d,e) = 0.  
 \end{equation}

 To get the polynomial $f$, we take $v$ as variable and rewrite \eqref{Cuv} 
 as a quadratic equation 
 \begin{equation}
 \label{quadratic}
  a b v^2 + (-b u^2 + a u + 1) v - u^3 = 0, 
 \end{equation}
 where the leading coefficient $a b$ is invertible in 
 $\s = \BZ [1/4] [a, b, \Delta^{-1}]$ as $\Delta = a^2 b^4 (a^2 - 16 b)$.  
 Define 
 \begin{equation}
 \label{Tfdef}
  \Tf(u) \ce \Tp_3(u,v) \Tp_3(u,\bar{v}) 
 \end{equation}
 where $v$ and $\bar{v}$ are formally the conjugate roots of 
 \eqref{quadratic} so that we compute $\Tf$ in terms of $u$ by substituting 
 \[
  v + \bar{v} = \frac{b u^2 - a u - 1}{a b} \qquad \ad \qquad v \bar{v} = -\frac{u^3}{a b}.  
 \]
 We then factor $\Tf$ over $\s$ as 
 \begin{equation}
 \label{Tffactor}
  \Tf(u) = -\frac{u^4 f(u)}{a^2 b} 
 \end{equation}
 with $f$ the stated polynomial of order 8.  We check that $f$ is irreducible 
 by applying Eisenstein's criterion to the homogeneous prime ideal $(3,H)$ of 
 $\s$.  

 We have $\Tf(d) = 0$ by \eqref{Tfdef} and \eqref{Tp}.  To see $f(d) = 0$, 
 consider the closed subscheme $D \subset C[3]$ of points of exact order 3.  
 By Katz and Mazur \cite[Theorem 2.3.1]{KM} it is finite locally free of rank 
 8 over $\s$.  By the Cayley--Hamilton theorem, as a global section of $D$, 
 $u$ locally satisfies a homogeneous monic equation of order 8, and this 
 equation locally defines the rank-8 scheme $D$.  Since $D$ is affine, it is 
 then globally defined by such an equation.  In view of $\Tf(d) = 0$ and 
 \eqref{Tffactor}, we determine this equation, and (up to a unit in $\s$) get 
 the first stated identity \eqref{f}.  

 To get the polynomial $g$, we note that both the quartic polynomial 
 \[
  A(v) \ce \Tp_3(d,v) 
 \]
 and the quadratic polynomial 
 \[
  B(v) \ce a b v^2 + (-b d^2 + a d + 1) v - d^3 
 \]
 defined from \eqref{quadratic} vanish at $e$, and thus so does their 
 greatest common divisor (gcd).  Applying the Euclidean algorithm (see 
 \fullref{apx:tors} for explicit expressions), we have 
 \begin{equation*}
 \begin{split}
  A(v) = & ~ Q_1(v) B(v) + R_1(v), \\
  B(v) = & ~ Q_2(v) R_1(v) + R_2, 
 \end{split}
 \end{equation*}
 where 
 \[
  R_1(v) = K(d) v + L(d) 
 \]
 for some polynomials $K$ and $L$, and $R_2 = 0$ in view of \eqref{f}.  Thus 
 $R_1(v)$ is the gcd of $A(v)$ and $B(v)$, and hence 
 \[
  K(d) e + L(d) = R_1(e) = 0.  
 \]
 To write $e$ in terms of $d$ from the above identity, we apply the Euclidean 
 algorithm to $f$ and $K$.  Their gcd turns out to be 1, and thus there are 
 polynomials $M$ and $N$ with 
 \[
  M(u) f(u) + N(u) K(u) = 1.  
 \]
 By \eqref{f} we then have $N(d) K(d) = 1$, and thus 
 \[
  e = -N(d) L(d) = g(d) 
 \]
 where $g$ is as stated.  
\end{proof}


\subsection{A universal isogeny and a total power operation}

\begin{prop}
\label{prop:isog}
 \mbox{}
 \begin{enumerate}[(i)]
  \item \label{isog(i)} The universal degree-3 isogeny $\psi$ with source $C$ 
  is defined over the graded ring 
  \[
   \s_3 \ce \s [\K] \big/ \big( W(\K) \big) 
  \]
  where $|\K| = -2$ and 
  \begin{equation}
  \label{W}
   W(\K) = \K^4 - \frac{6}{b^2} ~ \K^2 + \frac{a^2 - 8 b}{b^4} ~ \K - \frac{3}{b^4}, 
  \end{equation}
  and has target the elliptic curve 
  \[
   C' \co v + a' u v + a' b' v^2 = u^3 + b' u^2 v 
  \]
  where 
  \begin{equation*}
  \begin{split}
   a' = & ~ \frac{1}{a} \big( (a^2 b^4 - 4 b^5) \K^3 + 4 b^4 \K^2 + (-6 a^2 b^2 + 20 b^3) \K + a^4 - 12 a^2 b + 12 b^2 \big), \\
   b' = & ~ b^3.  
  \end{split}
  \end{equation*}

  \item \label{isog(ii)} The kernel of $\psi$ is generated by a point $Q$ of 
  exact order 3 with coordinates $(d,e)$ satisfying 
  \begin{equation}
  \label{K}
  \begin{split}
   \K = & -\frac{1}{a^2 - 16 b} \big( a b^3 d^7 + (3 a^2 b^2 - 2 b^3) d^6 + (3 a^3 b - 6 a b^2) d^5 + (a^4 \\
        & + a^2 b + 2 b^2) d^4 + (4 a^3 - 15 a b) d^3 + (a^2 + 2 b) d^2 - 12 a d - 18 \big) \\
      = & ~ a e - d^2.  
  \end{split}
  \end{equation}

  \item \label{isog(iii)} The restriction of $\psi$ to the supersingular 
  locus at the prime 3 is the 3--power Frobenius endomorphism.  

  \item \label{isog(iv)} The induced map $\psi^*$ on the relative cotangent 
  space of $C'$ at the identity sends $du$ to $\K du$.  
 \end{enumerate}
\end{prop}
\begin{proof}
 \footnote{See \fullref{apx:isog} for the power series expansion of $v$ and 
 details of the calculations involving the group law on $C$ that appear in 
 the proof.  }
 Let $P = (u,v)$ be a point on $C$, and $Q = (d,e)$ be a point of exact order 
 3.  Rewriting \eqref{Cuv} as 
 \[
  v = u^3 + b u^2 v - a u v - a b v^2, 
 \]
 we express $v$ as a power series in $u$ by substituting this equation into 
 itself recursively.  For the purpose of our calculations, we take this power 
 series up to $u^{12}$ as an expression for $v$, and write $e = g(d)$ as in 
 \eqref{g}.  

 Define functions $u'$ and $v'$ by 
 \begin{equation}
 \label{u'v'}
 \begin{split}
  u' \ce & ~ u(P) \cdot u(P-Q) \cdot u(P+Q), \\
  v' \ce & ~ v(P) \cdot v(P-Q) \cdot v(P+Q), 
 \end{split}
 \end{equation}
 where $u(-)$ and $v(-)$ denote the $u$--coordinate and $v$--coordinate of a 
 point respectively.  By computing the group law on $C$, we express $u'$ and 
 $v'$ as power series in $u$: 
 \begin{equation}
 \label{KL}
 \begin{split}
  u' = & ~ \K u + (\text{higher-order terms}), \\
  v' = & ~ \lambda u^3 + (\text{higher-order terms}), 
 \end{split}
 \end{equation}
 where the coefficients ($\K$, $\lambda$, etc) involve $a$, $b$ and $d$.  In 
 particular, in view of \eqref{f}, we compute that $\K$ satisfies $W(\K) = 0$ 
 with $|\K| = -2$ as stated in \eqref{isog(i)}.  

 Now define the isogeny $\psi \co C \to C'$ by 
 \begin{equation}
 \label{psi}
  u\big( \psi(P) \big) \ce u' \qquad \ad \qquad v\big( \psi(P) \big) \ce \frac{\K^3}{\lambda} \cdot v', 
 \end{equation}
 where we introduce the factor $\K^3 / \lambda$ so that the equation of $C'$ 
 will be in the Weierstrass form.  Using \eqref{KL} (see \fullref{apx:isog} 
 for explicit expressions), we then determine the coefficients in a 
 Weierstrass equation and get the stated equation of $C'$.  

 We next check the statement of \eqref{isog(ii)}.  In view of \eqref{psi} and 
 \eqref{u'v'}, the kernel of $\psi$ is the order-3 subgroup generated by $Q$.  
 In \eqref{K}, the first identity is computed in \eqref{KL}; we then compare 
 it with \eqref{g} and get the second identity.  

 For \eqref{isog(iii)}, recall from \fullref{subsec:ec} that the 
 supersingular locus at the prime 3 is $\Spec \BF_3$.  Over $\BF_3$, since 
 the group $C[3](\BF_3) = 0$ by Silverman \cite[V.3.1a]{AEC}, $Q$ coincides 
 with the identity, and thus 
 \[
  u\big( \psi(P) \big) = u(P) \cdot u(P-Q) \cdot u(P+Q) = \big( u(P) \big)^3.  
 \]
 As the $u$--coordinate is a local uniformizer at the identity, $\psi$ then 
 restricts to $\BF_3$ as the 3--power Frobenius endomorphism.  

 The statement of \eqref{isog(iv)} follows by definition of $\K$ in 
 \eqref{KL}.  
\end{proof}

\begin{rmk}
 In view of \isog{iii}, the formal completion of $\psi \co C \to C'$ at the 
 identity of $C$ is a {\em deformation of Frobenius} in the sense of Rezk 
 \cite[11.3]{cong}.  When it is clear from the context, we will simply call 
 $\psi$ itself a deformation of Frobenius.  
\end{rmk}

\begin{rmk}
\label{rmk:K}
 From \eqref{u'v'} and \eqref{KL} we have 
 \begin{equation}
 \label{norm}
  u(P-Q) \cdot u(P+Q) = \K + u \cdot (\text{higher-order terms}).  
 \end{equation}
 In particular $u(-Q) \cdot u(Q) = \K$ (cf.~Katz and Mazur 
 \cite[Proposition 7.5.2 and Section 7.7]{KM}).  The analog of $\K$ at the 
 prime 2 coincides with the parameter $d$ studied in Rezk 
 \cite[Section 3]{h2p2}.  
\end{rmk}

Recall from \fullref{subsec:ec} that 
\[
 E^0 \cong \BZ_9 \llbracket h \rrbracket = \HS \cong S_{(3,h)}^\wedge 
\]
in which $c$ and $i$ are elements with $c^2 + 1 = h$ and $i^2 = -1$.  Given 
the graded ring $\s_3$ in \isog{i}, define 
\begin{equation}
\label{S_3}
 S_3 \ce S [\A] \big/ \big( w(\A) \big) 
\end{equation}
where 
\[
 w(\A) = \A^4 - 6 \A^2 + (c^2 - 8) \A - 3 
\]
(cf.~the definition of $S$ from $\s$ in \eqref{S}; in particular 
$\K = u^{-2} \A$ where $u$ is defined in \eqref{uc}).  By Strickland's 
theorem \cite[Theorem 1.1]{Str98} and the Serre--Tate theorem 
\cite[Theorem 2.9.1]{KM} we have 
\[
 E^0 B\Sigma_3 / I \cong \big( S_3 \big)_{(3,h)}^\wedge 
\]
where 
\begin{equation}
\label{transfer}
 I \ce \bigoplus_{0<i<3} {\rm image} \big( E^0 B(\Sigma_i \times \Sigma_{3-i}) \xrightarrow{\rm transfer} E^0 B\Sigma_3 \big) 
\end{equation}
is the {\em transfer ideal}.  In view of this and the construction of 
{\em total power operations} for Morava $E$--theories in Rezk 
\cite[3.23]{cong}, we have the following corollary.  
\begin{cor}
\label{cor:psi3}
 The total power operation 
 \[
  \p \co E^0 \to E^0 B\Sigma_3 / I \cong E^0 [\A] \big/ \big( w(\A) \big) 
 \]
 is given by 
 \begin{equation*}
 \begin{split}
  \p(h) = & ~ h^3 - 27 h^2 + 201 h -342 + (-6 h^2 + 108 h - 334) \A + (3 h - 27) \A^2 \\
          & + (h^2 - 18 h + 57) \A^3, \\
  \p(c) = & ~ c^3 -12 c + 12 c^{-1} + (-6 c + 20 c^{-1}) \A + 4 c^{-1} \A^2 + (c - 4 c^{-1}) \A^3, \\
  \p(i) \thinspace = & -i.  
 \end{split}
 \end{equation*}
\end{cor}
\begin{proof}
 By \isog{i}, in $xy$--coordinates, $C'$ restricts to $S_3$ as 
 \[
  y^2 + c' x y + c' y = x^3 + x^2 
 \]
 where 
 \[
  c' = \frac{1}{c} \big( (c^2 - 4) \A^3 + 4 \A^2 + (-6 c^2 + 20) \A + c^4 - 12 c^2 + 12 \big).  
 \]
 By Rezk \cite[Theorem B]{cong}, since the above equation is in the form of 
 \eqref{Cc}, there is a correspondence between the restriction to $S_3$ of 
 the universal isogeny $\psi$, which is a deformation of Frobenius, and the 
 total power operation $\p$.  In particular $\p(c)$ is given by $c'$.  As 
 $\p$ is a ring homomorphism, we then get the formula for 
 $\p(h) = \p(c^2 + 1)$.  We also have 
 \[
  \big( \p(i) \big)^2 = \p(-1) = -1, 
 \]
 and thus $\p(i) = i$ or $-i$.  By Rezk 
 \cite[Propositions 3.25 and 10.5]{cong} the value of 
 $\p(i) \in E^0 [\A] \big/ \big( w(\A) \big)$, viewed as a cubic polynomial 
 in $\A$, has constant term congruent to $i^3$ modulo 3.  Hence $\p(i) = -i$.  
\end{proof}


\section{Individual power operations}
\label{sec:individual}

\subsection{A composite of deformations of Frobenius}

Recall from \fullref{prop:isog} that over $\s_3$ we have the universal 
degree-3 isogeny $\psi \co C \to C' = C/G$ where $G$ is an order-3 subgroup 
of $C$; in particular, $\psi$ is a deformation of the 3--power Frobenius 
endomorphism over the supersingular locus.  We want to construct a similar 
isogeny $\psi'$ with source $C'$ so that the composite $\psi' \circ \psi$ 
will correspond to a composite of total power operations via Rezk 
\cite[Theorem B]{cong} (cf.~Katz and Mazur \cite[11.3.1]{KM}).  

Let $G' \ce C[3]/G$ which is an order-3 subgroup of $C'$.  Recall from 
\fullref{subsec:ec} that $C$ is a universal deformation of a supersingular 
elliptic curve $C_0$.  Since the 3--divisible group of $C_0$ is formal, 
$C_0[3]$ is connected.  Thus over a formal neighborhood of the supersingular 
locus, if $G$ is the unique connected order-3 subgroup of $C$, $G'$ is then 
the unique connected order-3 subgroup of $C'$.  As in the proof of 
\fullref{prop:isog}, we define $\psi' \co C' \to C'/G'$ using a point of 
exact order 3 in $G'$ (see \eqref{u'v'} and \eqref{psi}), and $\psi'$ is then 
a deformation of Frobenius.  Over the supersingular locus, the pair 
$(\psi, \psi')$ is {\em cyclic in standard order} in the sense of Katz and 
Mazur \cite[6.7.7]{KM}.  We describe it more precisely as below.  

\begin{prop}
\label{prop:frob^2}
 The following diagram of elliptic curves over $\s_3$ commutes: 
 \begin{equation}
 \label{frob^2}
  \begin{tikzpicture}[baseline=(current bounding box.center)]
          \node (LT) at (0, 2) {$C$}; 
          \node (MT) at (3.8, 2) {$C/G = $}; 
          \node (RT) at (4.65, 2.04) {$C'$}; 
          \node (LB) at (1.9, 0) {$C/C[3]$}; 
          \node (MB) at (3.5, 0) {$\cong \frac{C/G}{C[3]/G} = $}; 
          \node (RB) at (4.65, 0.025) {$\frac{C'}{G'}$}; 
          \draw [->] (LT) -- node [above] {$\scriptstyle \psi$} (MT); 
          \draw [->] (LT) -- node [left] {$\scriptstyle [-3]$} (LB); 
          \draw [->] (RT) -- node [right] {$\scriptstyle \psi'$} (RB); 
  \end{tikzpicture}
 \end{equation}
\end{prop}
\begin{proof}
 By Katz and Mazur \cite[Theorem 2.4.2]{KM}, since $\Proj \s_3$ is connected, 
 we need only show that the locus over which $\psi' \circ \psi = [-3]$ is not 
 empty, where by abuse of notation $[-3]$ denotes the map $[-3]$ on $C$ 
 composed with the canonical isomorphism from $C/C[3]$ to $C'/G'$.  

 Recall from \fullref{subsec:ec} that $C$ restricts to the supersingular 
 locus $\BF_3$ as 
 \[
  C_0 \co y^2 + x y - y = x^3 - x^2.  
 \]
 By \isog{iii} both $\psi$ and $\psi'$ restrict as the 3--power Frobenius 
 endomorphism $\psi_0$ on $C_0$.  By \cite[Theorem 2.6.3]{KM}, in the 
 endomorphism ring of $C_0$, $\psi_0$ is a root of the polynomial 
 \begin{equation}
 \label{charpoly}
  X^2 - {\rm trace}(\psi_0) \cdot X + 3 
 \end{equation}
 with ${\rm trace}(\psi_0)$ an integer satisfying 
 \[
  \big( {\rm trace}(\psi_0) \big)^2 \leq 12.  
 \]
 Moreover by Silverman \cite[Exercise 5.10a]{AEC}, since $C_0$ is 
 supersingular, we have 
 \[
  {\rm trace}(\psi_0) \equiv 0 \md 3.  
 \]
 Thus ${\rm trace}(\psi_0) = 0$, 3 or $-3$.  We exclude the latter two 
 possibilities by checking the action of $\psi_0$ at the 2--torsion point 
 $(1,0)$.  It then follows from \eqref{charpoly} that $\psi_0 \circ \psi_0$ 
 agrees with $[-3]$ on $C_0$ over $\BF_3$.  
\end{proof}

Analogous to \isog{iv}, let $\K'$ be the element in $\s_3$ such that 
$(\psi')^*$ sends $du$ to $\K' du$.  Note that $|\K'| = -6$.  

\begin{cor}
\label{cor:K'}
 The following relations hold in $\s_3$: 
 \[
  b^4 \K \K' + 3 = 0 
 \]
 and 
 \[
  \K' = -\K^3 + \frac{6}{b^2} ~ \K - \frac{a^2 - 8 b}{b^4}.  
 \]
\end{cor}
\begin{proof}
 The isogenies in \eqref{frob^2} induce maps on relative cotangent spaces at 
 the identity.  For the first stated relation, by \isog{iv} we need only show 
 that $[3]^*$ sends $du$ to $3 du / b^4$, where by abuse of notation $[3]$ 
 denotes the map $[3]$ on $C$ composed with the canonical isomorphism from 
 $C/C[3]$ to $C'/G'$.  

 For $i = 1$, 2, 3 and 4, let $Q_i$ be a generator for each of the four 
 order-3 subgroups of $C$.  Each $Q_i$ can be chosen as $Q$ in \eqref{u'v'}, 
 and we denote the corresponding quantity $\K$ in \eqref{KL} by $\K_i$.  Let 
 $P = (u,v)$ be a point on $C$.  Define an isogeny $\Psi$ with source $C$ by 
 \begin{equation*}
 \begin{split}
  u\big( \Psi(P) \big) \ce & ~ u(P) \prod_{i=1}^4 \big( u(P-Q_i) \cdot u(P+Q_i) \big), \\
  v\big( \Psi(P) \big) \ce & ~ v(P) \prod_{i=1}^4 \big( v(P-Q_i) \cdot v(P+Q_i) \big).  
 \end{split}
 \end{equation*}
 In view of \eqref{norm}, since $[3]$ has the same kernel as $\Psi$, we have 
 \begin{equation}
 \label{s}
  [3]^* (du) = s \cdot \K_1 \K_2 \K_3 \K_4 \cdot du 
 \end{equation}
 where $s$ is a degree-0 unit in $\s$ coming from an automorphism of $C$ over 
 $\s$.  In view of \eqref{W} we have 
 \[
  \K_1 \K_2 \K_3 \K_4 = -\frac{3}{b^4}.  
 \]
 We compute that $s = -1$ by comparing the restrictions of the two sides of 
 \eqref{s} to the ordinary point corresponding to the homogeneous maximal 
 ideal $(5,H)$ of $\s$, and then comparing the restrictions to the point 
 corresponding to $(7,H)$: over both points, $[3]^*$ becomes the 
 multiplication-by-3 map, and $-3 / b^4$ becomes $-3$ as $b = 1$ in 
 \eqref{Cc}.  Thus $[3]^*$ sends $du$ to $3 du / b^4$.  

 The second stated relation follows by a computation from the first relation 
 and the relation $W(\K) = 0$ in \isog{i}.  
\end{proof}

\begin{rmk}
\label{rmk:KK'}
 As noted in \fullref{rmk:K}, the (local) analog of $\K$ at the prime 2 
 coincides with the parameter $d$ in Rezk \cite[Section 3]{h2p2}.  In 
 particular, with the notations in \cite[Section 3]{h2p2} and Mahowald and 
 Rezk \cite[Proposition 3.2]{tmf3}, $d$ and $d'$ satisfy an analogous 
 relation $A_3 d d' + 2 = 0$ which locally reduces to $d d' + 2 = 0$ (the 
 analog of the factor $s$ in the proof of \fullref{cor:K'} equals 1; 
 cf.~Ando \cite[Theorem 2.6.4]{andoduke}).  These arise as examples of Baker, 
 Gonz{\'a}lez-Jim{\'e}nez, Gonz{\'a}lez and Poonen \cite[Lemma 3.21]{poonen}.  
\end{rmk}

\begin{rmk}
\label{rmk:K'}
 In view of \eqref{frob^2}, $-\psi'$ (composed with the canonical isomorphism 
 on the target) turns out to be the dual isogeny of $\psi$ (cf.~the proof of 
 Katz and Mazur \cite[Theorem 2.9.4]{KM}).  By \fullref{cor:K'} and \eqref{H} 
 we have 
 \[
  -\K' = \K^3 - \frac{6}{b^2} ~ \K + \frac{a^2 - 8 b}{b^4} \equiv \frac{H}{b^4} \md (3,\K).  
 \]
 This congruence agrees with the interpretation of $H$ as defined by the 
 tangent map of the Verschiebung isogeny over $\BF_3$ (see 
 \cite[12.4.1]{KM}).  
\end{rmk}


\subsection{Individual power operations}

Let $A$ be a $K(2)$--local commutative $E$--algebra.  By Rezk 
\cite[3.23]{cong} and \fullref{cor:psi3}, we have a total power operation 
\[
 \p \co A_0 \to A_0 \otimes_{E_0} (E^0 B\Sigma_3 / I) \cong A_0 [\A] \big/ \big( w(\A) \big).  
\]
We also have a composite of total power operations 
\begin{equation}
\label{psi3^2}
\begin{split}
 A_0 \stackrel{\p}{\longrightarrow} A_0 \otimes_{E_0} (E^0 B\Sigma_3 / I) \stackrel{\p}{\longrightarrow} 
 & ~ \big( A_0 \otimes_{E_0} (E^0 B\Sigma_3 / I) \big) \tensor[^\p]{\otimes}{_{E_0 [\A]}} (E^0 B\Sigma_3 / I) \\
 \cong \thinspace \thinspace & ~ \Big( A_0 [\A] \big/ \big( w(\A) \big) \Big) \tensor[^\p]{\otimes}{_{E_0 [\A]}} \Big( E^0 [\A] \big/ \big( w(\A) \big) \Big) 
\end{split}
\end{equation}
where the elements in the target $M \tensor[^\p]{\otimes}{_R} N$ are subject 
to the equivalence relation 
\[
 m \otimes (r \cdot n) \sim \big( m \cdot \p(r) \big) \otimes n 
\]
for $m \in M$, $n \in N$ and $r \in R$, with 
\[
 \p(\A) = -\A^3 + 6 \A - h + 9 
\]
in view of \fullref{cor:K'}, as well as other relations in a usual tensor 
product.  

\begin{defn}
\label{def:Q}
 Define {\em individual power operations} 
 \[
  Q_k \co A_0 \to A_0 
 \]
 for $k = 0$, 1, 2 and 3 by 
 \[
  \p (x) = Q_0(x) + Q_1(x) \A + Q_2(x) \A^2 + Q_3(x) \A^3.  
 \]
\end{defn}

\begin{prop}
\label{prop:Q}
 The following relations hold among the individual power operations $Q_0$, 
 $Q_1$, $Q_2$ and $Q_3$: 
 \begin{enumerate}[(i)]
  \item \label{Q(i)} $Q_0(1) = 1, \quad Q_1(1) = Q_2(1) = Q_3(1) = 0;$ 

  \item \label{Q(ii)} $Q_k(x+y) = Q_k(x) + Q_k(y) \text{~for all~} k;$ 

  \item \label{Q(iii)} {\em Commutation relations }
  \begin{equation*}
  \begin{split}
   Q_0(h x) = & ~ (h^3 - 27 h^2 + 201 h - 342) Q_0(x) + (3 h^2 - 54 h + 171) Q_1(x) \\
              & + (9 h - 81) Q_2(x) + 24 Q_3(x), \\
   Q_1(h x) = & ~ (-6 h^2 + 108 h - 334) Q_0(x) + (-18 h + 171) Q_1(x) + (-72) Q_2(x) \\
              & + (h - 9) Q_3(x), \\
   Q_2(h x) = & ~ (3 h - 27) Q_0(x) + 8 Q_1(x) + 9 Q_2(x) + (-24) Q_3(x), \\
   Q_3(h x) = & ~ (h^2 - 18 h + 57) Q_0(x) + (3 h - 27) Q_1(x) + 8 Q_2(x) + 9 Q_3(x), \\
   Q_0(c x) = & ~ (c^3 - 12 c + 12 c^{-1}) Q_0(x) + (3 c - 12 c^{-1}) Q_1(x) + (12 c^{-1}) Q_2(x) \\
              & + (-12 c^{-1}) Q_3(x), \\
   Q_1(c x) = & ~ (-6 c + 20 c^{-1}) Q_0(x) + (-20 c^{-1}) Q_1(x) + (- c + 20 c^{-1}) Q_2(x) \\
              & + (4 c - 20 c^{-1}) Q_3(x), \\
   Q_2(c x) = & ~ (4 c^{-1}) Q_0(x) + (-4 c^{-1}) Q_1(x) + (4 c^{-1}) Q_2(x) + (- c - 4 c^{-1}) Q_3(x), \qquad \\
   Q_3(c x) = & ~ (c - 4 c^{-1}) Q_0(x) + (4 c^{-1}) Q_1(x) + (-4 c^{-1}) Q_2(x) + (4 c^{-1}) Q_3(x), \\
   Q_k(i x) = & ~ (-i) Q_k(x) \text{~for all~} k; 
  \end{split}
  \end{equation*}

  \item \label{Q(iv)} {\em Adem relations }
  \begin{equation*}
  \begin{split}
   Q_1Q_0(x) = & ~ (-6) Q_0Q_1(x) + 3 Q_2Q_1(x) + (6 h - 54) Q_0Q_2(x) + 18 Q_1Q_2(x) \\
               & + (-9) Q_3Q_2(x) + (-6 h^2 + 108 h - 369) Q_0Q_3(x) \\
               & + (-18 h + 162) Q_1Q_3(x) + (-54) Q_2Q_3(x), \\
   Q_2Q_0(x) = & ~ 3 Q_3Q_1(x) + (-3) Q_0Q_2(x) + (3 h - 27) Q_0Q_3(x) + 9 Q_1Q_3(x), \qquad \qquad \\
   Q_3Q_0(x) = & ~ Q_0Q_1(x) + (-h + 9) Q_0Q_2(x) + (-3) Q_1Q_2(x) \\
               & + (h^2 - 18 h + 63) Q_0Q_3(x) + (3 h - 27) Q_1Q_3(x) + 9 Q_2Q_3(x); 
  \end{split}
  \end{equation*}

  \item \label{Q(v)} {\em Cartan formulas }
  \begin{equation*}
  \begin{split}
   Q_0(xy) = & ~ Q_0(x) Q_0(y) + 3 \big( Q_3(x) Q_1(y) + Q_2(x) Q_2(y) + Q_1(x) Q_3(y) \big) \\
             & + 18 Q_3(x) Q_3(y), \\
   Q_1(xy) = & ~ \big( Q_1(x) Q_0(y) + Q_0(x) Q_1(y) \big) \\
             & + (-h + 9) \big( Q_3(x) Q_1(y) + Q_2(x) Q_2(y) + Q_1(x) Q_3(y) \big) \\
             & + 3 \big( Q_3(x) Q_2(y) + Q_2(x) Q_3(y) \big) + (-6 h + 54) Q_3(x) Q_3(y), \\
   Q_2(xy) = & ~ \big( Q_2(x) Q_0(y) + Q_1(x) Q_1(y) + Q_0(x) Q_2(y) \big) \\
             & + 6 \big( Q_3(x) Q_1(y) + Q_2(x) Q_2(y) + Q_1(x) Q_3(y) \big) \\
             & + (-h + 9) \big( Q_3(x) Q_2(y) + Q_2(x) Q_3(y) \big) + 39 Q_3(x) Q_3(y), \\
   Q_3(xy) = & ~ \big( Q_3(x) Q_0(y) + Q_2(x) Q_1(y) + Q_1(x) Q_2(y) + Q_0(x) Q_3(y) \big) \qquad \qquad \qquad \\
             & + 6 \big( Q_3(x) Q_2(y) + Q_2(x) Q_3(y) \big) + (-h + 9) Q_3(x) Q_3(y); 
  \end{split}
  \end{equation*}

  \item \label{Q(vi)} {\em The Frobenius congruence }
  \begin{equation*}
   Q_0(x) \equiv x^3 \md 3.  \qquad \qquad \qquad \qquad \qquad \qquad \qquad \qquad \qquad \qquad \qquad \qquad
  \end{equation*}
 \end{enumerate}
\end{prop}
\begin{proof}
 The relations in \eqref{Q(i)}, \eqref{Q(ii)}, \eqref{Q(iii)} and 
 \eqref{Q(v)} follow computationally from the formulas in \fullref{cor:psi3} 
 together with the fact that $\p$ is a ring homomorphism.  

 For \eqref{Q(iv)}, there is a canonical isomorphism $C/C[3] \cong C$ of 
 elliptic curves over $\s \subset \s_3$.  Given the correspondence between 
 deformations of Frobenius and power operations in Rezk 
 \cite[Theorem B]{cong}, the commutativity of \eqref{frob^2} then implies 
 that the composite \eqref{psi3^2} lands in $A_0$.  In terms of formulas, we 
 have 
 \begin{equation*}
 \begin{split}
  \p \big( \p(x) \big) = & ~ \p \big( Q_0(x) + Q_1(x) \A + Q_2(x) \A^2 + Q_3(x) \A^3 \big) \\
                       = & ~ \sum_{k = 0}^3 \p \big( Q_k(x) \big) \big( \p(\A) \big)^k \\
                       = & ~ \sum_{k = 0}^3 \sum_{j = 0}^3 Q_jQ_k(x) \A^j (-\A^3 + 6 \A - h + 9)^k \\
                  \equiv & ~ \Psi_0(x) + \Psi_1(x) \A + \Psi_2(x) \A^2 + \Psi_3(x) \A^3 \md \big( w(\A) \big) 
 \end{split}
 \end{equation*}
 where each $\Psi_i$ is an $E_0$--linear combination of the $Q_jQ_k$'s.  The 
 vanishing of $\Psi_1(x)$, $\Psi_2(x)$ and $\Psi_3(x)$ gives the three 
 relations in \eqref{Q(iv)}.  

 For \eqref{Q(vi)}, we note that $Q_0$ is a representative of the 
 {\em Frobenius class} in the sense of Rezk \cite[10.3]{cong}.  Since $A$ is 
 a $K(2)$--local commutative $E$--algebra, the congruence then follows from 
 \cite[Theorem A]{cong}.  
\end{proof}

\begin{ex}
\label{ex}
 We have $E^0 S^2 \cong \BZ_9 \llbracket h \rrbracket [u] / (u^2)$.  Via the 
 isomorphism $\Spf E^0(\BC\BP^\infty) \cong \HC$ and in view of the 
 definition of $\K$ in \eqref{KL}, the $Q_k$'s act canonically on 
 $u \in E^0 S^2$: 
 \[
  Q_k(u) = \left\{
  \begin{array}{ll}
    u,  & \quad {\rm if}~k = 1, \\
    0,  & \quad {\rm if}~k \neq 1.  \\
  \end{array}
  \right.
 \]
 We then get the values of the $Q_k$'s on elements in $E^0 S^2$ from 
 \q{i}--\eqref{Q(iii)}.  
\end{ex}


\subsection{The \DL algebra}

\begin{defn}
\label{def:go}
 \mbox{}
 \begin{enumerate}[(i)]
  \item \label{go(i)} Let $i$ be an element generating $\BZ_9$ over $\BZ_3$ 
  with $i^2 = -1$.  Define $\g$ to be the associative ring generated over 
  $\BZ_9 \llbracket h \rrbracket$ by elements $q_0$, $q_1$, $q_2$ and $q_3$ 
  subject to the following relations: the $q_k$'s commute with elements in 
  $\BZ_3 \subset \BZ_9 \llbracket h \rrbracket$, and satisfy {\em commutation 
  relations} 
  \begin{equation*}
  \begin{split}
   q_0 h = & ~ (h^3 - 27 h^2 + 201 h - 342) q_0 + (3 h^2 - 54 h + 171) q_1 + (9 h - 81) q_2 \\
           & + 24 q_3, \\
   q_1 h = & ~ (-6 h^2 + 108 h - 334) q_0 + (-18 h + 171) q_1 + (-72) q_2 + (h - 9) q_3, \\
   q_2 h = & ~ (3 h - 27) q_0 + 8 q_1 + 9 q_2 + (-24) q_3, \\
   q_3 h = & ~ (h^2 - 18 h + 57) q_0 + (3 h - 27) q_1 + 8 q_2 + 9 q_3, \\
   q_k i ~ = & ~ (-i) q_k \text{~for all~} k, 
  \end{split}
  \end{equation*}
  and {\em Adem relations} 
  \begin{equation*}
  \begin{split}
   q_1q_0 = & ~ (-6) q_0q_1 + 3 q_2q_1 + (6 h - 54) q_0q_2 + 18 q_1q_2 + (-9) q_3q_2 \\
            & + (-6 h^2 + 108 h - 369) q_0q_3 + (-18 h + 162) q_1q_3 + (-54) q_2q_3, \quad~~ \\
   q_2q_0 = & ~ 3 q_3q_1 + (-3) q_0q_2 + (3 h - 27) q_0q_3 + 9 q_1q_3, \\
   q_3q_0 = & ~ q_0q_1 + (-h + 9) q_0q_2 + (-3) q_1q_2 + (h^2 - 18 h + 63) q_0q_3 \\
            & + (3 h - 27) q_1q_3 + 9 q_2q_3.  
  \end{split}
  \end{equation*}

  \item \label{go(ii)} Write $\omega \ce \pi_2 E$, viewed as a free module 
  with one generator $u$ over $E_0 \cong \BZ_9 \llbracket h \rrbracket$.  
  Define $\omega$ as a left $\g$--module, compatible with its $E_0$--module 
  structure, by 
  \[
   q_k \cdot u \ce \left\{
   \begin{array}{ll}
     u,  & \quad {\rm if}~k = 1, \\
     0,  & \quad {\rm if}~k \neq 1.  
   \end{array}
   \right.
  \]
 \end{enumerate}
\end{defn}

\begin{rmk}
\label{rmk:rank}
 In \go{i}, an element $r \in \BZ_9 \llbracket h \rrbracket \cong E_0$ 
 corresponds to the multiplication-by-$r$ operation (see Rezk 
 \cite[discussion following Proposition 6.3]{cong}), and each $q_k$ 
 corresponds to the individual power operation $Q_k$ in \fullref{def:Q} (also 
 compare \go{ii} and \fullref{ex}).  Under this correspondence, the relations 
 in \q{ii}--\eqref{Q(v)} describe explicitly the structure of $\g$ as that of 
 a {\em graded twisted bialgebra over $E_0$} in the sense of 
 \cite[Section 5]{cong}.  The grading of $\g$ comes from the number of the 
 $q_k$'s in a monomial.  For example, commutation relations are in degree 1, 
 and Adem relations are in degree 2.  Under these relations, $\g$ has an 
 {\em admissible basis}: it is free as a left $E_0$--module on the elements 
 of the form 
 \[
  q_0^m q_{k_1} \cdots q_{k_n} 
 \]
 where $m, n \geq 0$ ($n = 0$ gives $q_0^m$), and $k_i = 1$, 2 or 3.  If we 
 write $\g[r]$ for the degree-$r$ part of $\g$, then $\g[r]$ is of rank 
 $1 + 3 + \cdots + 3^r$.  
\end{rmk}

We now identify $\g$ with the \DL algebra of power operations on 
$K(2)$--local commutative $E$--algebras.  

\begin{thm}
\label{thm:gamma}
 Let $A$ be a $K(2)$--local commutative $E$--algebra.  Let $\g$ be the graded 
 twisted bialgebra over $E_0$ in \go{i}, and $\omega$ be the $\g$--module in 
 \go{ii}.  Then $A_*$ has the structure of an {\em $\omega$--twisted 
 $\BZ/2$--graded amplified $\g$--ring} in the sense of Rezk 
 \cite[Section 2]{cong} and \cite[2.5 and 2.6]{h2p2}.  In particular, 
 \[
  \pi_* L_{K(2)} \BP_E (\Sigma^d E) \cong \big( F_d \big)_{(3,h)}^\wedge 
 \]
 where $F_d$ is the free graded amplified $\g$--ring with one generator in 
 dimension $d$.  
\end{thm}
Formulas for $\g$ aside, this result is due to Rezk \cite{cong, h2p2}.  
\begin{proof}
 Let $\G$ be the graded twisted bialgebra of power operations on $E_0$ in 
 Rezk \cite[Section 6]{cong}.  We need only identify $\G$ with $\g$.  

 There is a direct sum decomposition $\G = \bigoplus_{r \geq 0} \G[r]$ where 
 the summands come from the completed $E$--homology of $B\Sigma_{3^r}$ (see 
 \cite[6.2]{cong}).  As in \fullref{rmk:rank}, we have a degree-preserving 
 ring homomorphism 
 \[
  \phi \co \g \to \G, \qquad q_k \mapsto Q_k 
 \]
 which is an isomorphism in degrees 0 and 1.  We need to show that $\phi$ is 
 both surjective and injective in all degrees.  

 For the surjectivity of $\phi$, we use a transfer argument.  We have 
 \[
  \nu_3(|\Sigma_3^{\wr r}|) = \nu_3(|\Sigma_{3^r}|) = (3^r - 1) / 2 
 \]
 where $\nu_3(-)$ is the 3--adic valuation, and $(-)^{\wr r}$ is the $r$-fold 
 wreath product.  Thus following the proof of \cite[Proposition 3.17]{cong}, 
 we see that $\G$ is generated in degree 1, and hence $\phi$ is surjective.  

 By \fullref{rmk:rank} and Strickland \cite[Theorem 1.1]{Str98}, $\g[r]$ and 
 $\G[r]$ are of the same rank $1 + 3 + \cdots + 3^r$ as free modules over 
 $E_0$.  Hence $\phi$ is also injective.  
\end{proof}


\section{$K(1)$--local power operations}
\label{sec:K(1)}

Let $F \ce L_{K(1)} E$ be the $K(1)$--localization of $E$.  The following 
diagram describes the relationship between $K(1)$--local power operations on 
$F^0$ (cf.~Hopkins \cite[Section 3]{hopkins} and Bruner, May, McClure and 
Steinberger \cite[Section IX.3]{H_infty}) and the power operation on $E^0$ in 
\fullref{cor:psi3}: 
\begin{center}
\begin{tikzpicture}
        \node (LT) at (0, 2) {$E^0$}; 
        \node (RT) at (3, 2) {$E^0 B\Sigma_3 / I$}; 
        \node (LB) at (0, 0) {$F^0$}; 
        \node (MB) at (3, 0) {$F^0 B\Sigma_3 / J$}; 
        \node (RB) at (4.3, 0) {$\cong F^0$}; 
        \draw [->] (LT) -- node [above] {$\scriptstyle \p$} (RT); 
        \draw [->] (LT) -- (LB); 
        \draw [->] (RT) -- (MB); 
        \draw [->] (LB) -- node [above] {$\scriptstyle \psi_F^3$} (MB); 
\end{tikzpicture}
\end{center}
Here $\psi_F^3$ is the $K(1)$--local power operation induced by $\p$, and 
$J \cong F^0 \otimes_{E^0} I$ is the transfer ideal (cf.~\eqref{transfer}).  
Recall from \isog{i}, \eqref{S_3} and \fullref{cor:psi3} that $\p$ arises 
from the universal degree-3 isogeny which is parametrized by the ring $\s_3$ 
with 
\[
 \big( S_3 \big)_{(3,h)}^\wedge \cong E^0 B\Sigma_3 / I.  
\]
The vertical maps are induced by the $K(1)$--localization $E \to F$.  In 
terms of homotopy groups, this is obtained by inverting the generator $h$ and 
completing at the prime 3 (see Hovey \cite[Corollary 1.5.5]{hovey}): 
\[
 E_* = \BZ_9 \llbracket h \rrbracket [u^{\pm1}] \qquad \ad \qquad F_* = \BZ_9 \llbracket h \rrbracket [h^{-1}]_3^\wedge [u^{\pm1}] 
\]
with 
\[
 F_0 = \BZ_9 (\!(h)\!)_3^\wedge = \left.\left\{\sum_{n = -\infty}^{\infty} k_n h^n~\right|~k_n \in \BZ_9, \lim_{n \to -\infty} k_n = 0\right\}.  
\]
The formal group $\HC$ over $E^0$ has a unique order-3 subgroup after being 
pulled back to $F^0$, and the map 
\[
 E^0 B\Sigma_3 / I \to F^0 B\Sigma_3 / J \cong F^0 
\]
classifies this subgroup via the Serre--Tate theorem (see Katz and Mazur 
\cite[Theorem 2.9.1]{KM}).\footnote{Strickland's theorem 
\cite[Theorem 1.1]{Str98} does not apply here, as this map is not a local 
homomorphism; cf.~Mazel-Gee, Peterson and Stapleton \cite{MGPS}.  }  Along 
the base change 
\[
 E^0 B\Sigma_3 / I \to F^0 \otimes_{E^0} (E^0 B\Sigma_3 / I) \cong (F^0 \otimes_{E^0} E^0 B\Sigma_3) / J \cong F^0 B\Sigma_3 / J, 
\]
the special fiber of the 3--divisible group of $\HC$ which consists solely of 
a formal component may split into formal and \'etale components.  We want to 
take the formal component so as to keep track of the unique order-3 subgroup 
of the formal group over $F^0$.  This subgroup gives rise to the 
$K(1)$--local power operation $\psi_F^3$.  

Recall from \eqref{S_3} that $S_3 = S[\A] \big/ \big( w(\A) \big)$.  Since 
\[
 w(\A) = \A^4 - 6 \A^2 + (h - 9) \A - 3 \equiv \A (\A^3 + h) \md 3, 
\]
the equation $w(\A) = 0$ has a unique root $\A = 0$ in $\BF_9 (\!(h)\!)$.  By 
Hensel's lemma this unique root lifts to a root in 
$\BZ_9 (\!(h)\!)_3^\wedge$; it corresponds to the unique order-3 subgroup of 
$\HC$ over $F^0$.  Plugging this specific value of $\A$ into the formulas for 
$\p$ in \fullref{cor:psi3}, we then get an endomorphism of the ring $F^0$.  
This endomorphism is the $K(1)$--local power operation $\psi_F^3$.  

Explicitly, with $h$ invertible in $F^0$, we solve for $\A$ from $w(\A) = 0$ 
by first writing 
\[
 \A = (3 + 6 \A^2 - \A^4) / (h - 9) = (3 + 6 \A^2 - \A^4) \sum_{n = 1}^\infty 9^{n-1} h^{-n} 
\]
and then substituting this equation into itself recursively.  We plug the 
power series expansion for $\A$ into $\p(h)$ and get 
\[
 \psi_F^3(h) = h^3 - 27 h^2 + 183 h - 180 + 186 h^{-1} + 1674 h^{-2} + (\text{lower-order terms}).  ~~~
\]
Similarly, writing $h$ as $c^2 + 1$ in $w(\A) = 0$, we solve for $\A$ in 
terms of $c$ and get 
\[
 \psi_F^3(c) = c^3 - 12 c - 6 c^{-1} - 84 c^{-3} - 933 c^{-5} - 10956 c^{-7} + (\text{lower-order terms}).  
\]


\appendix
\section*{}
\addcontentsline{toc}{appendix}{Appendices}
\section*{Appendices}

Here we list long formulas whose appearance in the main body might affect 
readability.  The calculations involve power series expansions and 
manipulations of long polynomials with large coefficients (division, 
factorization and finding greatest common divisors).  They are done using 
the software {\em Wolfram Mathematica 8}.  The commands \texttt{Reduce} and 
\texttt{Solve} are used to extract relations out of given identities.  


\section{Formulas in the proof of Proposition 2.2}
\label{apx:tors}

In the proof of \fullref{prop:tors}, we have 
\begin{equation*}
\begin{split}
 \Tf(u) = & -\frac{u^4}{a^2 b} \big( b^4 u^8 + 3 a b^3 u^7 + 3 a^2 b^2 u^6 + (a^3 b + 7 a b^2) u^5 + (6 a^2 b - 6 b^2) u^4 \\
          & + 9 a b u^3 + (-a^2 + 8 b) u^2 - 3 a u - 3 \big), \\
 Q_1(v) = & ~ a b^2 v^2 + (b^2 d^2 + 2 a b d - b) v + \frac{b^2 d^4}{a} + 2 b d^3 + a d^2 - \frac{2 b d^2}{a} - d + \frac{1}{a}, \\
 R_1(v) = & ~ (\frac{b^3 d^6}{a} + 2 b^2 d^5 + a b d^4 - \frac{3 b^2 d^4}{a} + 2 b d^3 + \frac{3 b d^2}{a} - \frac{1}{a}) v + \frac{b^2 d^7}{a} + 2 b d^6 \qquad \\
          & + a d^5 - \frac{2 b d^5}{a} + 2 d^4 + \frac{d^3}{a}, 
\end{split}
\end{equation*}
\begin{equation*}
\begin{split}
 Q_2(v) = & ~ \frac{a}{(b^3 d^6 + 2 a b^2 d^5 + a^2 b d^4 - 3 b^2 d^4 + 2 a b d^3 + 3 b d^2 - 1)^2} \big( (a b^4 d^6 + 2 a^2 b^3 d^5 \\
          & + a^3 b^2 d^4 - 3 a b^3 d^4 + 2 a^2 b^2 d^3 + 3 a b^2 d^2 - a b) v - b^4 d^8 - 2 a b^3 d^7 - a^2 b^2 d^6 \\
          & + 4 b^3 d^6 - a b^2 d^5 + a^2 b d^4 - 6 b^2 d^4 + 4 a b d^3 + 4 b d^2 - a d - 1 \big), \\
    R_2 = & - \frac{a d^4}{(b^3 d^6 + 2 a b^2 d^5 + a^2 b d^4 - 3 b^2 d^4 + 2 a b d^3 + 3 b d^2 - 1)^2} (b^4 d^8 + 3 a b^3 d^7 \\
          & + 3 a^2 b^2 d^6 + a^3 b d^5 + 7 a b^2 d^5 + 6 a^2 b d^4 - 6 b^2 d^4 + 9 a b d^3 - a^2 d^2 + 8 b d^2 \\
          & - 3 a d - 3), \\
   K(u) = & ~ \frac{b^3 u^6}{a} + 2 b^2 u^5 + (a b - \frac{3 b^2}{a}) u^4 + 2 b u^3 + \frac{3 b u^2}{a} - \frac{1}{a}, \\
   L(u) = & ~ \frac{b^2 u^7}{a} + 2 b u^6 + (a - \frac{2 b}{a}) u^5 + 2 u^4 + \frac{u^3}{a}, \\
   M(u) = & ~ \frac{b}{a^2 (a^2 - 16 b)^2} \big( (10 a^3 b^3 - 112 a b^4) u^5 + (19 a^4 b^2 - 217 a^2 b^3 - 16 b^4) u^4 \\
          & + (8 a^5 b - 126 a^3 b^2 + 304 a b^3) u^3 + (-a^6 + 34 a^4 b -266 a^2 b^2 +32 b^3) u^2 \\
          & + (28 a^3 b - 384 a b^2) u - 4 a^4 + 51 a^2 b - 16 b^2 \big), \\
   N(u) = & -\frac{1}{a (a^2 - 16 b)^2} \big( (10 a^3 b^5 - 112 a b^6) u^7 + (29 a^4 b^4 - 329 a^2 b^5 - 16 b^6) u^6 \\
          & + (27 a^5 b^3 - 313 a^3 b^4 - 48 a b^5 ) u^5 + (7 a^6 b^2 - 15 a^4 b^3 - 837 a^2 b^4 - 16 b^5) u^4 \\
          & + (-a^7 b + 66 a^5 b^2 - 714 a^3 b^3 + 528 a b^4) u^3 + (-4 a^6 b + 137 a^4 b^2 \\
          & - 1147 a^2 b^3 + 80 b^4) u^2 + (-12 a^5 b + 237 a^3 b^2 - 1200 a b^3) u + a^6 - 44 a^4 b \\
          & + 409 a^2 b^2 - 48 b^3 \big).  
\end{split}
\end{equation*}


\section{Formulas in the proof of Proposition 2.3}
\label{apx:isog}

In the proof of \fullref{prop:isog}, the power series expansion of $v$ in 
terms of $u$ (up to $u^{12}$) is 
\begin{equation*}
\begin{split}
 v = & ~ u^3 - a u^4 + (a^2 + b) u^5 + (-a^3 - 3 a b) u^6 + (a^4 + 6 a^2 b + b^2) u^7 + (-a^5 - 10 a^3 b \\
     & - 6 a b^2) u^8 + (a^6 + 15 a^4 b + 20 a^2 b^2 + b^3) u^9 + (-a^7 - 21 a^5 b - 50 a^3 b^2 \\
     & - 10 a b^3) u^{10} + (a^8 + 28 a^6 b + 105 a^4 b^2 + 50 a^2 b^3 + b^4) u^{11} + (-a^9 - 36 a^7 b \\
     & - 196 a^5 b^2 - 175 a^3 b^3 - 15 a b^4) u^{12}.  
\end{split}
\end{equation*}

The group law on $C$ satisfies: 
\begin{itemize}
 \item Given $P(u,v)$, the coordinates of $-P$ are 
 \[
  \left( -\frac{v}{u (u + b v)},-\frac{v^2}{u^2 (u + b v)} \right); 
 \]

 \item Given $P_1(u_1,v_1)$ and $P_2(u_2,v_2)$, the coordinates of 
 $-(P_1 + P_2)$ are 
 \[
  u_3 \ce a k - \frac{b m}{1 + b k} - u_1 - u_2 \qquad \ad \qquad v_3 \ce k u_3 + m 
 \]
 where 
 \[
  k = \frac{v_1 - v_2}{u_1 - u_2} \qquad \ad \qquad m = \frac{u_1 v_2 - u_2 v_1}{u_1 - u_2}.  
 \]
\end{itemize}
Given $P(u,v)$ and $Q(d,e)$, with the above notations and formulas, we have: 
\begin{itemize}
 \item Set 
 \[
  (u_1,v_1) = \left( -\frac{v}{u (u + b v)},-\frac{v^2}{u^2 (u + b v)} \right) \qquad \ad \qquad (u_2,v_2) = (d,e) 
 \]
 so that 
 \[
  P - Q = (u_3,v_3); 
 \]

 \item Set 
 \[
  (u_1,v_1) = (u,v) \qquad \ad \qquad (u_2,v_2) = (d,e) 
 \]
 so that 
 \[
  P + Q = \left( -\frac{v_3}{u_3 (u_3 + b v_3)},-\frac{v_3^2}{u_3^2 (u_3 + b v_3)} \right).  
 \]
\end{itemize}
Plugging the coordinates of $P - Q$ and $P + Q$ into \eqref{u'v'}, in view of 
\eqref{f}, we have in \eqref{KL} 
\begin{equation*}
\begin{split}
      \K = & -\frac{1}{a^2 - 16 b} \big( a b^3 d^7 + (3 a^2 b^2 - 2 b^3) d^6 + (3 a^3 b - 6 a b^2) d^5 + (a^4 + a^2 b + 2 b^2) d^4 \\
           & + (4 a^3 - 15 a b) d^3 + (a^2 + 2 b) d^2 - 12 a d - 18 \big), \\
 \lambda = & -\frac{1}{a^2 b^2 (a^2 - 16 b)} \big( (a^3  b^3 - 11 a b^4) d^7 + (3 a^4 b^2 - 33 a^2 b^3 - 4 b^4) d^6 + (3 a^5 b \\
           & - 33 a^3 b^2 - 15 a b^3) d^5 + (a^6 - 4 a^4 b - 96 a^2 b^2 - 4 b^3) d^4 + (6 a^5 - 80 a^3 b \\
           & + 31 a b^2) d^3 + (10 a^4 - 153 a^2 b + 20 b^2) d^2 + (3 a^3 - 117 a b) d - 6 a^2 - 12 b \big).  
\end{split}
\end{equation*}
More extended power series expansions in $u$ for $u'$ (up to $u^6$) and $v'$ 
(up to $u^9$) are needed in \eqref{KL} to determine the coefficients in the 
equation of $C'$: 
\begin{equation*}
\begin{split}
 u' = & -\frac{1}{a^2 - 16 b} \big( (a b^3 d^7 + 3 a^2 b^2 d^6 - 2 b^3 d^6 + 3 a^3 b d^5 - 6 a b^2 d^5 + a^4 d^4 + a^2 b d^4 \\
      & + 2 b^2 d^4 + 4 a^3 d^3 - 15 a b d^3 + a^2 d^2 + 2 b d^2 - 12 a d - 18) u + (-a^2 b^3 d^7 \\
      & + 12 b^4 d^7 - 3 a^3 b^2 d^6 + 36 a b^3 d^6 - 3 a^4 b d^5 + 36 a^2 b^2 d^5 + 4 b^3 d^5 - a^5 d^4 \\
      & + 5 a^3 b d^4 + 94 a b^2 d^4 - 6 a^4 d^3 + 85 a^2 b d^3 - 76 b^2 d^3 - 9 a^3 d^2 + 136 a b d^2 + 60 b d \\
      & + 6 a) u^2 + (a^3 b^3 d^7 - 17 a b^4 d^7 + 3 a^4 b^2 d^6 - 50 a^2 b^3 d^6 - 8 b^4 d^6 + 3 a^5 b d^5 \\
      & - 48 a^3 b^2 d^5 - 27 a b^3 d^5 + a^6 d^4 - 7 a^4 b d^4 - 150 a^2 b^2 d^4 - 16 b^3 d^4 + 7 a^5 d^3 \\
      & - 113 a^3 b d^3 + 9 a b^2 d^3 + 16 a^4 d^2 - 258 a^2 b d^2 + 56 b^2 d^2 + 15 a^3 d - 237 a b d \\
      & + 2 a^2 - 32 b) u^3 + (-a^4 b^3 d^7 + 16 a^2 b^4 d^7 + 12 b^5 d^7 - 3 a^5 b^2 d^6 + 46 a^3 b^3 d^6 \\
      & + 64 a b^4 d^6 - 3 a^6 b d^5 + 42 a^4 b^2 d^5 + 121 a^2 b^3 d^5 + 4 b^4 d^5 - a^7 d^4 + 3 a^5 b d^4 \\
      & + 209 a^3 b^2 d^4 + 122 a b^3 d^4 - 8 a^6 d^3 + 114 a^4 b d^3 + 248 a^2 b^2 d^3 - 76 b^3 d^3 \\
      & - 24 a^5 d^2 + 384 a^3 b d^2 - 4 a b^2 d^2 - 33 a^4 d + 519 a^2 b d + 60 b^2 d - 18 a^3 \\
      & + 282 a b) u^4 + (a^5 b^3 d^7 - 9 a^3 b^4 d^7 - 117 a b^5 d^7 + 3 a^6 b^2 d^6 - 24 a^4 b^3 d^6 \\
      & - 396 a^2 b^4 d^6 - 24 b^5 d^6 + 3 a^7 b d^5 - 18 a^5 b^2 d^5 - 484 a^3 b^3 d^5 - 111 a b^4 d^5 + a^8 d^4 \\
      & + 7 a^6 b d^4 - 307 a^4 b^2 d^4 - 1038 a^2 b^3 d^4 + 9 a^7 d^3 - 73 a^5 b d^3 - 1181 a^3 b^2 d^3 \\
      & + 573 a b^3 d^3 + 33 a^6 d^2 - 451 a^4 b d^2 - 1236 a^2 b^2 d^2 + 72 b^3 d^2 + 54 a^5 d \\
      & - 807 a^3 b d - 873 a b^2 d + 36 a^4 - 570 a^2 b - 48 b^2) u^5 + (-a^6 b^3 d^7 - 5 a^4 b^4 d^7 \\
      & + 337 a^2 b^5 d^7 + 12 b^6 d^7 - 3 a^7 b^2 d^6 - 19 a^5 b^3 d^6 + 1064 a^3 b^4 d^6 + 204 a b^5 d^6 \\
      & - 3 a^8 b d^5 - 27 a^6 b^2 d^5 + 1164 a^4 b^3 d^5 + 638 a^2 b^4 d^5 + 4 b^5 d^5 - a^9 d^4 - 24 a^7 b d^4 \\
      & + 441 a^5 b^2 d^4 + 3195 a^3 b^3 d^4 + 182 a b^4 d^4 - 10 a^8 d^3 - 22 a^6 b d^3 + 2956 a^4 b^2 d^3 \\
      & - 645 a^2 b^3 d^3 - 76 b^4 d^3 - 43 a^7 d^2 + 403 a^5 b d^2 + 4594 a^3 b^2 d^2 - 544 a b^3 d^2 \\
      & - 78 a^6 d + 996 a^4 b d + 4014 a^2 b^2 d + 60 b^3 d - 57 a^5 + 852 a^3 b + 942 a b^2) u^6 \big), \\
~v' = & -\frac{1}{a^2 b^2 (a^2 - 16 b)} \big( (a^3 b^3 d^7 - 11 a b^4 d^7 + 3 a^4 b^2 d^6 - 33 a^2 b^3 d^6 - 4 b^4 d^6 \\
      & + 3 a^5 b d^5 - 33 a^3 b^2 d^5 - 15 a b^3 d^5 + a^6 d^4 - 4 a^4 b d^4 - 96 a^2 b^2 d^4 - 4 b^3 d^4 \\
      & + 6 a^5 d^3 - 80 a^3 b d^3 + 31 a b^2 d^3 + 10 a^4 d^2 - 153 a^2 b d^2 + 20 b^2 d^2 + 3 a^3 d \\
      & - 117 a b d - 6 a^2 - 12 b) u^3 + (-2 a^4 b^3 d^7 + 28 a^2 b^4 d^7 - 6 a^5 b^2 d^6 + 82 a^3 b^3 d^6 \\
      & + 28 a b^4 d^6 - 6 a^6 b d^5 + 78 a^4 b^2 d^5 + 90 a^2 b^3 d^5 - 2 a^7 d^4 + 8 a^5 b d^4 + 294 a^3 b^2 d^4 \\
      & + 20 a b^3 d^4 - 14 a^6 d^3 + 202 a^4 b d^3 + 72 a^2 b^2 d^3 - 32 a^5 d^2 + 510 a^3 b d^2 \\
      & - 124 a b^2 d^2 - 30 a^4 d + 546 a^2 b d - 6 a^3 + 204 a b) u^4 + (3 a^5 b^3 d^7 - 38 a^3 b^4 d^7 
\end{split}
\end{equation*}
\begin{equation*}
\begin{split}
\qquad& - 107 a b^5 d^7 + 9 a^6 b^2 d^6 - 108 a^4 b^3 d^6 - 409 a^2 b^4 d^6 - 4 b^5 d^6 + 9 a^7 b d^5 \\
      & - 96 a^5 b^2 d^5 - 590 a^3 b^3 d^5 - 47 a b^4 d^5 + 3 a^8 d^4 + a^6 b d^4 - 646 a^4 b^2 d^4 \\
      & - 912 a^2 b^3 d^4 - 4 b^4 d^4 + 24 a^7 d^3 - 292 a^5 b d^3 - 1249 a^3 b^2 d^3 + 639 a b^3 d^3 \\
      & + 70 a^6 d^2 - 1057 a^4 b d^2 - 849 a^2 b^2 d^2 + 20 b^3 d^2 + 93 a^5 d - 1512 a^3 b d \\
      & - 597 a b^2 d + 48 a^4 - 870 a^2 b - 12 b^2) u^5 + (-4 a^6 b^3 d^7 + 24 a^4 b^4 d^7 + 583 a^2 b^5 d^7 \\
      & - 12 a^7 b^2 d^6 + 60 a^5 b^3 d^6 + 1923 a^3 b^4 d^6 + 156 a b^5 d^6 - 12 a^8 b d^5 + 36 a^6 b^2 d^5 \\
      & + 2268 a^4 b^3 d^5 + 639 a^2 b^4 d^5 - 4 a^9 d^4 - 40 a^7 b d^4 + 1256 a^5 b^2 d^4 + 5128 a^3 b^3 d^4 \\
      & + 140 a b^4 d^4 - 36 a^8 d^3 + 229 a^6 b d^3 + 5409 a^4 b^2 d^3 - 2227 a^2 b^3 d^3 - 127 a^7 d^2 \\
      & + 1597 a^5 b d^2 + 6835 a^3 b^2 d^2 - 748 a b^3 d^2 - 201 a^6 d + 2952 a^4 b d + 5277 a^2 b^2 d \\
      & - 129 a^5 + 2130 a^3 b + 708 a b^2) u^6 + (5 a^7 b^3 d^7 + 35 a^5 b^4 d^7 - 1754 a^3 b^5 d^7 \\
      & - 275 a b^6 d^7 + 15 a^8 b^2 d^6 + 125 a^6 b^3 d^6 - 5511 a^4 b^4 d^6 - 1833 a^2 b^5 d^6 - 4 b^6 d^6 \\
      & + 15 a^9 b d^5 + 165 a^7 b^2 d^5 - 5988 a^5 b^3 d^5 - 4312 a^3 b^4 d^5 - 103 a b^5 d^5 + 5 a^{10} d^4 \\
      & + 130 a^8 b d^4 - 2183 a^6 b^2 d^4 - 17022 a^4 b^3 d^4 - 2940 a^2 b^4 d^4 - 4 b^5 d^4 + 50 a^9 d^3 \\
      & + 159 a^7 b d^3 - 15035 a^5 b^2 d^3 + 179 a^3 b^3 d^3 + 1703 a b^4 d^3 + 206 a^8 d^2 \\
      & - 1708 a^6 b d^2 - 25304 a^4 b^2 d^2 + 1431 a^2 b^3 d^2 + 20 b^4 d^2 + 363 a^7 d - 4398 a^5 b d \\
      & - 23694 a^3 b^2 d - 1437 a b^3 d + 258 a^6 - 3816 a^4 b - 7026 a^2 b^2 - 12 b^3) u^7 \\
      & + (-6 a^8 b^3 d^7 - 164 a^6 b^4 d^7 + 3864 a^4 b^5 d^7 + 3365 a^2 b^6 d^7 - 18 a^9 b^2 d^6 \\
      & - 522 a^7 b^3 d^6 + 11837 a^5 b^4 d^6 + 13701 a^3 b^5 d^6 + 448 a b^6 d^6 - 18 a^{10} b d^5 \\
      & - 582 a^8 b^2 d^5 + 12275 a^6 b^3 d^5 + 21828 a^4 b^4 d^5 + 2395 a^2 b^5 d^5 - 6 a^{11} d^4 \\
      & - 296 a^9 b d^4 + 3283 a^7 b^2 d^4 + 43960 a^5 b^3 d^4 + 30290 a^3 b^4 d^4 + 424 a b^5 d^4 \\
      & - 66 a^{10} d^3 - 1099 a^8 b d^3 + 32246 a^6 b^2 d^3 + 30529 a^4 b^3 d^3 - 17045 a^2 b^4 d^3 \\
      & - 310 a^9 d^2 + 679 a^7 b d^2 + 66726 a^5 b^2 d^2 + 24833 a^3 b^3 d^2 - 2192 a b^4 d^2 - 588 a^8 d \\
      & + 4809 a^6 b d + 73578 a^4 b^2 d + 23685 a^2 b^3 d - 444 a^7 + 5316 a^5 b + 30936 a^3 b^2 \\
      & + 1704 a b^3) u^8 + (7 a^9 b^3 d^7 + 392 a^7 b^4 d^7 - 6863 a^5 b^5 d^7 - 17458 a^3 b^6 d^7 \\
      & - 515 a b^7 d^7 + 21 a^{10} b^2 d^6 + 1218 a^8 b^3 d^6 - 20647 a^6 b^4 d^6 - 61745 a^4 b^5 d^6 \\
      & - 6709 a^2 b^6 d^6 - 4 b^7 d^6 + 21 a^{11} b d^5 + 1302 a^9 b^2 d^5 - 20664 a^7 b^3 d^5 \\
      & - 81924 a^5 b^4 d^5 - 22146 a^3 b^5 d^5 - 183 a b^6 d^5 + 7 a^{12} d^4 + 567 a^{10} b d^4 \\
      & - 3982 a^8 b^2 d^4 - 97733 a^6 b^3 d^4 - 158644 a^4 b^4 d^4 - 8392 a^2 b^5 d^4 - 4 b^6 d^4 \\
      & + 84 a^{11} d^3 + 2878 a^9 b d^3 - 57242 a^7 b^2 d^3 - 160981 a^5 b^3 d^3 + 59447 a^3 b^4 d^3 \\
      & + 3223 a b^5 d^3 + 442 a^{10} d^2 + 2563 a^8 b d^2 - 142138 a^6 b^2 d^2 - 189134 a^4 b^3 d^2 
\end{split}
\end{equation*}
\begin{equation*}
\begin{split}
\qua  & + 18323 a^2 b^4 d^2 + 20 b^5 d^2 + 885 a^9 d - 2382 a^7 b d - 179958 a^5 b^2 d \\
      & - 164688 a^3 b^3 d - 2637 a b^4 d + 696 a^8 - 5400 a^6 b - 92938 a^4 b^2 - 29078 a^2 b^3 \\
      & - 12 b^4) u^9 \big).  
\end{split}
\end{equation*}

%%%%%%%%%%%%%%%%%%%%   End of main body of article
%
%                             References
%
%   BiBTeX users uncomment the following line:
%
\bibliographystyle{gtart}
%
\bibliography{draft}

%\begin{thebibliography}

%\end{thebibliography}

\end{document}