\documentclass[12pt]{article}
\usepackage{tikz}
\usepackage{pifont}
\usepgflibrary[arrows]
\usepackage{graphicx}
\usepackage{amsmath,amssymb,amsthm,stmaryrd}
\usepackage[all]{xy}
\xyoption{arc}
\usepackage{color}

\bibliographystyle{amsalpha}
\parskip 0.7pc
\parindent 0pt

\newtheorem{thm}{Theorem}[section]
\newtheorem{cor}[thm]{Corollary}
\newtheorem{prop}[thm]{Proposition}
\newtheorem{lem}[thm]{Lemma}
\theoremstyle{definition}
\newtheorem{defn}[thm]{Definition}
\theoremstyle{remark}
\newtheorem{rmk}[thm]{Remark}
\newtheorem{exam}[thm]{Example}
\newtheorem{case}[thm]{Case}
\newtheorem{remark}[thm]{Working Remark}

\def\co{\colon\thinspace}
\newcommand{\mb}[1]{\mathbb{#1}}
\newcommand{\mf}[1]{\mathfrak{#1}}

\newcommand{\NB}[1]{{\bf (NB: #1)}}

\newcommand{\Ext}{\ensuremath{{\rm Ext}}}
\newcommand{\Coext}{\ensuremath{{\rm Coext}}}
\newcommand{\Hom}{\ensuremath{{\rm Hom}}}
\newcommand{\Tor}{\ensuremath{{\rm Tor}}}
\newcommand{\Ind}{\ensuremath{{\rm Ind}}}
\newcommand{\Res}{\ensuremath{{\rm Res}}}
\newcommand{\colim}{\ensuremath{\mathop{\rm colim}}}
\newcommand{\hocolim}{\ensuremath{\mathop{\rm hocolim}}}
\newcommand{\holim}{\ensuremath{\mathop{\rm holim}}}
\newcommand{\overto}{\mathop\rightarrow}
\newcommand{\overfrom}{\mathop\leftarrow}
\newcommand{\into}{\mathop\hookrightarrow}
\newcommand{\longoverto}{\mathop{\longrightarrow}}
\newcommand{\Irr}{\ensuremath{\mathop{\rm Irr}}}
\newcommand{\Rep}{\ensuremath{\mathop{\rm Rep}}}
\newcommand{\Map}{\ensuremath{{\rm Map}}}
\newcommand{\GL}{{\rm GL}}
\newcommand{\Uni}{{\rm U}}
\newcommand{\Sp}{{\cal S}p}
\newcommand{\Sym}{{\rm Sym}}
\newcommand{\thh}{{\rm thh}}
\newcommand{\TAF}{{\rm TAF}}
\newcommand{\TAQ}{{\rm TAQ}}
\newcommand{\End}{{\rm End}}
\newcommand{\Aut}{{\rm Aut}}
\newcommand{\md}{{\rm mod}}
\newcommand{\As}{{\cal As}}
\newcommand{\Spec}{{\rm Spec}}
\newcommand{\Spf}{{\rm Spf}}
\newcommand{\tmf}{{\rm tmf}}
%\newcommand{\tmf}{\mathit{tmf}}
\newcommand{\ftmf}{{\rm ftmf}}
\newcommand{\FTMF}{{\rm FTMF}}
\newcommand{\TMF}{{\rm TMF}}
\newcommand{\Lie}{{\rm Lie}}
\newcommand{\Sh}{{\rm Sh}}
\newcommand{\HF}{{\rm H}{\mb F}}
\newcommand{\cF}{\overline {\mb F}}
\newcommand{\cQ}{\overline {\mb Q}}
\newcommand{\Tr}{{\rm Tr}}
\newcommand{\Nm}{{\rm N}}
\newcommand{\PGL}{{\rm PGL}}
\newcommand{\Cl}{{\rm Cl}}
\newcommand{\mass}[1]{\left|{#1}\right|}
\newcommand{\card}[1]{\#\left\{#1\right\}}
\newcommand{\lsym}[2]{\left\{\frac{#1}{#2}\right\}}

\newcommand{\Zos}{\mb Z[1/6]}
\newcommand{\Zoh}{\mb Z[1/2]}
\newcommand{\rt}{{{}^R \otimes}}
\newcommand{\lt}{{\otimes^L}}
\newcommand{\dt}{{{}^R \otimes^L}}
\newcommand{\divi}[1]{{\rm div}\left(#1\right)}
\newcommand{\dlog}[1]{{\rm dlog}\left(#1\right)}

\newcommand{\eilm}[1]{\ensuremath{{\mb H} #1}}
\newcommand{\smsh}[1]{\ensuremath{\mathop{\wedge}_{#1}}}
\newcommand{\tens}[1]{\ensuremath{\mathop{\otimes}_{#1}}}
\newcommand{\susp}{\ensuremath{\Sigma}}
\newcommand{\mapset}[3]{\ensuremath{\left[#2,#3\right]_{#1}}}
\newcommand{\form}[2]{\ensuremath{\left\langle#1,#2\right\rangle}}
\newcommand{\bilin}[2]{\ensuremath{\left(#1,#2\right)}}
\newcommand{\comp}[1]{\ensuremath{#1^\wedge}}
\newcommand{\loca}[3]{\ensuremath{L^{#1}_{#2}(#3)}}
\newcommand{\tc}[3]{\ensuremath{\Omega_{#2 / #1}^{#3}}}
\newcommand{\atc}[3]{\ensuremath{\L_{#2 / #1}^{#3}}}
\newcommand{\pow}[1]{\left\llbracket{#1}\right\rrbracket}

\newcommand{\xym}[1]{
\vskip 0.7pc
\centerline{\xymatrix{#1}}
\vskip 0.7pc
}

\newcommand{\xynm}[1]
{\vskip 0.7pc
\centerline{\xy #1 \endxy}
\vskip 0.7pc}

\begin{document}

Consider the universal elliptic curve with a choice of 4-torsion point, given 
by the Weierstrass equation 
\[
 y^2 + a x y + a c y = x^3 + c x^2 
\]
over the graded ring ${\mb Z}_{(3)}[a,c]$, where $|a| = 1$ and $|c| = 2$.  To 
facilitate calculations, we work in the affine coordinate chart $c = 1$ of the 
moduli stack ${\cal M}\big(\Gamma_1(4)\big)$ associated to the level 
$\Gamma_1(4)$-structure on the curve. It is then given by 
\[
 y^2 + a x y + a y = x^3 + x^2, 
\]
with discriminant $\Delta = a^2(a + 4)(a - 4)$ and Hasse invariant 
$h = a^2 + 4$.  In $uv$-coodinates, with $u = x/y$ and $v = 1/y$, the 
equation becomes 
\[
 v + a u v + a v^2 = u^3 + u^2 v.  
\]
We denote this elliptic curve by $\cal E$.  

Let $F$ be the Morava $E$-theory associated to the restriction of $\cal E$ to 
the supersingular locus, so that $F^0 = {\mb Z}_9 \llbracket h \rrbracket$.  

\begin{enumerate}
 \item The universal degree 3 isogeny with source $\cal E$ is defined over 
 the ring 
 \[
  F^0 \llbracket t \rrbracket / 
  \big(a^2 t^4 + 3 a^2 t^3 + 3 a^2 t^2 + (a^2 + 4) t + 3\big), 
 \]
 and has target the elliptic curve 
 \[
  y^2 + r x y + r y = x^3 + x^2, 
 \]
 where 
 \[
  r(a,t) = a^3 t^3 + 3 a^3 t^2 + 3 a^3 t - 4 a t + a^3 - 3 a.  
 \]
 The kernel of this isogeny is generated by the 3-torsion point whose 
 $x$-coordinate is $1/t$.  
 \item The power operation $\psi^3\co F^0  \to F^0 \llbracket t \rrbracket / 
 \big(a^2 t^4 + 3 a^2 t^3 + 3 a^2 t^2 + (a^2 + 4) t + 3\big)$ is given by 
 \[
  \psi^3(h) = (t + 1)^3 h^3 - (22 t^3 + 69 t^2 + 75 t + 27) h^2 + (128 t^3 + 
  424 t^2 + 512 t + 201) h 
 \]
 \[
  - 16 (14 t^3 + 49 t^2 + 65 t + 27), 
  ~~~~~~~~~~~~~~~~~~~~~~~~~~~~~~~~ 
 \]
 \[
  \psi^3(a) = (t + 1)^3 a^3 - (4 t + 3) a.  
  ~~~~~~~~~~~~~~~~~~~~~~~~~~~~~~~~~~~~~~~~~~~~~~~~~~~~~ 
 \]
\end{enumerate}

Let $A$ be a $K(2)$-local commutative $F$-algebra.  Define 
$Q_0, Q_1, Q_2, Q_3\co\pi_0A \to \pi_0A$ by 
\[
 \psi^3 (x) = Q_0(x) + Q_1(x) \alpha + Q_2(x) \alpha^2 +Q_3(x) \alpha^3, 
\]
where $\alpha$ satisfies 
\[
 \alpha^4 - 6\alpha^2 + (a^2-8)\alpha-3 = 0.  
\]
($\alpha$ appears as the coefficient of $u$ in the expression of $u'$, 
comparable to the coefficient $d$ on page 6 of Rezk's paper ``Power operations for 
Morava $E$-theory of height 2 at the prime 2''.  It is invariant under change 
of coordinates.  The formulas in 1 and 2 above can be written in terms of 
$\alpha$ instead of $t$.  For the formulas below, writing in terms of $t$ 
will introduce more denominators, and it will be hard to find the Adem 
relations.  $\alpha$ and $t$ are related by $\alpha = a^2 t^3 + 2 a^2 t^2 + a^2 t + 3$, and $t = -\alpha/(\alpha + 1)$, i.e.\thinspace$(t + 1) (\alpha + 1) = 1$.)

Commutation relations: 

$Q_0(h x) = (h^3 - 36 h^2 + 390 h - 1212) Q_0(x) + (3 h^2 - 72 h + 360) Q_1(x) + (9 h - 108) Q_2(x) + 24 Q_3(x)$, 

$Q_1(h x) = (-6 h^2 + 144 h - 712) Q_0(x) + (-18 h + 228) Q_1(x) + (-72) Q_2(x) + (h - 12) Q_3(x)$, 

$Q_2(h x) = (3 h - 36) Q_0(x) + 8 Q_1(x) + 12 Q_2(x) + (-24) Q_3(x)$, 

$Q_3(h x) = (h^2 - 24 h + 120) Q_0(x) + (3 h - 36) Q_1(x) + 8 Q_2(x) + 12 Q_3(x)$; 

$Q_0(a x) = (a^3 - 12 a + 12 a^{-1}) Q_0(x) + (3 a - 12 a^{-1}) Q_1(x) + (12 a^{-1}) Q_2(x) + (-12 a^{-1}) Q_3(x)$, 

$Q_1(a x) = (-6 a + 20 a^{-1}) Q_0(x) + (-20 a^{-1}) Q_1(x) + (- a + 20 a^{-1}) Q_2(x) + (4 a - 20 a^{-1}) Q_3(x)$, 

$Q_2(a x) = (4 a^{-1}) Q_0(x) + (-4 a^{-1}) Q_1(x) + (4 a^{-1}) Q_2(x) + (- a - 4 a^{-1}) Q_3(x)$, 

$Q_3(a x) = (a - 4 a^{-1}) Q_0(x) + (4 a^{-1}) Q_1(x) + (-4 a^{-1}) Q_2(x) + (4 a^{-1}) Q_3(x)$.  

Adem relations: 

$\alpha' = -\alpha^3 + 6 \alpha + (-h + 12)$;

$\Psi(x) = Q_0Q_0(x) + (-h + 12) Q_0Q_1(x) + (h^2 - 24 h + 126) Q_0Q_2(x) + (-3) Q_1Q_1(x) + (-h^3 + 36 h^2 - 396 h + 1296) Q_0Q_3(x) + (3h - 36) Q_1Q_2(x) + (-3 h^2 + 72 h - 378) Q_1Q_3(x) + 9 Q_2Q_2(x) + (-9 h + 108) Q_2Q_3(x) + (-27) Q_3Q_3(x)$;

$Q_1Q_0(x) = (-6) Q_0Q_1(x) + (6 h - 72) Q_0Q_2(x) + (-6 h^2 + 144 h - 747) Q_0Q_3(x) + 18 Q_1Q_2(x) + 3 Q_2Q_1(x) + (-18 h + 216) Q_1Q_3(x) + (-54) Q_2Q_3(x) + (-9) Q_3Q_2(x)$, 

$Q_2Q_0(x) = (-3) Q_0Q_2(x) + (3 h - 36) Q_0Q_3(x) + 9 Q_1Q_3(x) + 3 Q_3Q_1(x)$, 

$Q_3Q_0(x) = Q_0Q_1(x) + (-h + 12) Q_0Q_2(x) + (h^2 - 24 h + 126) Q_0Q_3(x) + (-3) Q_1Q_2(x) + (3 h - 36) Q_1Q_3(x) + 9 Q_2Q_3(x)$.  

Cartan formula: 

$Q_0(xy) = Q_0(x) Q_0(y) + 3 [Q_1(x) Q_3(y) + Q_2(x) Q_2(y) + Q_3(x) Q_1(y)] + 18 Q_3(x) Q_3(y)$, 

$Q_1(xy) = [Q_0(x) Q_1(y) + Q_1(x) Q_0(y)] + (-h + 12) [Q_1(x) Q_3(y) + Q_2(x) Q_2(y) + Q_3(x) Q_1(y)] + 3 [Q_2(x) Q_3(y) + Q_3(x) Q_2(y)] + (-6h + 72) Q_3(x) Q_3(y)$, 

$Q_2(xy) = [Q_0(x) Q_2(y) + Q_1(x) Q_1(y) + Q_2(x) Q_0(y)] + 6 [Q_1(x) Q_3(y) + Q_2(x) Q_2(y) + Q_3(x) Q_1(y)] + (-h + 12) [Q_2(x) Q_3(y) + Q_3(x) Q_2(y)] + 39 Q_3(x) Q_3(y)$, 

$Q_3(xy) = [Q_0(x) Q_3(y) + Q_1(x) Q_2(y) + Q_2(x) Q_1(y) + Q_3(x) Q_0(y)] + 6 [Q_2(x) Q_3(y) + Q_3(x) Q_2(y)] + (-h + 12) Q_3(x) Q_3(y)$.  

Additivity: 

$Q_i(x+y) = Q_i(x) + Q_i(y)$.  

Action on scalars: 

$Q_0(1) = 1, Q_1(1) = Q_2(1) = Q_3(1) = 0$; 

$Q_0(h) = h^3 - 36 h^2 + 390 h - 1212$, 

$Q_1(h) = -6 h^2 + 144 h - 712$, 

$Q_2(h) = 3 h - 36$, 

$Q_3(h) = h^2 - 24 h + 120$; 

$Q_0(a) = a^3 - 12 a + 12 a^{-1}$, 

$Q_1(a) = -6 a + 20 a^{-1}$, 

$Q_2(a) = 4 a^{-1}$, 

$Q_3(a) = a - 4 a^{-1}$.  

Frobenius congruence: 

$Q_0(x) \equiv x^3~\text{mod}~3$,

$\theta\co\pi_0A \to \pi_0A$ such that $Q_0(x) = x^3 + 3 \theta(x)$.  
\end{document}