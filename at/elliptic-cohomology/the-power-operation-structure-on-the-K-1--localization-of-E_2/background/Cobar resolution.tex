\documentclass[12pt]{article}
\usepackage{graphicx}
\usepackage{amsmath,amssymb,amsthm,stmaryrd}
\usepackage[all]{xy}
\xyoption{arc}

\bibliographystyle{amsalpha}
\parskip 0.7pc
\parindent 0pt

\newtheorem{thm}{Theorem}[section]
\newtheorem{cor}[thm]{Corollary}
\newtheorem{prop}[thm]{Proposition}
\newtheorem{lem}[thm]{Lemma}
\theoremstyle{definition}
\newtheorem{defn}[thm]{Definition}
\theoremstyle{remark}
\newtheorem{rmk}[thm]{Remark}
\newtheorem{exam}[thm]{Example}
\newtheorem{case}[thm]{Case}
\newtheorem{remark}[thm]{Working Remark}

\def\co{\colon\thinspace}
\newcommand{\mb}[1]{\mathbb{#1}}
\newcommand{\mf}[1]{\mathfrak{#1}}

\newcommand{\NB}[1]{{\bf (NB: #1)}}

\newcommand{\Ext}{\ensuremath{{\rm Ext}}}
\newcommand{\Coext}{\ensuremath{{\rm Coext}}}
\newcommand{\Hom}{\ensuremath{{\rm Hom}}}
\newcommand{\Tor}{\ensuremath{{\rm Tor}}}
\newcommand{\Ind}{\ensuremath{{\rm Ind}}}
\newcommand{\Res}{\ensuremath{{\rm Res}}}
\newcommand{\colim}{\ensuremath{\mathop{\rm colim}}}
\newcommand{\hocolim}{\ensuremath{\mathop{\rm hocolim}}}
\newcommand{\holim}{\ensuremath{\mathop{\rm holim}}}
\newcommand{\overto}{\mathop\rightarrow}
\newcommand{\overfrom}{\mathop\leftarrow}
\newcommand{\into}{\mathop\hookrightarrow}
\newcommand{\longoverto}{\mathop{\longrightarrow}}
\newcommand{\Irr}{\ensuremath{\mathop{\rm Irr}}}
\newcommand{\Rep}{\ensuremath{\mathop{\rm Rep}}}
\newcommand{\Map}{\ensuremath{{\rm Map}}}
\newcommand{\GL}{{\rm GL}}
\newcommand{\Uni}{{\rm U}}
\newcommand{\Sp}{{\cal S}p}
\newcommand{\Sym}{{\rm Sym}}
\newcommand{\thh}{{\rm thh}}
\newcommand{\TAF}{{\rm TAF}}
\newcommand{\TAQ}{{\rm TAQ}}
\newcommand{\End}{{\rm End}}
\newcommand{\Aut}{{\rm Aut}}
\newcommand{\md}{{\rm mod}}
\newcommand{\As}{{\cal As}}
\newcommand{\Spec}{{\rm Spec}}
\newcommand{\Spf}{{\rm Spf}}
\newcommand{\tmf}{{\rm tmf}}
%\newcommand{\tmf}{\mathit{tmf}}
\newcommand{\ftmf}{{\rm ftmf}}
\newcommand{\FTMF}{{\rm FTMF}}
\newcommand{\TMF}{{\rm TMF}}
\newcommand{\Lie}{{\rm Lie}}
\newcommand{\Sh}{{\rm Sh}}
\newcommand{\HF}{{\rm H}{\mb F}}
\newcommand{\cF}{\overline {\mb F}}
\newcommand{\cQ}{\overline {\mb Q}}
\newcommand{\Tr}{{\rm Tr}}
\newcommand{\Nm}{{\rm N}}
\newcommand{\PGL}{{\rm PGL}}
\newcommand{\Cl}{{\rm Cl}}
\newcommand{\mass}[1]{\left|{#1}\right|}
\newcommand{\card}[1]{\#\left\{#1\right\}}
\newcommand{\lsym}[2]{\left\{\frac{#1}{#2}\right\}}

\newcommand{\Zos}{\mb Z[1/6]}
\newcommand{\Zoh}{\mb Z[1/2]}
\newcommand{\rt}{{{}^R \otimes}}
\newcommand{\lt}{{\otimes^L}}
\newcommand{\dt}{{{}^R \otimes^L}}
\newcommand{\divi}[1]{{\rm div}\left(#1\right)}
\newcommand{\dlog}[1]{{\rm dlog}\left(#1\right)}

\newcommand{\eilm}[1]{\ensuremath{{\mb H} #1}}
\newcommand{\smsh}[1]{\ensuremath{\mathop{\wedge}_{#1}}}
\newcommand{\tens}[1]{\ensuremath{\mathop{\otimes}_{#1}}}
\newcommand{\susp}{\ensuremath{\Sigma}}
\newcommand{\mapset}[3]{\ensuremath{\left[#2,#3\right]_{#1}}}
\newcommand{\form}[2]{\ensuremath{\left\langle#1,#2\right\rangle}}
\newcommand{\bilin}[2]{\ensuremath{\left(#1,#2\right)}}
\newcommand{\comp}[1]{\ensuremath{#1^\wedge}}
\newcommand{\loca}[3]{\ensuremath{L^{#1}_{#2}(#3)}}
\newcommand{\tc}[3]{\ensuremath{\Omega_{#2 / #1}^{#3}}}
\newcommand{\atc}[3]{\ensuremath{\L_{#2 / #1}^{#3}}}
\newcommand{\pow}[1]{\left\llbracket{#1}\right\rrbracket}

\newcommand{\xym}[1]{
\vskip 0.7pc
\centerline{\xymatrix{#1}}
\vskip 0.7pc
}

\newcommand{\xynm}[1]
{\vskip 0.7pc
\centerline{\xy #1 \endxy}
\vskip 0.7pc}

\begin{document}

\textbf{The standard resolution of $A$ as an $(A,\Gamma)$-comodule}

Let's progress from general to specific.  Fix a Hopf algebroid
$(A,\Gamma)$ with units $\eta_L$ and $\eta_R$ and comultiplication
$\Delta\co \Gamma \to \Gamma \dt_A \Gamma$, and a comodule $N$ over this
Hopf algebroid with structure map (of left $A$-modules) $\phi\co N \to \Gamma \rt_A N$.
(From now on all the tensor products will be implicitly over $A$; if
$\Gamma$ appears on the left side of the tensor it is implicitly using
$\eta_R$, the right unit, as the right module structure, and similarly
using $\eta_L$ if it appears on the opposite side.) Cf def 5.12 on p18 of [coctalos]. In particular, $\phi = \eta_L$ if $N = A$ (see below), and $\phi = \Delta$ if $N = \Gamma$.

We have an augmented cosimplicial object $C(\Gamma,\Gamma,N) = X$ with
\[
X^p = \Gamma \otimes \underbrace{\Gamma \otimes \cdots \otimes \Gamma}_{p\text{
    times}} \otimes N
\]
The coface maps $d^i\co X^p \to X^{p+1}$ for $0 \leq i \leq p+1$ are
given by
\[
d^i = \begin{cases}
1 \otimes \cdots \otimes \Delta \otimes \cdots
\otimes 1 &\text{if }i \leq p\\
1 \otimes \cdots \otimes \phi &\text{if }i = p+1
\end{cases}
\]
For example, when $p=0$ the two maps $\Gamma \otimes N \to \Gamma
\otimes \Gamma \otimes N$ are $d^0 = \Delta \otimes 1$ and $d^1 = 1
\otimes \phi$.  The augmentation of this cosimplicial object $X$ is the
structure map $\phi\co N \to \Gamma \otimes N$ (cf pp274-5 of [ha]; in particular, $d^0\phi = d^1\phi$ by coassociativity).  Codegeneracies are given by applying the augmentation of the Hopf algebroid $\epsilon\co \Gamma \to A$ to one of the middle factors of $\Gamma$.

(If you prefer, a cosimplicial object is supposed to be a functor from
nonempty finite ordered sets.  In this case, if $S = \{0,1,\ldots,p\}$,
then the tensor symbols in $X^p$ are in bijection with the elements of
$S$; this helps me to remember the effect of a general map of ordered
sets.)

We thus get a cobar \textbf{resolution}
\[
0 \to N \to \Gamma \otimes N \to \Gamma \otimes \Gamma \otimes N \to
\cdots
\]
where the coboundary maps are $\phi$, $\Delta \otimes 1 - 1 \otimes
\phi$, $\Delta \otimes 1 \otimes 1 - 1 \otimes \Delta \otimes 1 + 1
\otimes 1 \otimes \phi$, et cetera. (Cf pp274-5 of [ha]. Generally we just get an augmented cochain complex. The exactness comes from the extra degeneracy, ie applying the augmentation $\epsilon\co \Gamma \to A$ to the leftmost factor of $\Gamma$.) If $N$ is $A$ (whose comodule
structure $A \to \Gamma$ is the {\em   left} unit $\eta_L$ because
it's supposed to be a map of left $A$-modules), then you simply
replace $\phi$ with $\eta_L$.
\\
\hrule

Before talking about the specific quadratic Hopf algebroid, let me
just say what happens when we're going to compute Ext.

Applying $\Hom_{(A,\Gamma)}(A,-)$ takes an induced comodule $\Gamma
\otimes N$ to $A \otimes N \cong N$, and has a straightforward effect
on maps - specifically, $f\co \Gamma \otimes N \to \Gamma \otimes N'$
becomes $(\epsilon \otimes 1) \circ f \circ (\eta_R \otimes 1)$. (Lemma 12.4 on p38 of [coctalos] says that we have an adjoint pair $forget \co (A,\Gamma)$-comodules $\leftrightarrow A$-modules $\co \Gamma \otimes_A -$. In particular, $\Hom_{(A,\Gamma)}(A,\Gamma \otimes N) \cong \Hom_A(A,N) \cong N \cong A \otimes N$ (cf Wed-12/23-1). To see $\Hom_{(A,\Gamma)}(A,-)$ takes $\Gamma \otimes N \stackrel{f}{\to} \Gamma \otimes N'$ to $A \otimes N \stackrel{\eta_R \otimes 1}{\to} \Gamma \otimes N \stackrel{f}{\to} \Gamma \otimes N' \stackrel{\epsilon \otimes 1}{\to} A \otimes N'$, we can interpret $\eta_R \otimes 1$ as the functor $\Gamma \otimes_A -$ because $\Gamma$ becomes a right $A$-module via $\eta_R$, and $\epsilon \otimes 1$ as the forgetful functor because of the counital property: $N \stackrel{\phi}{\to} \Gamma \otimes N \stackrel{\epsilon \otimes 1}{\to} N$ is the identity.)
Applying this to the cobar resolution we get
\[
0 \to \Hom_{(A,\Gamma)}(A,N) \to A \otimes N \to A \otimes \Gamma \otimes N \to
\cdots
\]
and we drop the left-hand term (cf p50 of [ha]) and have the cobar complex
\[
0 \to A \otimes N \to A \otimes \Gamma \otimes N \to A \otimes \Gamma
\otimes \Gamma \otimes N \to \cdots,
\]
which computes Ext.

The coface maps, after applying $\Hom_{(A,\Gamma)}(A,-)$, are mostly
the same, with the exception of $\Delta \otimes \cdots \otimes 1$; it
becomes $\eta_R \otimes \cdots \otimes 1$.  So the coboundary maps in
the cobar complex are: $\eta_R \otimes 1 - 1 \otimes \phi$, $\eta_R
\otimes 1 \otimes 1 - 1 \otimes \Delta \otimes 1 + 1 \otimes 1 \otimes
\phi$, et cetera.

There's a lot of redundant notation here when I leave the tensor
factors of $A$ on either side; it's more compact to write the cobar
complex when $N=A$ as
\[
0 \to A \to \Gamma \to \Gamma \otimes \Gamma \to \cdots
\]
with coboundary maps
\begin{eqnarray*}
&&\eta_R - \eta_L\\
&&\eta_R \otimes 1 - \Delta + 1 \otimes \eta_L\\
&&\eta_R \otimes 1 \otimes 1 - \Delta \otimes 1 + 1 \otimes \Delta - 1
\otimes 1 \otimes \eta_L,
\end{eqnarray*}
et cetera.  But it's sometimes handy to remember the implicit factors
of $A$ on either side.  By convention, we use the shorthand $[\gamma_1 |
\cdots | \gamma_p]$ for the representative elements $1 \otimes
\gamma_1 \otimes \cdots \otimes \gamma_p \otimes 1$ in the cobar
complex.  Under these conventions, the coboundary maps are:
\begin{eqnarray*}
&&a \mapsto [\eta_R(a)] - [\eta_L(a)]\\
&&[\gamma] \mapsto [1|\gamma] - [\Delta \gamma] + [\gamma|1]\\
&&[\gamma_1|\gamma_2] \mapsto [1|\gamma_1|\gamma_2] - [\Delta \gamma_1
| \gamma_2] + [\gamma_1 | \Delta \gamma_2 ] - [\gamma_1 | \gamma_2 | 1]
\end{eqnarray*}
Note that the left-right tensor product notation means that you can
move an element $a \in A$ across the tensor product, but it switches
from the left to right units; e.g. $[\gamma_1 | \eta_L(a) \gamma_2] =
[\eta_R(a) \gamma_1 | \gamma_2]$.
\\
\hrule

OK.

Let's specifically talk about the quadratic Hopf algebroid, which has
$A = \mb Z[b,c]$, $\Gamma = A[r]$, $\eta_L\co A \to A[r]$ being the
standard inclusion, $\eta_R(b) = b + 2r$, $\eta_R(c) = c + br + r^2$,
$\Delta(a) = \stackrel{\text{1st}}{a} \otimes \stackrel{\text{2nd}}{1}$ for $a \in A$, $\Delta(r) = r \otimes 1 + 1 \otimes r$. Cf Sat-11/14.

Our cobar resolution is isomorphic to
\[
0 \to \mb Z[b,c] \to \mb Z[b,c,r] \stackrel{d^0}{\to} \mb Z[b,c,r'_1, r'_2] \stackrel{d^1}{\to} \mb Z[b,c,r''_1,r''_2,r''_3]
\to \cdots
\]
where we've collapsed some identifications.  The cobar representatives
are given by $r = r$, $r'_1 = r\otimes 1$, $r'_2 = 1\otimes r$, et cetera.

The coboundary $d^0$ is $\Delta \otimes 1 - 1 \otimes \eta_L$.  This
sends $a$ to $a\otimes 1 - a \otimes 1 = 0$ for $a \in A$, and sends $r$
to $1 \otimes r = r'_2$.  You can calculate the image of any element
but a general formula is kind of annoying to write down.

The coboundary $d^1$ is $\Delta \otimes 1 \otimes 1 - 1 \otimes \Delta
\otimes 1 + 1 \otimes 1 \otimes \eta_L$.  For example, it sends $a
\otimes 1$ to $a \otimes 1 \otimes 1$ (which I called ``$a$'' in the
above identification) for $a \in A$, and $r'_1 = r \otimes 1$ to $r
\otimes 1 \otimes 1 + 1 \otimes r \otimes 1 = r''_1 + r''_2$.

For a similar discussion of the cobar complex, cf Sat-11/14.
\end{document}
